\documentclass[10pt,aspectratio=169]{beamer}
\usepackage{poly}

\usepackage[UTF8]{ctex}
\usepackage{amsmath, amssymb}
\usepackage{graphicx}
\usepackage{booktabs}

%==============================
% Basic Info
%==============================
\title{AI4Medicine:跨域表格预测研究}
\subtitle{PANDA:预训练表格基础模型与TCA对齐的领域适应框架}
\author{刘青沅}
\institute[PolyU]{Department of Computing, The Hong Kong Polytechnic University}
\date{\today}

\begin{document}

%==============================
% Title
%==============================
\maketitle

%==============================
\section{研究背景与问题动机}
%==============================

\begin{frame}{临床背景:肺结节与诊断风险预测}
\begin{itemize}
    \item 低剂量 CT(LDCT)已被证实可降低高危人群肺癌死亡率。
    \item 小肺结节(sPN)诊断不确定性高,\textbf{需要可靠的恶性风险评分}辅助临床决策。
    \item 现有临床风险模型(Mayo、PKUPH 等)具有中心依赖性:
    \begin{itemize}
        \item 在开发中心内部表现较好;
        \item 在跨医院外部验证中性能显著下降。
    \end{itemize}
    \item 现实部署的关键挑战:\textbf{跨医院分布漂移 + 小样本 + 特征异构}。
\end{itemize}
\end{frame}

\begin{frame}{AI 背景:跨医院/跨人群泛化难题}
\begin{itemize}
    \item 医学表格数据普遍面临:
    \begin{itemize}
        \item 样本量有限、类别不均衡;
        \item 特征定义、检测面板与单位存在医院差异;
        \item 采集流程、筛查策略导致分布与先验不同。
    \end{itemize}
    \item 传统树模型与深度表格模型:
    \begin{itemize}
        \item 对小样本敏感,易过拟合;
        \item 在域偏移下泛化不稳定。
    \end{itemize}
    \item 需要一种\textbf{轻量、可解释、可复用}的跨域表格诊断框架。
\end{itemize}
\end{frame}

%==============================
\section{问题定义}
%==============================

\begin{frame}{问题定义:跨医院肺结节恶性预测}
\begin{itemize}
    \item 目标:预测结节恶性风险 $Y \in \{0,1\}$。
    \item 源域(Hospital A):有标签小样本临床表格数据。
    \item 目标域(Hospital B):\textbf{训练阶段无标签},仅用于对齐。
    \item 关键困难:
    \begin{itemize}
        \item 小样本、高方差;
        \item 类别不均衡;
        \item 特征异构:$F^S \neq F^T$;
        \item 分布偏移:检测标准、患者构成、流程差异。
    \end{itemize}
    \item 视角:\textbf{无监督领域适应(UDA)}的跨中心风险预测。
\end{itemize}
\end{frame}

\begin{frame}{设计原则:从理论到可落地方法}
\begin{itemize}
    \item 目标域误差可分解为:
    \begin{itemize}
        \item 源域可学习性(source risk);
        \item 域间分布差异(divergence);
        \item 可迁移性/共享结构(adaptability)。
    \end{itemize}
    \item 对应策略:
    \begin{itemize}
        \item 用\textbf{预训练表格基础模型}提升小样本学习稳定性;
        \item 用\textbf{统计对齐}缩小跨域差异;
        \item 用\textbf{跨域稳定特征选择}降低负迁移风险。
    \end{itemize}
\end{itemize}
\end{frame}

%==============================
\section{核心思想与贡献}
%==============================

\begin{frame}{PANDA 核心思想}
\begin{itemize}
    \item PANDA = \textbf{“选稳定特征 + 对齐分布 + 用预训练模型预测”}。
    \item 在共享特征空间中构建跨医院鲁棒风险映射。
    \item 以轻量统计对齐连接真实临床场景与表格基础模型能力。
    \item 面向可部署的跨中心诊断决策支持。
\end{itemize}
\end{frame}

\begin{frame}{主要贡献(Contributions)}
\begin{itemize}
    \item 提出 PANDA:融合\textbf{跨域 RFE、TCA 对齐与 TabPFN}的 UDA 框架。
    \item 设计跨域稳定特征筛选策略,得到紧凑共享特征集合。
    \item 在两家癌症中心的肺结节任务上验证跨医院增益。
    \item 在 TableShift BRFSS Diabetes 任务上验证跨人群稳健性。
    \item 提供可复用的实现与分析范式,连接理论动机与临床部署需求。
\end{itemize}
\end{frame}

%==============================
\section{方法:PANDA 框架}
%==============================

\begin{frame}{PANDA 框架形式化}
\begin{itemize}
    \item 复合函数定义:
    \[
    f_{\text{PANDA}}(\mathbf{x}) = h \circ \psi \circ \pi_{\cap} \circ \pi_{\text{RFE}}(\mathbf{x})
    \]
    \item $\pi_{\text{RFE}}$:跨域 RFE 选出判别性子集 $\mathcal{F}^\ast$。
    \item $\pi_{\cap}$:强制源/目标 schema 交集,保证目标域可用。
    \item $\psi$:TCA 进行域对齐,减小源/目标分布差异。
    \item $h$:TabPFN 集成分类 + 温度校准,输出鲁棒概率。
\end{itemize}
\end{frame}

\begin{frame}{PANDA 流程示意}
\begin{figure}
    \centering
    \includegraphics[width=0.76\linewidth]{img/cross_hospital/Feature Selection and UDA.pdf}
    \caption{Pipeline aligned with论文结构:RFE 先优化可迁移子空间,随后 schema 交集 $\pi_{\cap}$、TCA 映射 $\psi$,最终 TabPFN 集成分类器 $h$。}
    \label{fig:panda-pipeline}
\end{figure}
\end{frame}

\begin{frame}{Stage 1:跨域 RFE($\pi_{\text{RFE}}$)}
\begin{itemize}
    \item 目标:学习特征子集 $\mathcal{F}^\ast$,满足
    \begin{itemize}
        \item \textbf{在目标医院 schema 中可获得};
        \item \textbf{在源域上具有强判别能力}。
    \end{itemize}
    \item TabPFN-based permutation importance 在源域交叉验证中为特征打分。
    \begin{itemize}
        \item 候选子集大小 $k \in [k_{\min}, k_{\max}]$;
        \item \textbf{复合评价指数(CEI)}平衡判别力、计算效率、fold 稳定性与简约性;
        \item 取 $k^\ast$ 最大化 CEI,得到 $\mathcal{F}^\ast$(本实验 $|\mathcal{F}^\ast|=9$)。
    \end{itemize}
    \item 详尽实现与 CEI 组成见论文 Section 5.2.1。
\end{itemize}
\end{frame}

\begin{frame}{Stage 2:特征对齐($\pi_{\cap}$)}
\begin{itemize}
    \item RFE 产出 $|\mathcal{F}^\ast| = 9$,因目标域缺失,交集约束为
    \[
        \mathcal{F}_{\cap} = \mathcal{F}^\ast \cap \mathcal{F}^T
    \]
    \item 维度从 9 减至 8,显式处理目标医院未检测的生物标志物。
    \item 确保后续域对齐与分类仅在\textbf{临床可用的共享子空间}上进行。
\end{itemize}
\end{frame}

\begin{frame}{Stage 3:域适配映射($\psi$,TCA)}
\begin{itemize}
    \item 对齐后的特征 $\mathcal{F}_{\cap}$ 送入 TCA,学习线性映射:
    \[
        \mathbf{x}' = \psi(\mathbf{x}; \mathcal{F}_{\cap}) = \mathbf{W}^\top \mathbf{x}
    \]
    \item 在线性核 RKHS 中最小化源/目标的 MMD,\textbf{降低 $\mathcal{H}\Delta\mathcal{H}$-divergence}。
    \begin{itemize}
        \item $\mathbf{W}$ 在保持方差的约束下优化;
        \item 以未标注目标数据参与对齐,符合隐私/标签缺失场景。
    \end{itemize}
\end{itemize}
\end{frame}

\begin{frame}{Stage 4:分类与集成($h$,TabPFN)}
\begin{itemize}
    \item 适配后的特征 $\mathbf{x}'$ 经 TabPFN 集成分类:
    \[
        h(\mathbf{x}') = \frac{1}{N} \sum_{i=1}^{N} \sigma\!\left(\frac{f_i^{\text{TabPFN}}(\mathbf{x}')}{\tau}\right)
    \]
    \item 温度参数 $\tau$ 用于校准不同医院的患病率差异。
    \item 集成平滑不确定性,提高对协变量/概念漂移的鲁棒性。
    \item PANDA 在文中分别评估:
    \begin{itemize}
        \item 肺结节实验:主要是 covariate 与 concept shift;
        \item TableShift Diabetes:显式考察 label shift/患病率差异。
    \end{itemize}
\end{itemize}
\end{frame}

%==============================
\section{数据集与实验设置}
%==============================

\begin{frame}{数据集 1:跨医院肺结节队列}
\begin{itemize}
    \item 任务:预测肺结节恶性风险(结构化临床表格)。
    \item 源域 Hospital A vs 目标域 Hospital B。
    \item 特征:
    \begin{itemize}
        \item 临床人口学、实验室指标、影像语义/临床整理特征等;
        \item 存在特征缺失与面板差异,需取共享 schema。
    \end{itemize}
\end{itemize}
\begin{table}
    \centering
    \caption{Training (Cohort A) and testing (Cohort B) cohorts.}
    \label{tab:cohort_summary_slide}
    \scriptsize
    \begin{tabular}{lcc}
        \toprule
        \textbf{Characteristic} & \textbf{Cohort A (n = 295)} & \textbf{Cohort B (n = 190)} \\
        \midrule
        Upper lobe (Yes) & 41.0\% & 51.6\% \\
        Age (years) & 56.95 $\pm$ 11.03 & 58.26 $\pm$ 9.57 \\
        DLCO1 & 5.90 $\pm$ 2.89 & 6.31 $\pm$ 1.55 \\
        VC & 3.33 $\pm$ 0.80 & 2.92 $\pm$ 0.73 \\
        CEA & 4.23 $\pm$ 6.90 & 4.15 $\pm$ 10.61 \\
        Malignant (Yes) & 64.1\% & 65.8\% \\
        \bottomrule
    \end{tabular}
\end{table}
\end{frame}

\begin{frame}{数据集 2:TableShift BRFSS Diabetes}
\begin{itemize}
    \item 公共基准:BRFSS Diabetes(TableShift)。
    \item 训练/测试划分:
    \begin{itemize}
        \item ID:White respondents;
        \item OOD:non-White respondents。
    \end{itemize}
    \item Shift 类型:\textbf{race-driven population shift}。
    \item 目的:验证 PANDA 在\textbf{跨人群}分布变化下的稳健性。
\end{itemize}
\begin{table}
    \centering
    \caption{Source (ID) vs target (OOD) cohorts for the BRFSS Diabetes race shift.}
    \label{tab:tableshift_cohort_summary_slide}
    \scriptsize
    \begin{tabular}{p{0.42\linewidth}p{0.26\linewidth}p{0.26\linewidth}}
        \toprule
        \textbf{Characteristic} & \textbf{ID (White)} & \textbf{OOD (Non-White)} \\
        \midrule
        Sample size (full) & Train 969{,}229; Val 121{,}154; Test 121{,}154 & Val 23{,}264; Test 209{,}375 \\
        Diabetes rate & 12.5\% (train) & 17.4\% (ood\_test) \\
        Domain shift var. & PRACE1 = 1 (non-Hispanic White) & PRACE1 $\in \{2,3,4,5,6\}$ (other races) \\
        Modeling sample (seed 42) & 1{,}024 (train) & 2{,}048 (eval) \\
        Sampled diabetes rate & 13.2\% & 17.3\% \\
        Sampled pos / neg & 135 / 889 & 355 / 1{,}693 \\
        \bottomrule
    \end{tabular}
\end{table}
\end{frame}

\begin{frame}{实验协议与实现要点}
\begin{itemize}
    \item 源域:交叉验证评估内部性能。
    \item 目标域:
    \begin{itemize}
        \item 训练阶段不使用标签(UDA 设置);
        \item 测试阶段用于外部/ OOD 评估。
    \end{itemize}
    \item 关键模块:
    \begin{itemize}
        \item RFE 的稳定性筛选;
        \item TCA 的核空间对齐;
        \item TabPFN ensemble 预测与校准。
    \end{itemize}
\end{itemize}
\end{frame}

%==============================
\section{对比方法与评价指标}
%==============================

\begin{frame}{对比基线(Baselines)}
\begin{itemize}
    \item 临床评分:
    \begin{itemize}
        \item Mayo、PKUPH 等传统风险模型。
    \end{itemize}
    \item 经典机器学习:
    \begin{itemize}
        \item Logistic / LASSO、SVM;
        \item Random Forest、GBDT、XGBoost 等树集成。
    \end{itemize}
    \item PANDA 变体:
    \begin{itemize}
        \item TabPFN-only(含 RFE,去掉 TCA 对齐);
        \item PANDA(含 RFE + TCA + TabPFN,全管线)。
    \end{itemize}
\end{itemize}
\end{frame}

\begin{frame}{评价指标(核心公式)}
\begin{itemize}
    \item 分类与临床指标:
    \[
        \text{Accuracy} = \frac{TP + TN}{TP+TN+FP+FN},\quad
        \text{Precision} = \frac{TP}{TP+FP}
    \]
    \[
        \text{Recall} = \frac{TP}{TP+FN},\quad
        \text{F1} = \frac{2 \cdot \text{Prec} \cdot \text{Rec}}
        {\text{Prec} + \text{Rec}}
    \]
    \item AUC:
    \[
        \text{AUC} = \int_{0}^{1} \text{TPR}(\tau)\, d\,\text{FPR}(\tau),
        \quad \text{TPR}=\tfrac{TP}{TP+FN},\ \text{FPR}=\tfrac{FP}{FP+TN}
    \]
\end{itemize}
\end{frame}

\begin{frame}{校准与决策曲线}
\begin{itemize}
    \item 校准曲线:分桶比较 $\hat{p}$ 与观测频率,越接近对角线越好(概率越可信),偏离表示过/欠估计风险。
    \item DCA:净获益
    \[
        \text{NB}(\tau) = \frac{TP(\tau)}{n} - \frac{FP(\tau)}{n} \cdot \frac{\tau}{1-\tau}
    \]
    \item 参考策略:Treat-all $\text{NB}_{\text{all}}(\tau)$ 与 Treat-none (0);曲线越高、覆盖阈值范围越宽,临床效用越好。
    \item 判读指南:校准看“贴线”,DCA 看“高于基线的面积/高度”,选择外部敏感性优先的阈值段。
\end{itemize}
\end{frame}

%==============================
\section{实验结果}
%==============================

\begin{frame}{实验 1:肺结节(跨医院 UDA)}
\begin{itemize}
    \item 跨医院 UDA(Cohort A $\rightarrow$ Cohort B),特征异构+分布漂移。
    \item PANDA(RFE+TCA+TabPFN):外部 AUC 0.7046,Recall 0.9440,内部 10-fold AUC 0.8287。
    \item TabPFN-only(含 RFE,无 TCA):外部 AUC 0.6980,Recall 0.8880。
    \item 传统树/临床基线跨域 AUC 多数 < 0.65,召回明显下降。
\end{itemize}
\end{frame}

\begin{frame}{实验 1:综合分析图}
\begin{figure}
    \centering
    \includegraphics[width=0.42\linewidth]{img/cross_hospital/combined_analysis_figure.pdf}
    \caption{\textbf{Performance and utility across source/target.} \textbf{a,b} ROC(Cohort A/B);\textbf{c,d} 校准;\textbf{e,f} 决策曲线,展示 TCA 提升外部敏感性与净获益。}
\end{figure}
\end{frame}

\begin{frame}{实验 1:内外部热力图}
\begin{figure}
    \centering
    \includegraphics[width=0.35\linewidth]{img/cross_hospital/combined_heatmaps_nature.pdf}
    \caption{\textbf{Performance comparison across domains.} \textbf{a} 源域 10-fold 热力图;\textbf{b} 外部验证热力图,PANDA+TCA 在 AUC/Recall 上领先。}
\end{figure}
\end{frame}

\begin{frame}{实验 2:TableShift BRFSS Diabetes(跨人群 OOD)}
\begin{itemize}
    \item 跨人群 OOD(White $\rightarrow$ Non-White)。
    \item PANDA + TCA:OOD AUC 0.804,几乎不劣于 ID AUC 0.802。
    \item TabPFN-only 亦保持稳定,传统树模型 OOD 性能下降更明显。
    \item 说明 PANDA 在公共基准上具有\textbf{可迁移性与可复现性}。
\end{itemize}
\end{frame}

\begin{frame}{实验 2:ROC/校准/决策曲线}
    \begin{figure}
        \centering
        \includegraphics[width=0.42\linewidth]{img/tableshift/combined_analysis_figure.pdf}
        \caption{\textbf{TableShift BRFSS Diabetes analysis.} \textbf{a,b} ROC;\textbf{c,d} 校准;\textbf{e,f} 决策曲线,展示 PANDA 在种族驱动分布漂移下的稳健性。}
    \end{figure}
    \end{frame}

\begin{frame}{实验 2:热力图}
\begin{figure}
    \centering
    \includegraphics[width=0.35\linewidth]{img/tableshift/combined_heatmaps_nature.pdf}
    \caption{\textbf{Performance comparison on TableShift.} 训练/种族迁移划分的多指标热力图,PANDA+TCA 维持稳定表现。}
\end{figure}
\end{frame}

%==============================
\section{讨论与结论}
%==============================

\begin{frame}{讨论:优势与局限}
\begin{itemize}
    \item 优势:
    \begin{itemize}
        \item \textbf{预训练 TabPFN 作为稳健先验}:小样本($n<300$)显著优于树模型。
        \item \textbf{跨域 RFE}:收敛到 8 个稳定特征,降低维度并抑制站点特异性噪声。
        \item \textbf{TCA 对齐}:减小 MMD,AUC 提升约 +0.007,并改善外部校准。
    \end{itemize}
    \item 局限:
    \begin{itemize}
        \item \textbf{封闭特征交集}:无法利用目标新采集的强预测因子。
        \item \textbf{未建模 MNAR}:完整案例策略依赖 MCAR/MAR;MNAR 下可能有选择偏倚。
        \item \textbf{资源约束}:TabPFN 在 CPU 也可运行但较慢;GPU 可达 $\sim$20ms/例,$O(N^2)$ 注意力限制大样本;肺结节实验非低患病率标签移位场景。
    \end{itemize}
\end{itemize}
\end{frame}

\begin{frame}{总结与 take-home messages}
\begin{itemize}
    \item PANDA = \textbf{预训练先验 + 稳定特征 + 线性对齐}。
    \item 肺结节(A$\rightarrow$B):内部 AUC 0.8287;外部 AUC 0.7046,Recall 0.944,优于临床评分/树模型。
    \item TableShift Diabetes:OOD AUC 0.804,接近 ID AUC 0.802。
    \item 为跨医院、跨人群表格预测提供可复用的轻量 UDA 模板。
\end{itemize}
\end{frame}

\begin{frame}{未来工作}
\begin{itemize}
    \item \textbf{联邦/协同对齐}:仅共享二阶统计学习 TCA,保护隐私。
    \item \textbf{多模态 PANDA}:融合 CT 影像与表格,跨模态对齐。
    \item \textbf{低患病率/真实筛查}队列验证,补足标签移位应对。
    \item \textbf{资源优化}:轻量化推理、稀疏注意力,扩展到边缘设备。
\end{itemize}
\end{frame}

%==============================
\section{致谢}
%==============================

\begin{frame}{致谢}
\begin{itemize}
    \item 感谢导师:Prof. Wenqi Fan 的指导与支持。
    \item 感谢合作医院与临床团队的数据支持与专业建议。
    \item 感谢课题组同学与朋友的讨论与帮助。
    \item 感谢 PolyU 计算机学院(COMP)与相关项目/资源支持。
\end{itemize}
\end{frame}

\begin{frame}{谢谢!}
\begin{center}
\LARGE 谢谢各位老师聆听 \\
\vspace{1em}
\Large 欢迎提问
\end{center}
\end{frame}

\end{document}
