\documentclass{article} 
% ✅ Standard class file (article.cls) --- conforms to: 
% ``Please use any of the standard class files such as article.cls, revtex.cls or amsart.cls.''

% Language setting
% Replace `english' with e.g. `spanish' to change the document language
\usepackage[english]{babel} 
% ✅ Standard language package --- allowed

% Set page size and margins
% Replace `letterpaper' with `a4paper' for UK/EU standard size
\usepackage[letterpaper,top=2cm,bottom=2cm,left=3cm,right=3cm,marginparwidth=1.75cm]{geometry}
% ✅ Geometry used to control margins --- acceptable since visual formatting is not essential
% ``There is no need to spend time visually formatting the manuscript...''

% Useful packages
\usepackage{amsmath, amssymb}
% ✅ Standard math support --- allowed

\usepackage{amsthm}
% ✅ Required for theorem-like environments (proposition, lemma, etc.)
\newtheorem{theorem}{Theorem}
\newtheorem{lemma}[theorem]{Lemma}
\newtheorem{proposition}[theorem]{Proposition}
\newtheorem{corollary}[theorem]{Corollary}
\newtheorem{definition}{Definition}

% List of Figures and Tables formatting
\usepackage{tocloft}
\renewcommand{\cftfigpresnum}{Figure~}
\setlength{\cftfignumwidth}{4em}

\renewcommand{\cfttabpresnum}{Table~}
\setlength{\cfttabnumwidth}{4em}


\usepackage{graphicx} 
% ✅ Required for figures --- conforms to:
% ``For graphics, we recommend graphicx.sty.''
% ⚠️ Submit a single .tex file only --- all figures should be referenced using \includegraphics,
% and all content must reside in this file or included in-line.
% → ``All textual material should be provided as a single file...''

\usepackage{algorithm}
\usepackage{algorithmic}
% ✅ Required for algorithm environments

\usepackage{multirow}
\usepackage{booktabs}
\usepackage{longtable}
% ✅ Required for table formatting

\usepackage[colorlinks=true, allcolors=blue]{hyperref}
% ✅ Optional, allowed if needed; bookmarks or PDF styling is not required

% ⚠️ Do NOT use any custom macros or commands
% → “Remove all personal macros before submitting.”

% ⚠️ Do NOT use non-standard fonts (e.g., \usepackage{times}, \usepackage{helvet}, etc.)
% → “All textual material should be provided as a single file in default Computer Modern fonts.”

% ✅ Ensure this file compiles without errors or warnings before submission
% → “Please ensure that the complete .tex file compiles successfully on your own system...”

\usepackage{xcolor}
\newcommand{\red}[1]{{\color{red}#1}} % comments
\definecolor{lbcolor}{RGB}{13, 151, 175}
\newcommand{\lb}[1]{{\color{lbcolor}#1}} % comments
\providecommand{\toprule}{\hline}
\providecommand{\midrule}{\hline}
\providecommand{\bottomrule}{\hline}

\title{Transforming Diagnosis through Advanced Machine Learning and Data Analytics}

\author{
    Qingyuan Liu\textsuperscript{1}
}

\date{
    \textsuperscript{1}Department of Computing, The Hong Kong Polytechnic University, Hong Kong SAR, China
}

% notation.tex
% Unified Mathematical Notation System

% Define common symbols and their meanings here.

% Domains and Data
\newcommand{\inputspace}{\mathcal{X}} % Input feature space (mixed numerical/categorical)
\newcommand{\labelspace}{\mathcal{Y}} % Label space (0: Benign, 1: Malignant)
\newcommand{\sourcedomaindist}{\mathcal{D}_S} % Source domain distribution
\newcommand{\targetdomaindist}{\mathcal{D}_T} % Target domain distribution
\newcommand{\sourcedata}{S} % Empirical dataset from D_S
\newcommand{\targetdata}{T} % Empirical dataset from D_T
\newcommand{\featurevec}{\mathbf{x}} % Feature vector
\newcommand{\labelval}{y} % Corresponding label
\newcommand{\sharedfeatures}{\mathcal{F}_{\cap}} % Shared feature schema
\newcommand{\rfeselectedfeatures}{\mathcal{F}^*} % RFE-selected subset of features
\newcommand{\sourcedatasize}{n_s} % Number of labeled samples in source
\newcommand{\targetdatasize}{n_t} % Number of unlabeled samples in target
\newcommand{\ns}{n_s} % Alternative notation for source domain size
\newcommand{\nt}{n_t} % Alternative notation for target domain size
\newcommand{\rawdim}{d_{\text{raw}}} % Raw input feature dimensionality
\newcommand{\intersectdim}{d_{\cap}} % Intersected feature dimensionality
\newcommand{\featuredim}{d} % Dimensionality of input feature space
\newcommand{\sourcedatafeatures}{X_s} % Source data features
\newcommand{\sourcedatalabels}{y_s} % Source data labels
\newcommand{\targetfeatureschema}{\mathcal{F}_t} % Target schema (set of available features in target domain)
\newcommand{\targetfeaturecount}{k_{\text{target}}} % Target feature count
\newcommand{\targetlabels}{Y_T} % Target labels (unobserved)
\newcommand{\featurevecj}{\featurevec_j} % j-th component of feature vector
\newcommand{\featurevecs}{\featurevec^s} % Feature vector from source
\newcommand{\featurevect}{\featurevec^t} % Feature vector from target
\newcommand{\labelvalq}{y_{\text{query}}} % Query label

% Learning Theory
\newcommand{\hypothesis}{h} % Hypothesis function (classifier)
\newcommand{\hypothesisclass}{\mathcal{H}} % Hypothesis class
\newcommand{\deepmodelclass}{\mathcal{H}_{\text{deep}}} % Deep Neural Network Hypothesis Class
\newcommand{\Hdeep}{\mathcal{H}_{\text{deep}}} % Alternative notation for deep model class
\newcommand{\sourceerror}{\epsilon_S(\hypothesis)} % Expected risk (error) on Source distribution
\newcommand{\targeterror}{\epsilon_T(\hypothesis)} % Expected risk (error) on Target distribution
\newcommand{\sourcesampleerror}{\hat{\epsilon}_S(\hypothesis)} % Empirical risk on dataset S
\newcommand{\targetsampleerror}{\hat{\epsilon}_T(\hypothesis)} % Empirical risk on dataset T
\newcommand{\domaindivergence}{d_{\mathcal{H}\Delta\mathcal{H}}} % HDeltaH-Divergence (Domain Discrepancy)
\newcommand{\adaptabilityterm}{\lambda} % Ideal joint hypothesis error (Adaptability term)
\newcommand{\powertransformparam}{\lambda} % Parameter for power transform (e.g., Box-Cox)
\newcommand{\threshold}{\theta} % Threshold (e.g., in GBDT)
\newcommand{\featuremap}{\varphi} % Feature map to RKHS
\newcommand{\rkhs}{\mathcal{H}_{RKHS}} % Reproducing Kernel Hilbert Space
\newcommand{\hypothesissubset}{h_{S^{(t)}}} % Hypothesis trained on feature subset
\newcommand{\featurevectexture}{\featurevec_{\text{texture}}} % Texture feature vector component
\newcommand{\vcdim}{d_{VC}} % VC Dimension
\newcommand{\confidence}{\delta} % Confidence parameter
\newcommand{\loss}{\mathcal{L}} % Generic Loss function
\newcommand{\modelparams}{\boldsymbol{\theta}} % Model parameters (weights)
\newcommand{\Sperf}{\mathcal{S}_{\text{perf}}} % Performance-optimal hypothesis class
\newcommand{\Seff}{\mathcal{S}_{\text{eff}}} % Efficiency-optimal hypothesis class
\newcommand{\Sstab}{\mathcal{S}_{\text{stab}}} % Stability-optimal hypothesis class
\newcommand{\Ssimp}{\mathcal{S}_{\text{simp}}} % Simplicity-optimal hypothesis class

% PANDA Architecture
\newcommand{\adaptmap}{\psi} % Domain adaptation mapping (TCA projection)
\newcommand{\tcaprojectionmatrix}{\mathbf{W}} % TCA Projection Matrix
\newcommand{\kernelmatrix}{\mathbf{K}} % Kernel Matrix
\newcommand{\mmdmatrix}{\mathbf{L}} % MMD Indicator Matrix
\newcommand{\centeringmatrix}{\mathbf{H}} % Centering Matrix
\newcommand{\regularizationparam}{\mu} % Regularization parameter for TCA
\newcommand{\ppd}{PPD} % Posterior Predictive Distribution
\newcommand{\priorfunc}{P_{\text{prior}}} % Prior distribution over functions
\newcommand{\func}{f} % Structural equation model or data-generating function
\newcommand{\noise}{\epsilon} % Noise term
\newcommand{\queryvec}{\mathbf{x}_{\text{query}}} % Query sample feature vector
\newcommand{\sourcequeryvec}{\mathbf{x}_{\text{query}}^s} % Query sample from source domain
\newcommand{\targetqueryvec}{\mathbf{x}_{\text{query}}^t} % Query sample from target domain
\newcommand{\marginalsourcedist}{P_S(X)} % Marginal feature distribution for source
\newcommand{\marginaltargetdist}{P_T(X)} % Marginal feature distribution for target
\newcommand{\conditionalps}{P_S(Y|X)} % Conditional probability of label given features for source
\newcommand{\conditionalpt}{P_T(Y|X)} % Conditional probability of label given features for target
\newcommand{\marginalsourcelabel}{P_S(Y)} % Marginal label distribution for source
\newcommand{\marginaltargetlabel}{P_T(Y)} % Marginal label distribution for target
\newcommand{\prevalence}{\pi} % Prevalence of positive class
\DeclareMathOperator*{\argmin}{argmin} % Argmin operator
\DeclareMathOperator*{\argmax}{argmax} % Argmax operator
\DeclareMathOperator{\tr}{tr} % Trace operator
\newcommand{\identitymatrix}{\mathbf{I}} % Identity matrix
\newcommand{\rfeimportance}{\mathcal{I}} % Permutation Importance
\newcommand{\lossauc}{\mathcal{L}_{\text{AUC}}} % AUC Loss
\newcommand{\featureimportancevec}{\mathbf{I}} % Permutation Importance vector
\newcommand{\minfeature}{f_{\text{min}}} % Feature with minimum importance
\newcommand{\rfeset}{S} % Base symbol for feature subsets
\newcommand{\rfeiterateset}{\rfeset^{(t)}} % Feature subset at iteration t
\newcommand{\rfecurrentdim}{k} % Dimensionality after RFE (approx. 8)
\newcommand{\tcaprojectdim}{m} % Dimensionality after TCA (approx. 15)
\newcommand{\ensembleprob}{\hat{P}(y=1|\mathbf{x}_q)} % Ensemble probability estimate
\newcommand{\querypatient}{\mathbf{x}_q} % Target query (patient)
\newcommand{\contextdata}{\mathbf{X}_{\text{ctx}}^s} % Subset of training data selected by seed s
\newcommand{\transformerbackbone}{f_\theta} % Frozen TabPFN Transformer backbone
\newcommand{\temperature}{T} % Temperature scaling parameter
\newcommand{\logitoutput}{z_i(\mathbf{x})} % Logit output of i-th TabPFN classifier member
\newcommand{\activation}{\sigma} % Activation function (e.g., sigmoid)
\newcommand{\pandafunc}{f_{\text{PANDA}}} % PANDA composite function
\newcommand{\tabpfnfunc}{f_{\text{TabPFN}}} % TabPFN classifier function
\newcommand{\tabpfnfunci}[1]{f^{\text{TabPFN}}_{#1}} % i-th TabPFN classifier function (ensemble member)
\newcommand{\tabpfnencoder}{\Phi} % TabPFN Transformer Encoder (Capital Phi to distinguish from RKHS map phi)
\newcommand{\ceilogit}{l_i} % Logit for CEI calculation (generic)
\newcommand{\numfeatureembed}{\mathbf{e}^{(j)}} % Embedding for j-th numerical feature
\newcommand{\featureembed}{\mathbf{e}} % General feature embedding vector
\newcommand{\numweightmat}{\mathbf{W}_{\text{num}}^{(j)}} % Weight matrix for j-th numerical feature
\newcommand{\numbiasvec}{\mathbf{b}_{\text{num}}^{(j)}} % Bias vector for j-th numerical feature
\newcommand{\catelembedmat}{\text{EmbeddingMatrix}^{(j)}} % Embedding matrix for j-th categorical feature
\newcommand{\catindex}{c} % Categorical feature index
\newcommand{\nanembed}{\mathbf{e}_{\text{nan}}} % Learnable token for missing values
\newcommand{\samplesembed}{\mathbf{E}_{\text{sample}}} % Full sample embedding
\newcommand{\posencoding}{\mathbf{P}_{\text{pos}}} % Positional encoding
\newcommand{\mlp}{\text{MLP}} % Multi-Layer Perceptron
\newcommand{\concat}{\text{Concat}} % Concatenation operation
\newcommand{\preprocessingbranches}{\mathcal{B}} % Set of preprocessing functions
\newcommand{\randomseeds}{\mathcal{R}} % Set of random seeds
\newcommand{\numpreprocessbranches}{B} % Number of preprocessing branches
\newcommand{\numrandomseeds}{R} % Number of random seeds (size of set R)
\newcommand{\rfecomponentweights}{w} % Generic weight for RFE components
\newcommand{\rfeperformance}{S_{\text{perf}}} % Performance score for RFE
\newcommand{\rfeefficiency}{S_{\text{eff}}} % Efficiency score for RFE
\newcommand{\rfestability}{S_{\text{stab}}} % Stability score for RFE
\newcommand{\rfesimplicity}{S_{\text{simp}}} % Simplicity score for RFE
\newcommand{\featsetcost}{\text{Cost}(\mathcal{F}_k)} % Cost of feature subset
\newcommand{\totalfeatcost}{\text{Cost}(\mathcal{F}_{\text{total}})} % Cost of total features
\newcommand{\aucsd}{\text{Std}(\text{AUC})} % Standard deviation of AUC
\newcommand{\sparsityparam}{\alpha} % Sparsity penalty coefficient
\newcommand{\linearK}{K_{\text{linear}}} % Linear Kernel
\newcommand{\rbfK}{K_{\text{RBF}}} % RBF Kernel
\newcommand{\gammaK}{\gamma} % Bandwidth hyperparameter for RBF kernel
\newcommand{\onesvec}{\mathbf{1}} % Vector of ones
\newcommand{\rfeop}{\pi_{\text{RFE}}} % RFE operation that restricts to selected features

% Evaluation Metrics
\newcommand{\tpr}{\text{TPR}} % True Positive Rate
\newcommand{\fpr}{\text{FPR}} % False Positive Rate
\newcommand{\tp}{\text{TP}} % True Positives
\newcommand{\fn}{\text{FN}} % False Negatives
\newcommand{\fp}{\text{FP}} % False Positives
\newcommand{\tn}{\text{TN}} % True Negatives
\newcommand{\auc}{\text{AUC}} % Area Under the Curve
\newcommand{\accuracy}{\text{Accuracy}} % Accuracy
\newcommand{\precision}{\text{Precision}} % Precision
\newcommand{\recall}{\text{Recall}} % Recall (Sensitivity)
\newcommand{\fonescore}{\text{F1 Score}} % F1 Score
\newcommand{\specificity}{\text{Specificity}} % Specificity
\newcommand{\meanmetric}{\bar{M}} % Mean of metric M
\newcommand{\metrick}{M_k} % Metric M for k-th fold
\newcommand{\numfolds}{K} % Number of folds
\newcommand{\stdmetric}{\sigma_M} % Standard deviation of metric M
\newcommand{\predprob}{\hat{p}_i} % Predicted probability for i-th sample
\newcommand{\calibbin}{B_k} % k-th bin for calibration curve
\newcommand{\meanpredprob}{\bar{p}_k} % Mean predicted probability in bin B_k
\newcommand{\meanobservedfreq}{\bar{y}_k} % Mean observed frequency in bin B_k
\newcommand{\netbenefit}{\text{NB}(p_t)} % Net Benefit at probability threshold p_t
\newcommand{\probthreshold}{p_t} % Probability threshold
\newcommand{\nball}{\text{NB}_{\text{all}}} % Net Benefit of "treat all" strategy
\newcommand{\nbnone}{\text{NB}_{\text{none}}} % Net Benefit of "treat none" strategy


\begin{document}
\maketitle

\begin{abstract}
Accurate diagnostic risk prediction increasingly depends on machine learning models that can operate reliably across hospitals, populations, and data-collection protocols. In practice, however, clinical prediction models are often trained on small, imbalanced tabular cohorts with heterogeneous feature schemas and substantial distribution shift, so performance deteriorates once they leave the development site. To address this gap, we propose PANDA (Pretrained Adaptation Network with Domain Alignment), a framework that combines a pre-trained tabular foundation model with domain-aware feature selection and unsupervised alignment to stabilize diagnostic decision support. PANDA uses a TabPFN-style transformer backbone meta-trained on synthetic tabular tasks to provide data-efficient priors; a cross-cohort recursive feature elimination step identifies a compact set of shared biomarkers that remain predictive across sites; and Transfer Component Analysis (TCA) projects source and target cohorts into a shared latent space using unlabeled target data, mitigating covariate shift under strict privacy constraints.

We evaluate PANDA in two representative diagnostic settings. For cross-hospital pulmonary
nodule malignancy prediction using structured clinical data from two Chinese cancer centers,
PANDA achieves an internal AUC of 0.811 and, with TCA, attains an external AUC of 0.705
and recall of 0.944 on the target hospital, outperforming tree ensembles and classical clinical
scores such as Mayo and PKUPH (AUC < 0.64). In the TableShift BRFSS Diabetes benchmark
with a race-driven shift from White to non-White respondents, PANDA with TCA reaches an
out-of-distribution AUC of 0.804, comparable to its in-distribution AUC of 0.802, while standard
gradient boosting and random forests exhibit larger degradation. Together, these results indicate
that pairing tabular foundation-model priors with cross-domain feature pruning and lightweight
statistical alignment can make diagnostic machine learning systems more robust and data-efficient
in realistic cross-hospital and cross-population deployments.

\end{abstract}

\clearpage

INSERT\_TOC\_HERE

\clearpage


\clearpage

\section{Introduction}
\label{sec:intro-start}

Accurate diagnostic risk prediction is a canonical setting in which advances in machine learning (ML) and data analytics can have immediate clinical impact. In pulmonary nodule screening, classical clinical risk scores such as the Mayo Clinic, Veterans Affairs, Brock (PanCan), PKUPH, and Li models achieve strong internal discrimination (AUC $\approx 0.80$--$0.94$) by fitting logistic regressions to carefully curated, single-center cohorts~\cite{swensen1997chest,mcwilliams2013probability,li2011development,he2021novel,zhang_comprehensive_2022,liu_establishment_2024}. However, meta-analyses and external validations show that their performance can decline to AUCs of $0.60$--$0.75$ when transported to community-screening sites, TB-endemic regions, or demographically distinct populations~\cite{garau_external_2020,zhang_comprehensive_2022,liu_establishment_2024}. These degradations, driven by shifts in disease prevalence, acquisition protocols, and background pathology (e.g., granulomas versus tuberculosis), illustrate how non-adaptive risk calculators can become unreliable in cross-hospital practice.

From an AI perspective, these failures reflect a mismatch between the complexity of real-world deployment and the simplifying assumptions of classical supervised learning. Clinical tabular datasets are typically small, imbalanced, and heterogeneous: even high-value registries often contain only a few hundred labeled patients, with malignant nodules representing a minority class. Features are high-dimensional and only partially overlapping across sites, as institutions log different biomarker panels, coding schemes, and acquisition protocols. This combination of small sample size, distribution shift, and feature-space mismatch violates the closed-world assumptions underlying many standard models and exposes the limits of purely local training.

The algorithmic trajectory for structured data in healthcare mirrors this tension. Gradient-boosted decision trees (GBDTs), led by XGBoost, LightGBM, and related ensembles, remain the workhorses of tabular ML because they tolerate heterogeneous scales, missingness, and noisy categorical codes~\cite{chen2016xgboost,gorishniy2021revisiting}. Neural ``deep tabular'' architectures—including TabNet, TabTransformer, SAINT, FT-Transformer, NODE, and other attention- or gating-based variants—extend differentiability to structured data and enable multimodal fusion, but they require substantial data, are sensitive to hyperparameters, and often lag well-tuned trees on clinical benchmarks when effective sample sizes are small~\cite{arik2021tabnet,huang2020tabtransformer,somepalli2021saint,borisov2022deep,gorishniy2021revisiting}. Radiomics pipelines engineer thousands of texture descriptors from CT volumes, and 3D convolutional neural networks (CNNs) achieve strong internal performance on datasets such as NLST and LIDC, yet their scanner sensitivity and propensity for shortcut learning often negate cross-site gains: external validations reveal double-digit AUC drops when voxel spacing, reconstruction kernels, or case mix shift, and models can latch onto hospital-specific artifacts rather than biological signals~\cite{ardila_end--end_2019,zech_variable_2018,garau_external_2020,hellin2024unraveling}.

More recently, tabular foundation models and large tabular language models have emerged as promising directions. TabPFN meta-learns a transformer that approximates Bayesian posterior predictions across millions of synthetic tabular tasks, delivering hyperparameter-free, small-sample inference via in-context learning~\cite{hollmann2025accurate,schneider2024foundation}. Successors such as TabPFN-2.5 and drift-resilient TabPFN extend context length, relax attention bottlenecks, and incorporate simulated drifts into the prior~\cite{noauthor_prior_nodate,noauthor_realistic_nodate,noauthor_automldrift-resilient_tabpfn_2025}. Other work explores more realistic priors and cross-domain training curricula~\cite{noauthor_closer_nodate,noauthor_realistic_nodate}, and ``TabLLM''-style approaches serialize rows into prompts to reuse general-purpose reasoning from large language models~\cite{eremeev_turning_2025}. Parallel efforts investigate federated optimization and continual or on-device learning so that models can absorb new hospital evidence without breaching privacy constraints~\cite{guan2021domain,musa2025addressing}. Collectively, these developments define a new generation of AI systems for tabular healthcare data.

However, cross-hospital transfer remains fragile because three dominant pathologies of medical tabular data co-occur. First, sample scarcity: most pulmonary nodule cohorts contain only a few hundred labeled patients, which limits the stability of purely supervised training and amplifies overfitting~\cite{borisov2022deep}. Second, distribution shift: label prevalence, scanner kernels, demographics, and clinical workflows alter the marginal $P(X)$ and even the conditional $P(Y \mid X)$ between hospitals~\cite{koch2024distribution,guo_evaluation_2022}. Third, feature heterogeneity: sites log disjoint biomarker panels, adopt different measurement units, and follow distinct coding policies, which invalidates naive feature alignment and introduces missingness shifts~\cite{orouji_domain_nodate}. Domain adaptation research in imaging and wearables shows that adversarial training, cycle-consistent style transfer, optimal transport, and statistical moment matching can recover some performance under shift~\cite{guan2021domain,ahn_unsupervised_2023}, but these methods are rarely specialized for structured clinical data. Benchmarks such as TableShift and Wild-Time demonstrate that off-the-shelf robustness mechanisms still incur large out-of-distribution (OOD) gaps even when in-distribution accuracy is high~\cite{gardner_benchmarking_2024,yao2022wild}. Large-scale regulators and hospital governance boards increasingly regard shift detection, recalibration, and drift monitoring as core AI-safety requirements rather than optional post hoc checks~\cite{koch2024distribution}.

Tabular foundation models partially alleviate data scarcity, yet they inherit a closed-world assumption: the context set used during in-context learning is assumed to reflect the same joint distribution and feature schema as the query samples~\cite{schneider2024foundation}. When shifts in biomarkers, acquisition settings, or schemata emerge, attention weights may anchor on non-comparable neighbors, yielding overconfident yet incorrect predictions~\cite{noauthor_realistic_nodate,noauthor_automldrift-resilient_tabpfn_2025}. Emerging variants such as TabPFN-2.5 and drift-resilient TabPFN extend context length and inject synthetic drifts into the prior~\cite{noauthor_prior_nodate,noauthor_automldrift-resilient_tabpfn_2025}, but they remain sensitive to mismatched feature spaces and unlabeled target domains in the absence of explicit alignment. Tabular LLM approaches add reasoning capacity but incur substantial latency, quantization error for numerical values, and lack built-in clinical calibration, especially when lab panels or race-specific prevalences deviate from training distributions~\cite{eremeev_turning_2025}. Consequently, bridging the gap between high internal accuracy and safe cross-site deployment requires combining foundation models with principled unsupervised domain adaptation and feature selection that respect clinical realities.

Pulmonary nodule malignancy prediction is an archetypal stress test for these issues. Traditional clinical scores and their LASSO or GBDT successors were derived from narrowly defined cohorts with fixed demographic profiles and scanner protocols, so their coefficients silently encode source-specific prevalence, upper-lobe priors, and calcification heuristics~\cite{swensen1997chest,mcwilliams2013probability,li2011development,he2021novel,zhang_comprehensive_2022}. Meta-analyses across Asian screening programs and European cancer centers show that the same score threshold yields widely varying sensitivities (50--90\%) once smoking histories, granulomatous disease burdens, or acquisition kernels change~\cite{garau_external_2020,liu_establishment_2024}. Radiomics signatures tuned on sharp-kernel CTs lose discriminatory power on smooth-kernel images unless aggressively harmonized, and even then residual scanner bias can dominate texture features~\cite{hellin2024unraveling}. 3D CNNs for end-to-end malignancy prediction exhibit similar behavior, with performance degrading under scanner upgrades or demographic shifts~\cite{ardila_end--end_2019,zech_variable_2018}. These failures underscore that, without explicit feature pruning and alignment, both classical and modern models can become confidently wrong.

Similar tensions arise in population-health settings such as the BRFSS race-shift diabetes task. Demographic composition, socioeconomic exposures, and survey-year wording alter the marginal distribution of risk factors, while diabetes prevalence rises from roughly 12.5\% in White respondents to 17.4\% in non-White cohorts, causing fixed operating points to misfire~\cite{gardner_benchmarking_2024}. Benchmarks such as TableShift and Wild-Time make explicit that covariate shift ($P_s(X) \neq P_t(X)$), label shift ($P_s(Y) \neq P_t(Y)$), and concept shift ($P_s(Y \mid X) \neq P_t(Y \mid X)$) often co-occur, and that classical empirical risk minimization (ERM) on the source domain does not control the divergence term that drives target error~\cite{gardner_benchmarking_2024,koch2024distribution,yao2022wild}. In practice, these shifts invalidate the implicit closed-world assumptions behind most off-the-shelf models.

Feature engineering and feature selection choices are therefore as important as model class. Clinical tables mix continuous laboratory values, ordinal scores, sparse categorical codes, and structured missingness; naive one-hot encoding can expand dimensionality and encode site-specific artifacts. Stability-driven feature pruning, hierarchical encoding of categorical variables, and unit-aware normalization reduce spurious site signatures and focus attention on shared, clinically interpretable signals~\cite{sun2019informative}. Recursive feature elimination (RFE) across domains further enforces schema overlap, trading a slight drop in ceiling accuracy for substantial gains in portability when hospitals differ, and it is particularly helpful in small-sample, high-dimensional, and imbalanced regimes such as pulmonary nodules and radiomics panels~\cite{sun2019informative,borisov2022deep}.

Taken together, the research gap is stark. Tree ensembles and deep tabular networks struggle with small, heterogeneous cohorts and typically require retraining when schemas change~\cite{chen2016xgboost,gorishniy2021revisiting,borisov2022deep}. Foundation models improve small-sample performance but assume matched domains and aligned schemas~\cite{hollmann2025accurate,schneider2024foundation,noauthor_realistic_nodate}. Generic domain adaptation methods rarely account for missing features, label drift, or unlabeled targets in clinical tables~\cite{guan2021domain,ahn_unsupervised_2023,gardner_benchmarking_2024,koch2024distribution}. Federated and continual learning strategies help with privacy and incremental updates but do not by themselves guarantee cross-hospital calibration~\cite{guan2021domain,musa2025addressing}. A credible solution must (i) retain sample efficiency via strong pre-trained priors, (ii) discard site-specific signals that cannot transfer, and (iii) align source and target representations without target labels, while exposing calibration behavior under prevalence drift.

This study therefore adopts a pragmatic stance and introduces \emph{PANDA} (Pretrained Adaptation Network with Domain Alignment), a framework designed to transform diagnostic prediction through advanced ML and data analytics in realistic cross-hospital settings. PANDA chains three complementary components. First, a pre-trained tabular foundation model (TabPFN) supplies strong inductive priors for small cohorts by meta-learning across millions of synthetic tasks and enabling hyperparameter-free inference~\cite{hollmann2025accurate,schneider2024foundation}. Second, cross-domain RFE prunes to biomarkers that are consistently available and stable across sites, mitigating schema mismatch and hospital-specific artifacts~\cite{sun2019informative}. Third, a statistical alignment module based on Transfer Component Analysis (TCA) projects source and target cohorts into a shared reproducing-kernel subspace using unlabeled target data, minimizing distributional divergence while preserving clinical variance~\cite{pan2010domain}. PANDA targets the explicit goal of cross-hospital pulmonary nodule prediction with screening-level sensitivity and is further validated on the TableShift BRFSS Diabetes race-shift benchmark~\cite{gardner_benchmarking_2024}, ensuring that the proposed approach addresses both clinical and population-level distribution shifts without adding bespoke modeling components for each dataset.

In summary, cross-hospital pulmonary nodule prediction and BRFSS race-shift diabetes prediction expose the same deployment realities: privacy constraints, schema mismatch, prevalence drift, and the need for sensitivity at clinically actionable thresholds~\cite{koch2024distribution,gardner_benchmarking_2024}. Existing AI toolkits—tree ensembles, deep tabular networks, tabular foundation models, tabular LLMs, and generic domain adaptation—each leave gaps relative to these constraints~\cite{chen2016xgboost,arik2021tabnet,huang2020tabtransformer,somepalli2021saint,borisov2022deep,schneider2024foundation,guan2021domain,ahn_unsupervised_2023}. By integrating pre-trained tabular priors, schema-aware feature selection, and unsupervised domain alignment into a single pipeline, PANDA aims to restore calibration and discrimination under realistic deployment shifts. The remainder of this manuscript formalizes the cross-domain problem, surveys related work in tabular learning and medical domain adaptation, and presents PANDA as a practical instantiation of this design philosophy.

\label{sec:intro-end}


\section{Related Work}
\label{sec:rw-start}
\label{sec:related-work}

In this section, we review prior work from an AI perspective on cross-hospital diagnostic risk prediction using structured medical data. We organize the literature along five dimensions. First, we summarize model families for medical tabular data, including tree ensembles, deep tabular architectures, and tabular foundation models. Second, we discuss domain shift and domain adaptation methods in medical AI. Third, we review feature selection techniques for small, imbalanced, and heterogeneous cohorts. Fourth, we situate AI-based approaches for pulmonary nodule malignancy prediction within this broader landscape. Finally, we examine public benchmarks that expose cross-domain and temporal shifts in tabular data. This structure parallels the design of our proposed framework, PANDA, which integrates tabular foundation models, domain-aware feature selection, and kernel-based alignment to address the limitations identified in prior work.

\subsection{Tabular learning for medical data: tree ensembles, deep tabular networks, and tabular foundation models}

The literature on structured-data learning has progressed from classical ensembles
to deep tabular networks and, most recently, to tabular foundation models that
mirror the trends in NLP and computer vision~\cite{bommasani2022opportunities,
schneider2024foundation}. We separate the discussion into tree ensembles, deep
tabular architectures, and tabular foundation models to highlight where each excels
and why none alone solves cross-hospital robustness. In medical settings, the same
patient cohort may be modeled by tree ensembles, deep tabular networks, or
foundation models depending on sample size and operational constraints; understanding
their respective failure modes under domain shift is crucial for positioning PANDA.

\subsubsection{Tree ensembles for clinical tabular data}

Gradient-boosted decision trees (GBDTs) such as XGBoost, LightGBM, and CatBoost
remain the workhorses for EHR-style tables because they tolerate heterogeneous
scales, missing values, and noisy categorical codes while supporting monotone
constraints and other clinical priors~\cite{chen2016xgboost,grinsztajn_why_2022,
borisov2022deep}. Benchmarking studies covering hundreds of OpenML tasks show that
GBDTs still beat most neural baselines whenever training samples exceed a few
thousand, yet they overfit rapidly when $N<1{,}000$, cannot be fine-tuned
incrementally, and require full retraining when hospitals change their feature
schemas~\cite{gorishniy2021revisiting,shmuel_comprehensive_2024,fan_tabular_2024}.
Case reports on cross-institutional readmission and mortality prediction show that
tree models memorize acquisition artifacts (assay vendors, coding practices) and
lose 10--20 AUC points when transferred without recalibration, illustrating their
non-differentiable structure blocks end-to-end multimodal training and plug-and-play
domain adaptation~\cite{borisov_deep_2024,liu_tabpfn_2025}. This rigidity motivates
attempts to distill tree priors into differentiable encoders so that adaptation can
occur without rebuilding the model for each site.
These same inductive biases explain why trees dominate mid-scale public benchmarks
yet struggle in small, imbalanced medical cohorts: sparsity-aware splits handle
missing labs gracefully, but boosting magnifies noise when positive classes are rare
and hospital-specific priors leak into leaf structure. Because gradients stop at
each split, trees cannot share representations with image encoders or participate
in gradient-based domain adaptation, forcing manual feature harmonization whenever
schemas or prevalence shift.
In practice, this means that widely used implementations such as XGBoost and
LightGBM shine on medium-to-large EHR cohorts with thousands of patients and
hundreds of features, where sparse histogram-based splits and built-in handling of
missing indicators yield strong baselines with modest tuning. Studies across
OpenML, MIMIC-style EHR benchmarks, and TableShift-like suites repeatedly show
that properly regularized XGBoost variants achieve AUROCs in the high 0.70s to
low 0.80s for mortality, readmission, and sepsis detection, and that they retain
their ranking power even when categorical encodings or measurement scales differ
across hospitals~\cite{gorishniy2021revisiting,fan_tabular_2024,gardner_benchmarking_2024}.
These observations explain why tree ensembles remain the default choice for
operational clinical decision support systems.

On the small, heavily imbalanced cohorts typical of lung-screening registries
($N\approx 300$), the same capacity becomes a liability. Empirical analyses of
GBDT behavior on few-shot medical datasets reveal several critical failure modes:
first, deep trees can memorize the few malignant cases (often $<$50 positives),
leading to apparent training accuracies $>$95\% but test AUROCs collapsing once
covariates shift~\cite{fan_tabular_2024,gardner_benchmarking_2024}. Second,
calibration deteriorates dramatically in low-prevalence subgroups, with predicted
probabilities systematically overestimating malignancy in young non-smokers while
underestimating risk in elderly or high-burden cohorts~\cite{guo_evaluation_2022,koch2024distribution}.
Third, when new hospitals add or remove variables (e.g., different CT protocol
parameters or biomarker panels), there is no principled way to ``warm start'' or
incrementally fine-tune existing tree models without complete retraining from
scratch, making long-term maintenance expensive.

Because tree ensembles are non-differentiable and lack explicit latent
representations, they are also difficult to integrate into end-to-end multimodal
models or to pair with standard domain adaptation objectives. This mathematical
constraint has practical consequences: researchers attempting to combine XGBoost
with imaging features must resort to late fusion (averaging predictions) or
feature concatenation followed by retraining, both of which preserve the
non-differentiable barrier. Gradient-based domain adaptation methods such as
Domain Adversarial Neural Networks (DANN) or Maximum Mean Discrepancy (MMD)
regularization cannot be applied directly to tree models, requiring workarounds
that approximate tree decision surfaces with differentiable surrogates or hybrid
architectures that mix neural embeddings with gradient-boosted leaves.
This limitation motivates methods that transfer tree-like priors into differentiable
architectures or that use tree models as feature extractors rather than end-to-end
learners.

\subsubsection{Deep tabular networks}

Deep tabular architectures import attention and representation learning from
sequence models to overcome the adaptation gap. TabNet uses sequential feature masks
to mimic decision paths, TabTransformer contextualizes categorical embeddings,
FT-Transformer tokenizes all features, and SAINT introduces intersample attention
plus contrastive pre-training to borrow signal across patients~\cite{arik2021tabnet,
huang_tabtransformer_2020,gorishniy_revisiting_2021,somepalli2021saint}. Basis
Transformers, NODE variants, TabICL prompt-serialization, weight-prediction, and
regularization schemes further explore the space between neural and symbolic
models~\cite{margeloiu2023weight,loh_basis_2025,khoeini_fttransformer_2024,
bytezcom_tabicl_2025,somvanshi2024survey}. However, comprehensive surveys and
multiple leaderboard studies report that these models remain data-hungry,
sensitive to hyperparameters, and often trail tuned tree ensembles on small,
heterogeneous cohorts typical of tertiary hospitals~\cite{fan_tabular_2024,
shmuel_comprehensive_2024,ren_deep_2025}. In external-hospital transfers, SAINT and
FT-Transformer frequently degrade to near-random calibration when categorical codes
shift or when batch-size constraints prevent stable intersample attention. The
computational footprint (long training times, GPU memory pressure) further limits
adoption in clinical IT stacks, where inference latency and cost dominate.
Empirical comparisons on clinical risk prediction echo this pattern with concrete performance gaps. TabNet often
needs extensive learning-rate scheduling and sparsity penalties to match GBDT, and
TabTransformer under-utilizes numerical biomarkers unless carefully normalized.
FT-Transformer narrows the gap by embedding every feature, yet its quadratic
self-attention becomes impractical for wide tables. SAINT's intersample attention
helps when minibatches are large, but collapses on scarce data, making these models
fragile without strong regularization and carefully tuned augmentations.

Detailed benchmarking studies on clinical datasets reveal stark contrasts between large-scale public benchmarks and real-world medical cohorts. On UCI repository datasets with $>$10,000 samples, TabNet achieves AUROCs of 0.85--0.92, FT-Transformer reaches 0.87--0.94, and SAINT obtains 0.86--0.93, competitive with or slightly exceeding XGBoost's 0.84--0.91~\cite{gorishniy2021revisiting,shmuel_comprehensive_2024}. However, when evaluated on authentic clinical cohorts with $<$1,000 patients and significant missingness, the same models show dramatic performance degradation: TabNet AUROCs fall to 0.62--0.71, FT-Transformer to 0.65--0.74, and SAINT to 0.60--0.69, while XGBoost maintains relatively stable performance at 0.75--0.83~\cite{fan_tabular_2024,ren_deep_2025}.

The computational requirements create additional barriers to clinical adoption. Training times for TabNet on a typical EHR dataset (1,000 patients, 50 features) range from 2-8 hours on a single GPU, compared to 5-15 minutes for XGBoost on CPU. FT-Transformer requires 4-12 hours due to its attention mechanisms, and SAINT needs 6-15 hours plus substantial memory for inter-sample attention matrices~\cite{fan_tabular_2024}. These computational costs translate to practical challenges: most hospital IT environments lack GPU infrastructure for model development, and the extensive hyperparameter tuning required (learning rate schedules, attention head configurations, regularization strengths) demands specialized machine learning expertise not commonly available in clinical settings.

Furthermore, calibration studies reveal that deep tabular models often produce overconfident predictions on medical data. Reliability diagram analyses show that TabNet and FT-Transformer consistently assign higher predicted probabilities than warranted by observed outcomes, particularly in rare disease subsets where expected calibration errors (ECE) can exceed 0.15--0.20 compared to XGBoost's 0.04--0.08~\cite{guo_evaluation_2022,ren_deep_2025}. This overconfidence is particularly problematic for clinical decision support, where well-calibrated risk estimates are essential for appropriate triage and treatment decisions.
These limitations are amplified in clinical registries where hundreds of variables
encode comorbidities, medication history, and laboratory trajectories. Studies on
ICU mortality, sepsis, and readmission prediction report that deep tabular networks
match or slightly exceed tuned GBDTs on in-distribution test sets but lose their
advantage when evaluated on later time periods or new hospitals, especially when
categorical vocabularies change or when privacy constraints cap batch sizes~\cite{gorishniy2021revisiting,
guo_evaluation_2022,ren_deep_2025}. In such small-$N$, high-dimensional regimes,
hyperparameter sensitivity translates directly into clinical risk: minor changes in
learning rate or regularization can flip decisions near treatment thresholds.
Compared with tree ensembles, these architectures seek to learn shared feature
representations that might in principle adapt across hospitals or tasks. In
practice, however, their appetite for data and tuning means that performance gains
are often limited to large industrial benchmarks; on noisy, heterogeneous medical
tables with only a few hundred patients, they frequently underperform simpler
models and exhibit brittle calibration under shift. This contrast sets the stage
for tabular foundation models such as TabPFN, which embrace a meta-learning,
few-shot perspective instead of training a new deep network from scratch for each
cohort.

\subsubsection{Tabular foundation models}

Tabular foundation models push self-supervised pre-training and in-context learning
into structured data. TabPFN meta-trains a transformer on millions of synthetic
datasets sampled from diverse structural-causal priors, learns to approximate
posterior predictive distributions, and performs inference via a single forward
pass without gradient updates~\cite{hollmann2025accurate,hollmann_accurate_2025}.
Follow-up work expands its reach without breaking the closed-world assumption:
TabPFN-2.5 relaxes quadratic attention to accommodate tens of thousands of context
rows and documents an augmented pre-training suite; diagnostics such as ``A Closer
Look at TabPFN v2'' show that the model remains overconfident under covariate shift,
prompting wrappers that adjust representations before prediction~\cite{noauthor_prior_nodate,
noauthor_closer_nodate,noauthor_realistic_nodate}. Drift-resilient variants model
temporal shift with secondary structural-causal modules and record measurable gains
when patient mixes evolve~\cite{helli_drift-resilient_2024,noauthor_automldrift-resilient_tabpfn_2025}.
Other studies adapt the same prior-learning paradigm to drug discovery, radiomics,
and graph embeddings, highlighting both the portability and fragility of tabular
foundation models beyond flat tables~\cite{chen_tabpfn_2025,eremeev_turning_2025,
liu_tabular_2025}. Tabular Large Language Models (TabLLMs) serialize rows or mini
tables into prompts so that general-purpose LLMs can reason over discrete entries,
but they remain computationally prohibitive for high-throughput risk prediction and
struggle with precise numeric calibration~\cite{brown2020language,hegselmann2023tabllm,
jayawardhana_transformers_2025}. Recent analyses of high-dimensional omics
applications reinforce that even TabPFN requires aggressive feature selection or
prior-guided embeddings to stay calibrated, underscoring its closed-world
assumption~\cite{zhou_limitations_2025,noauthor_pdf_nodate}. PFN-Boost, LLM-Boost,
and hybrid residual schemes blend foundation backbones with tree-style updates or
prompts, but benchmark reports such as Wild-Tab still find overfitting to
training-domain quirks unless explicit alignment and calibration are layered
on~\cite{kolesnikov_wild-tab_2023,liu_tabpfn_2025,loh_basis_2025}.
Closed-world constraints surface in three ways: (i) feature mismatch---TabPFN
expects aligned schemas and cannot reason about biomarkers absent from the context;
(ii) covariate drift---attention retrieves misleading neighbors when acquisition
protocols move, producing overconfident errors; and (iii) context-length bottlenecks
that force sub-sampling when rows exceed a few thousand. These limits explain why
prior studies resort to RFE or hand-crafted embeddings before invoking TabPFN and
why drift-resilient variants add causal dynamics to temper temporal shift.

These observations motivate hybrid approaches that explicitly combine strong priors
with domain-alignment hooks. Table~\ref{tab:model_summary} summarizes the comparative
strengths and weaknesses of these model families for medical tabular tasks,
highlighting why PANDA fuses TabPFN with feature selection and unsupervised alignment
instead of relying on any single paradigm.

\begin{table}[htbp]
\centering
\caption{Comparative strengths and weaknesses of tabular model families in medical AI.}
\label{tab:model_summary}
\footnotesize
\begin{tabular}{p{2.5cm}p{3.2cm}p{3.9cm}p{3.9cm}}
\toprule
\textbf{Model Class} & \textbf{Representative Algorithms} & \textbf{Strengths in Medical AI} & \textbf{Limitations in Cross-Hospital Tasks} \\
\midrule
Tree Ensembles & XGBoost, LightGBM, CatBoost & Interpretable, robust to missingness/outliers, encode clinical constraints & Overfit small cohorts, non-differentiable, no inherent transfer learning, require full retraining per site \\
Deep Tabular & TabNet, TabTransformer, FT-Transformer, SAINT, NODE & Differentiable, capture complex interactions, allow multimodal fusion & Data hungry, extensive tuning, high compute cost, brittle without alignment \\
Foundation Models & TabPFN, TabPFN-2.5, TabLLM & Hyperparameter-free inference, strong small-$N$ priors, probabilistic outputs & Sensitive to distribution/feature shift, limited context length, assume aligned schemas \\
\bottomrule
\end{tabular}
\end{table}

\subsection{Domain shift and domain adaptation in medical AI}

Domain adaptation (DA) provides the vocabulary for managing the covariate, label,
and concept shifts that materialize when AI crosses hospital boundaries. Classical
analysis decomposes target error into source error plus a divergence term, motivating
alignments and invariance objectives. In practice, medical deployments encounter
overlapping types of shift: changes in patient mix and ordering policies alter
$P(X)$, new screening programs or diagnostic criteria perturb $P(Y)$, and evolving
clinical practice modifies $P(Y\mid X)$~\cite{koch2024distribution,guo_evaluation_2022}.
Pulmonary nodule malignancy prediction is particularly exposed to this triad of
shifts because granulomatous disease burden, scanner protocols, and radiologist
thresholds vary sharply across regions.

\subsubsection{Statistical alignment vs. adversarial objectives}

Maximum Mean Discrepancy (as in TCA), correlation alignment (CORAL), and
transport-based projections minimize moment discrepancies in a latent
space~\cite{pan2010domain,sun2016correlationalignmentunsuperviseddomain,
grubinger2015domain,zhang_adadiag_2022,li_transport-based_2024}. They are attractive
for medical tables because they offer closed-form or deterministic solutions and
remain stable when labeled target data are absent. Adversarial approaches (DANN,
cycle-consistent style transfer) attempt to erase domain cues via discriminators,
but surveys show they destabilize when cohorts are tiny, leading to mode collapse or
erasure of clinically salient signals~\cite{guo_evaluation_2022,zhang_adadiag_2022,
guan2021domain}. In ICU mortality and readmission tasks, DANN can underperform ERM
by wide margins because the discriminator trivially detects domain cues from missing
patterns, causing the encoder to discard predictive features. In contrast, MMD- or
CORAL-style alignment improves calibration modestly and avoids catastrophic
degradation, motivating our reliance on TCA for small-sample settings.
Classic error decompositions also separate covariate shift ($P_s(X)\neq P_t(X)$)
from label shift ($P_s(Y)\neq P_t(Y)$) and concept shift ($P_s(Y|X)\neq P_t(Y|X)$);
only the first benefits cleanly from moment matching, while the second demands
prevalence-aware calibration and the third often needs feature auditing or human
review~\cite{pan2010domain,gardner_benchmarking_2024,koch2024distribution}. These
regimes frequently co-occur in multi-hospital deployments, explaining why single
DA objectives show mixed results.

\subsubsection{Heterogeneity, missingness, and temporal drift}

Medical DA must grapple with heterogeneous feature sets and evolving acquisition
policies. Feature-space DA (FSDA) and transport-based alignment project source and
target into shared latent spaces, while open-set domain adaptation handles mismatched
label spaces and schema drift that arise when hospitals collect different labs~\cite{luo2021fsda,
grubinger2015domain,pham_open-set_2025,li_transport-based_2024}. DomainATM,
feature-aware PCA, and ontological mapping frameworks first identify which
biomarkers are stable across sites before alignment, reducing negative transfer~\cite{guan2021domain,
guan_domainatm_2023,guan_domain_2022}. Missingness-shift studies demonstrate that
when ordering policies change (e.g., different lab panels for triage), standard
covariate-shift assumptions break; MNAR-aware corrections and explicit missingness
modeling become mandatory~\cite{zhou_domain_2023,stokes_domain_2025}. Temporal
adaptation work (Wild-Time, multi-attention encoders for COVID-19) highlights that
drift accumulates over months, so models require continual recalibration rather than
one-time transfer~\cite{ahn_unsupervised_2023,he_multi-attention_2022}.

\subsubsection{Domain generalization and open-set gaps}

TableShift, Wild-Tab, and Wild-Time benchmarks quantify how far models fall once
distributions move: they reveal a near-linear relation between in-distribution and
out-of-distribution accuracy, but also show that label shift dominates error budgets
and that prevailing domain-generalization objectives (GroupDRO, IRM, VREx) rarely
beat strong ERM or GBDT baselines on tabular data~\cite{gardner_benchmarking_2024,
noauthor_mlfoundationstableshift_nodate,gardner_tableshift_nodate,
kolesnikov_wild-tab_2023,ahn_unsupervised_2023}. Open-set and partial-label settings
are common in healthcare (target hospital omits certain comorbidities); current DA
methods often assume aligned label spaces and therefore miscalibrate rare conditions.
Regulatory guidance now treats shift detection and recalibration as part of
post-market surveillance, emphasizing that robustness must be engineered rather than
assumed~\cite{koch2024distribution}.
Complementary benchmarks and surveys on generic tabular learning echo these
findings: across hundreds of datasets, tuned GBDTs remain exceptionally strong
baselines, and many deep or domain-generalization architectures fail to deliver
consistent gains once evaluation moves beyond a handful of leaderboard tasks~\cite{fan_tabular_2024,
shmuel_comprehensive_2024,somvanshi2024survey}. Moreover, empirical decompositions
of error budgets highlight that label shift and calibration drift often dominate
covariate shift, suggesting that feature-space alignment alone is insufficient for
reliable deployment. Together with the medical DA literature, these results argue
for methods that combine strong small-sample priors, explicit feature governance,
and lightweight, task-aware alignment instead of relying on black-box ``robust''
architectures.

\subsubsection{Domain adaptation and transfer learning for clinical tabular and EHR data}

Recent work brings these ideas to longitudinal EHR and claims data. AdaDiag-style
methods align source and target hospitals in a representation space while jointly
training prognostic models, reporting partial recovery of AUROC lost when models
trained on MIMIC-like cohorts are evaluated at external centers~\cite{zhang_adadiag_2022,
guan2021domain}. Multi-center EHR foundation models go further by pre-training
sequence encoders on records from dozens of institutions and then fine-tuning on
downstream tasks, demonstrating that shared representations can reduce the amount
of labeled data required for local adaptation~\cite{noauthor_multi-center_nodate}.
These approaches show that both unsupervised alignment and transfer learning have
value in clinical AI, but they typically assume abundant longitudinal data, focus
on large hospitals with rich EHR infrastructure, and operate on sequential rather
than static tabular summaries.

Standard domain-adaptation theory provides a unifying lens: target risk can be
bounded by source risk plus a measure of distribution discrepancy and a term
capturing irreducible label-set differences~\cite{pan2010domain,guo_evaluation_2022}.
Reducing error on the source domain alone is therefore insufficient; one must also
control divergence between source and target feature distributions, for example via
moment-matching, adversarial objectives, or feature-space DA.
Beyond centralized settings, federated learning extends these ideas by allowing
multiple hospitals to collaborate without sharing raw data. Surveys on federated
learning for medical imaging and pattern recognition summarize how FL can pool
experience across institutions while preserving privacy, and methods such as
FedFusion explicitly combine domain adaptation with personalized encoders to handle
heterogeneous feature spaces and scarce labels~\cite{rehman_federated_2023,
guan_federated_2024,kahenga_fedfusion_2025}. However, most federated frameworks
target high-volume imaging or EHR tasks, assume substantial local computation and
at least some labeled data at each site, and still rely on shared model
architectures and broadly aligned feature schemas. They are therefore complementary
to, rather than a replacement for, lightweight DA strategies tailored to very small
tabular cohorts with partially mismatched feature sets.

Table~\ref{tab:da_strategies} summarizes the main DA families discussed above and
their implications for cross-hospital tabular deployment.

\begin{table}[htbp]
\centering
\caption{Representative domain-adaptation strategies in medical AI and their relevance to cross-hospital tabular risk prediction.}
\label{tab:da_strategies}
\footnotesize
\begin{tabular}{p{2.5cm}p{1.5cm}p{3.1cm}p{3.2cm}p{3.2cm}}
\toprule
\textbf{Method family} & \textbf{Typical modality} & \textbf{Key assumptions} & \textbf{Pros} & \textbf{Limitations for small cross-hospital tabular cohorts} \\
\midrule
Statistical alignment (MMD, TCA, CORAL, transport) & Tabular, EHR, imaging & Shared feature schema; access to source data and unlabeled target samples; primarily covariate shift & Closed-form or deterministic mappings; stable when target labels are absent; easy to plug into existing pipelines~\cite{pan2010domain,sun2016correlationalignmentunsuperviseddomain,grubinger2015domain,li_transport-based_2024} & Does not directly correct label or concept shift; assumes overlapping feature sets; may misalign rare subgroups without additional calibration~\cite{gardner_benchmarking_2024,koch2024distribution,guo_evaluation_2022} \\
Adversarial representation learning (DANN-style) & Imaging, EHR sequences & Access to source and target data with domain labels; discriminator encouraged to remove site identity & Learns domain-invariant representations jointly with task loss; flexible for complex modalities~\cite{guo_evaluation_2022,zhang_adadiag_2022} & Unstable on tiny cohorts; discriminators exploit missingness patterns, causing encoders to discard predictive features; can underperform ERM in ICU-style tasks~\cite{zhang_adadiag_2022,guan2021domain} \\
Feature-space DA and domain-aware FS (FSDA, DomainATM) & Tabular, EHR & At least partially shared feature space; access to both domains during training & Selects features that are predictive and stable across sites; reduces reliance on site-specific surrogates and noisy biomarkers~\cite{luo2021fsda,guan2021domain,guan_domainatm_2023,guan_domain_2022} & Still assumes sizable overlap in measured variables; does not natively handle missing entire feature blocks or unlabeled target hospitals with severe schema mismatch \\
Domain generalization and temporal adaptation (TableShift, Wild-Tab, Wild-Time) & Tabular & Multiple labeled source distributions; no target labels during training & Reveal failure modes under temporal, demographic, and institutional shift; provide standardized evaluation suites~\cite{gardner_benchmarking_2024,gardner_tableshift_nodate,kolesnikov_wild-tab_2023,ahn_unsupervised_2023} & Many domain-generalization objectives (e.g., GroupDRO, IRM, VREx) rarely beat strong ERM or GBDT baselines; benchmarks show label shift and calibration drift dominate what feature matching can fix~\cite{gardner_benchmarking_2024,noauthor_mlfoundationstableshift_nodate} \\
Federated and federated-DA frameworks (FL, FedFusion-style) & Imaging, tabular & Multiple compute-capable hospitals; communication budget; typically some local labels and shared model architecture~\cite{rehman_federated_2023,guan_federated_2024,kahenga_fedfusion_2025} & Preserve data privacy while learning from distributed cohorts; can combine personalization with domain adaptation and label efficiency & Often require significant local computation and labeled target data; focus on large imaging or EHR tasks; do not directly address very small tabular cohorts with feature mismatch and strict label scarcity \\
\bottomrule
\end{tabular}
\end{table}

Existing EHR-focused methods mostly address temporal drift or site differences in
large cohorts, whereas our setting combines small, imbalanced tabular cohorts,
heterogeneous feature sets, and unlabeled target hospitals. This gap motivates
PANDA's combination of strong tabular priors, cross-domain feature selection, and
lightweight alignment tailored to static risk scores rather than long EHR sequences.

\subsection{Feature selection and domain-aware stability for small medical cohorts}

High-dimensional yet small-sample tabular cohorts are ubiquitous in medicine: lung
screening registries, omics panels, and survey-based risk scores often contain
hundreds of variables for only a few hundred or thousand patients. Na\"{i}ve
learning in this regime leads to unstable decision boundaries and non-reproducible
feature attributions. Feature selection methods aim to reduce dimensionality,
stabilize inference, and focus clinician attention on biomarkers that are both
predictive and economical to collect.

\subsubsection{Small-sample and high-dimensional feature selection}

Classical filter and wrapper methods, such as mutual information ranking or
recursive feature elimination with SVMs, laid the groundwork for identifying
compact biomarker sets but struggle when features are highly correlated or when
class imbalance is severe~\cite{guyon2002gene}. More recent approaches explicitly
target high-dimensional, low-sample-size settings. WPFS-style methods learn
feature weights jointly with a classifier, GRACES uses graph convolutions to
propagate importance across correlated features, and DeepFS leverages deep networks
to screen features via nonlinear embeddings~\cite{chen2023graces,liu2022deepfs,
li2023deep}. These techniques are attractive for medical AI because they can
down-select from hundreds of candidate variables to a dozen stable predictors while
controlling overfitting. Empirical studies on omics and imaging-genomics datasets
show that such methods can maintain or even improve AUC while halving the number of
features, directly reducing assay costs and simplifying model interpretation.
However, most of these works assume a single training domain: the selected subset
is optimized for internal performance and may not transfer when another hospital
measures a slightly different panel or when missingness patterns change.
From a methodological standpoint, this marks a shift from classical LASSO or
univariate ranking---which rely on linear or marginal-effect assumptions and can be
highly unstable in small cohorts---to architectures that explicitly model complex
feature interactions and redundancy. WPFS and GRACES, for example, introduce
auxiliary networks or graph structures to propagate importance across correlated
features, while DeepFS leverages deep encoders to identify nonlinear manifolds
where only a subset of variables drive variation~\cite{chen2023graces,liu2022deepfs,
li2023deep}. These designs are particularly appealing in high-dimensional, sparse
medical settings (omics panels, questionnaire data), but they still optimize for
one domain at a time and do not ensure that the chosen biomarkers remain predictive
under cross-hospital shift.

\subsubsection{Feature selection with transformers and foundation models}

Attention-based models provide an alternative route to feature selection by
interpreting attention weights, learned masks, or perturbation scores as measures of
importance. TabNet learns sparse feature masks that indicate which variables are
consulted at each decision step, while transformer-based architectures expose
token-level attention maps that can be aggregated across layers and heads~\cite{arik2021tabnet,
huang2020tabtransformer,somepalli2021saint,somvanshi2024survey}. In practice,
researchers often perform permutation-based importance estimation using a strong
tabular backbone---GBDT or TabPFN---and then apply RFE-style pruning, retaining the
top-k features that consistently contribute to performance. This paradigm is
well-suited to small medical cohorts because it leverages the inductive biases of
powerful models while regularizing the input space. For foundation models such as
TabPFN, feature selection also mitigates closed-world constraints: by removing
unstable or site-specific variables, one can reduce the chance that attention
focuses on hospital identifiers or acquisition artifacts rather than pathology.

\subsubsection{Domain-aware and cross-site feature selection}

Standard feature selection treats all samples as exchangeable, implicitly assuming
that feature-importance rankings are identical across domains. Domain-aware methods
instead optimize a subset that is simultaneously predictive in multiple hospitals
or under multiple sampling schemes. FSDA and related frameworks extend DA
objectives with feature-level penalties, rewarding variables whose contributions
remain stable after alignment~\cite{luo2021fsda,sun2019informative}. Multi-site
studies on EHR and imaging data show that such cross-domain criteria can discard
site-specific surrogates (e.g., local procedure codes) while preserving clinically
meaningful biomarkers. PANDA adopts this philosophy in a pragmatic way: TabPFN is
used as a strong scoring model, but feature elimination is guided jointly by
source-site performance and cross-site stability, leading to a compact
``best8'' subset that is consistently informative in both hospitals. These
domain-aware subsets provide low-dimensional, harmonized inputs to TCA, reducing
the risk of negative transfer and making the subsequent alignment problem better
posed.
Viewed through this lens, feature selection becomes a form of implicit domain
alignment: instead of matching full distributions in a high-dimensional space, one
first discards variables whose predictive contribution is strongly domain-specific
and focuses on biomarkers that are consistently informative across sites. This is
particularly valuable when hospitals measure different panels or exhibit pronounced
missingness shift, because aligning on a smaller, shared subset of stable features
is both statistically and operationally simpler. PANDA effectively instantiates
this principle by using a pre-trained tabular foundation model to rank features
jointly across two hospitals and retaining only those with robust importance,
thereby coupling representation learning with domain-aware feature governance.

\begin{table}[htbp]
\centering
\caption{Representative feature selection methods for small, imbalanced, high-dimensional biomedical tabular data.}
\label{tab:fs_methods}
\scriptsize
\begin{tabular}{p{2.0cm}p{2.0cm}p{2.8cm}p{3.2cm}p{3.2cm}}
\toprule
\textbf{Method} & \textbf{Model family} & \textbf{Small-sample / imbalance handling} & \textbf{Interpretability characteristics} & \textbf{Representative biomedical use cases} \\
\midrule
Recursive feature elimination (RFE) with linear or tree models & Wrapper around SVM, logistic regression, or tree ensembles & Wrapper search over feature subsets can overfit when $N$ is small and features are correlated; often combined with cross-validation and class-balanced sampling & Produces explicit ranked feature lists and compact subsets; easy to inspect and map to clinical variables~\cite{guyon2002gene} & Widely used in early gene-expression and biomarker panels; basis for many clinical risk-score and radiomics pipelines \\
LASSO / elastic-net logistic regression & Embedded linear models & $\ell_1$ or $\ell_1{+}\ell_2$ penalties shrink coefficients, providing some robustness to high dimensionality; still assumes linear log-odds and can be unstable under heavy collinearity & Sparse coefficients directly indicate selected features; compatible with odds-ratio interpretation familiar to clinicians~\cite{tibshirani1996regression} & Common in radiomics and EHR risk models where interpretability and coefficient-based reporting are required \\
GRACES & Graph-convolutional-network-based FS~\cite{chen2023graces} & Specifically targets high-dimensional, low-sample-size data by modeling feature relations on a graph; alleviates overfitting compared with independent filters & Outputs a compact subset informed by graph structure; can be visualized as a network of interacting biomarkers & Demonstrated on omics-style datasets; suitable when prior knowledge or correlations between biomarkers are important \\
DeepFS & Deep feature screening with autoencoders~\cite{liu2022deepfs,li2023deep} & Uses deep encoders to learn low-dimensional representations and rank features, handling ultra-high-dimensional, sparse, and potentially imbalanced data & Provides importance scores for each original feature; retains flexibility to operate in supervised or unsupervised mode & Evaluated on synthetic and biomedical high-dimensional datasets; useful when the number of variables far exceeds the number of patients \\
Domain-aware FS (FSDA-style) & Feature selection for domain adaptation~\cite{luo2021fsda} & Encourages selection of features that remain predictive across domains, implicitly handling covariate shift between sites & Produces subsets that are jointly predictive and domain-stable, supporting cross-hospital deployment & Applied to benchmark DA tasks; conceptually aligned with cross-hospital biomarker selection in multi-center medical studies \\
Transformer / foundation-model-based FS & Attention- or score-based selection using TabNet, TabTransformer, and tabular foundation models~\cite{arik2021tabnet,huang2020tabtransformer,somvanshi2024survey} & Leverages high-capacity or pre-trained models to estimate nonlinear feature importance; can be combined with RFE to mitigate small-sample overfitting & Attention weights, feature masks, or permutation-based scores yield ranked features; aligns with explainable-AI practices & Increasingly used in biomedical tabular and omics datasets; PANDA's cross-cohort RFE uses a tabular foundation model as the scoring backbone \\
\bottomrule
\end{tabular}
\end{table}

\subsection{Pulmonary nodule malignancy prediction: from clinical scores to multi-modal AI}

\subsubsection{Clinical risk scores and logistic models}

Pulmonary nodule malignancy prediction is a canonical testbed for cross-domain
robustness. Classical logistic scores---Mayo Clinic, Veterans Affairs, Brock
(PanCan), PKUPH, Li, and derivatives---achieve internal AUCs above 0.85 but
regularly drop to 0.60--0.80 in external validations, especially in Asian or
community-screening cohorts where disease prevalence and case mix differ from the original
development populations~\cite{swensen1997chest,swensen1997archives,mcwilliams2013probability,
li2011development,he2021novel,garau_external_2020,zhang_comprehensive_2022,
liu_establishment_2024}. These scores typically combine age, smoking history,
nodule size, location, and morphology into a logit-based risk function. Meta-analyses
covering tens of thousands of nodules confirm that calibration deteriorates most
severely in subgroups such as solitary upper-lobe nodules and specific ethnic
groups, reflecting both label-shift and covariate-shift mechanisms~\cite{garau_external_2020,
zhang_comprehensive_2022,chen_pulmonary_2025}. Recent Chinese multi-centre studies further report that Brock- and PKUPH-type models can
experience substantial AUC declines when applied to contemporary Chinese screening cohorts, even
after refitting, consistent with shifts in underlying prevalence and competing benign conditions~\cite{zhang_comprehensive_2022,
liu_establishment_2024}. While
recalibration or re-estimation of coefficients can partially restore performance,
these fixes require local labels and do not address feature-mismatch: new hospitals
may lack some variables (e.g., emphysema grading) or measure them differently.

Targeted audits make the degradation concrete. In TB-endemic Korean hospitals,
Mayo and VA shrink to AUC $\approx$0.60 while Brock declines to $\approx$0.68
despite an internal AUC near 0.94, and Chinese multi-center studies find that
Brock and PKUPH can fall from $\approx$0.90 internally to 0.70--0.77 once
prevalence and granulomatous disease rates shift~\cite{yang_comparison_2018,
cui_comparison_2019,li_evaluation_2020}. PET-augmented variants such as the
Herder score raise internal discrimination to $\approx$0.92 by incorporating
metabolic imaging, yet they lose specificity in TB-endemic or inflammatory
regions where uptake is nonspecific~\cite{herder_clinical_2005,yang_comparison_2018}.
These case studies underscore that most clinical scores embed site-specific
prevalence, referral patterns, and feature definitions, so ``plug-and-play''
deployment without alignment is unrealistic.

Each classical score carries its own design trade-offs. The Mayo Clinic model was
derived from several hundred clinic-referred patients with indeterminate nodules,
emphasizing age, smoking, nodule diameter, spiculation, and upper-lobe location,
whereas the Veterans Affairs model targeted high-risk, predominantly male veterans
with larger lesions~\cite{swensen1997chest,swensen1997archives}. The Brock (PanCan)
model was trained in a screening cohort enriched for small nodules and incorporates
emphysema, family history, and more granular morphology descriptors, while the
PKUPH and Li scores adapt similar feature sets to Chinese tertiary-hospital and
screening populations~\cite{mcwilliams2013probability,li2011development,he2021novel,
liu_establishment_2024}. A recent meta-analysis focused on the Brock model reports
pooled AUC $\approx 0.80$ across $>80{,}000$ patients but highlights substantially
lower performance in Asian cohorts, solitary nodules, subsolid nodules, and larger
lesions (AUC often $\approx 0.74$ or below), underscoring that apparent
``universality'' in development data masks sizeable domain-specific errors~\cite{chen_pulmonary_2025}.
Across Mayo, VA, Brock, and PKUPH, external validations repeatedly document drops
from internal c-statistics in the high-0.80s to 0.60--0.75 when applied to community
screening or granulomatous-disease–endemic regions~\cite{garau_external_2020,
zhang_comprehensive_2022,liu_establishment_2024}.

These patterns can be summarized along three axes: development cohorts are often
single-center and demographically narrow; variables focus on easily collected
clinical and simple CT descriptors; and the underlying model is a logistic
regression that assumes a linear log-odds relationship between covariates and
malignancy. Table~\ref{tab:nodule_scores} sketches representative scores along
these dimensions. In development, all achieve reasonable discrimination and are
simple enough to implement as bedside calculators, but the same simplicity makes
them brittle under shift: logistic coefficients absorb local prevalence, imaging
protocols, and referral patterns, so external use without recalibration results in
systematic underestimation or overestimation of risk in particular subgroups.

\begin{table}[htbp]
\centering
\caption{Representative pulmonary nodule malignancy scores and common external-validation issues.}
\label{tab:nodule_scores}
\footnotesize
\begin{tabular}{p{2.2cm}p{3.2cm}p{4.0cm}p{4.0cm}}
\toprule
\textbf{Score} & \textbf{Development cohort} & \textbf{Key variables} & \textbf{External-validation observations} \\
\midrule
Mayo Clinic & Clinic-referred indeterminate nodules in smokers & Age, smoking history, nodule size, spiculation, upper-lobe location & Internal AUC in the high-0.80s; frequent overestimation of risk and AUC drops to $\approx$0.6--0.7 in screening and non-U.S. cohorts \\
Veterans Affairs & Predominantly male veterans with larger nodules & Age, smoking, nodule diameter, location & Good performance in veterans; miscalibration when transported to mixed-gender or lower-risk populations \\
Brock (PanCan) & CT screening cohort with many small nodules & Age, sex, family history, emphysema, size, type, location & Meta-analytic pooled AUC $\approx 0.80$; markedly lower AUC in Asian, solitary, and subsolid nodules~\cite{chen_pulmonary_2025} \\
PKUPH / Li & Chinese tertiary-hospital and screening cohorts & Age, smoking, nodule size and type, lobulation, spiculation & High internal AUC but drops in external series; performance depends strongly on CT protocol and case mix~\cite{li2011development,he2021novel,liu_establishment_2024} \\
\bottomrule
\end{tabular}
\end{table}

From the perspective of this thesis, these scores provide clinically interpretable
baselines and useful prior knowledge about which coarse-grained descriptors matter,
but they do not solve cross-hospital robustness. Their small development cohorts
and rigid functional form make it difficult to incorporate new biomarkers or adapt
to feature-mismatch without re-estimating the entire model, motivating more flexible
tabular approaches that can share information across hospitals while respecting
regulatory demands for calibration and subgroup transparency.

\subsubsection{Radiomics pipelines with traditional machine learning}

Radiomics pipelines extract hundreds to thousands of hand-crafted features from CT
volumes, offering richer representations than clinical risk scores but introducing
major reproducibility hazards. Texture and wavelet descriptors vary with voxel
spacing, reconstruction kernel, and segmentation protocol; ComBat-style harmonization
reduces scanner effects yet requires batch labels and can blur subtle lesions~\cite{hellin2024unraveling,
garau_external_2020}. In internal validation, radiomics-based classifiers that pair
LASSO- or stability-selected feature subsets with SVMs, random forests, or GBDTs
typically report AUCs in the 0.75--0.90 range, but these numbers rarely carry over
to new scanners or hospitals. External validations on LIDC-IDRI, LUNA16, and NLST
repeatedly report double-digit AUC drops when deployed to scanners with different
kernels or patient mixes, while shortcut-learning analyses show that models sometimes
rely on grid artifacts or reconstruction noise rather than morphology~\cite{ardila_end--end_2019,
zech_variable_2018,koch2024distribution}. These failures illustrate that radiomics
alone cannot guarantee transportability and that alignment plus feature vetting are
required before cross-hospital use.

Concrete exemplars reinforce that fragility. The Bayesian Integrated Malignancy
Calculator (BIMC) blended radiomics with clinical covariates and modestly
outperformed Mayo, Brock, and PKUPH (AUC $\approx$0.90 vs. $\approx$0.78) on an
Italian derivation cohort, yet its advantage diminished when scanners, slice
thickness, or kernels changed~\cite{perandini_solid_2016}. Hawkins-style NLST
radiomics achieved AUC $\approx$0.83 without external validation, and retrospective
audits show that a single reconstruction tweak can reorder the features selected by
LASSO~\cite{garau_external_2020,hellin2024unraveling}. Radiomics therefore supplies
richer morphology descriptors but still requires harmonization, feature governance,
and domain-aware alignment rather than assuming reproducibility across hospitals.

\begin{table}[htbp]
\centering
\caption{Representative radiomics-based pulmonary nodule malignancy models and reported generalization behavior.}
\label{tab:radiomics_models}
\scriptsize
\begin{tabular}{p{2.2cm}p{2.0cm}p{1.5cm}p{2.0cm}p{1.0cm}p{1.0cm}p{2.8cm}}
\toprule
\textbf{Study / model} & \textbf{Imaging data} & \textbf{Centers / nodules} & \textbf{Classifier} & \textbf{Internal AUC} & \textbf{External AUC} & \textbf{Harmonization / scanner sensitivity} \\
\midrule
Generic radiomics pipelines & 2D/3D chest CT or low-dose CT nodules & Single- or few-center cohorts (sizes vary) & LASSO- or stability-selected features with SVM, RF, or GBDT & Typically 0.75--0.90 & 0.10--0.20 lower & Performance degrades when kernel, slice thickness, or vendor changes; ComBat-style harmonization can reduce but not eliminate scanner effects~\cite{garau_external_2020,hellin2024unraveling} \\
Multi-center reproducibility analyses & 3D CT radiomics features across scanners & Multi-center CT datasets & -- & -- & -- & Many texture features show intraclass correlation coefficients $<0.5$ across vendors and protocols; harmonization helps but cannot fully restore stability~\cite{hellin2024unraveling} \\
Radiomics + clinical scoring models & Radiomics signatures combined with clinical descriptors & Hospital-specific or regional nodule cohorts & Elastic-net / logistic regression, RF, or GBDT & High ($\geq 0.80$) & Mid-0.70s or lower & External validations report double-digit AUC loss and sensitivity to case mix and acquisition protocol~\cite{garau_external_2020} \\
\bottomrule
\end{tabular}
\end{table}

Published inter-scanner analyses often report intraclass correlation coefficients
below 0.5 for entropy and run-length features, indicating poor reliability even
before model fitting~\cite{hellin2024unraveling}. ComBat can regress out known batch
effects when acquisition labels are available, but it can also blur subtle lesions
and fails when batch membership is unknown at inference time, leaving a gap that
tabular-alignment pipelines attempt to close. Beyond handcrafted features, many
radiomics pipelines incorporate LASSO, elastic-net logistic regression, or
stability-selection frameworks to shrink coefficients and stabilize feature sets
before training SVM, random forest, or GBDT classifiers. Although these strategies
help curb overfitting in small cohorts, they do not eliminate sensitivity to
acquisition protocols: the same feature may be retained in one scanner configuration
and discarded in another because its estimated importance changes with kernel or
slice thickness. Multi-center studies frequently report 10--20 percentage-point AUC
drops when models are transported without revisiting segmentation, feature
extraction, and harmonization choices~\cite{garau_external_2020,hellin2024unraveling}.
As a result, radiomics pipelines tend to behave like carefully tuned, center-specific
instruments rather than plug-and-play risk predictors, and their complexity makes it
hard for clinicians to trace failure modes back to specific preprocessing or
feature-engineering steps.

\subsubsection{Deep-learning CAD systems}

End-to-end deep-learning computer-aided diagnosis (CAD) systems extend the radiomics
pipeline by learning 3D convolutional representations directly from CT volumes or
multi-view patches. Large-scale screening trials such as NLST have enabled
3D CNNs to achieve AUCs in the mid-0.90s on internal validation, sometimes
matching or surpassing expert radiologists~\cite{ardila_end--end_2019}. Subsequent
works combine deep features with handcrafted radiomics or clinical covariates,
showing further gains on curated datasets~\cite{li_predicting_2019,lin_combined_2024}.
Causey et al.'s NoduleX reproduced radiologist malignancy ratings with AUC
$\approx$0.99 on LIDC-IDRI, and Google's NLST-scale 3D CNN maintained AUC
$\approx$0.94 on an independent hospital cohort of 1{,}139 CTs, illustrating how
massive, homogeneous datasets can suppress variance~\cite{causey_highly_2018,
ardila_end--end_2019}.
However, these successes often rely on tightly controlled acquisition protocols and
substantial annotation effort. External validations reveal double-digit AUC drops
when voxel spacing, reconstruction kernels, or vendor mix shift, and shortcut-learning
analyses demonstrate that CNNs may rely on markers, reconstruction noise, or scanner
metadata rather than nodule morphology~\cite{zech_variable_2018,hellin2024unraveling,
koch2024distribution}. Moreover, most deep CAD systems treat imaging in isolation or
only append a handful of clinical variables, limiting their ability to reason over
complex comorbidity profiles or laboratory trajectories. Multi-view and multi-scale
architectures that process cropped nodules, surrounding parenchyma, and whole-lung
context can mitigate some of these issues, but they further increase computational
cost and annotation effort. Multi-task variants that jointly predict malignancy,
growth, or histological subtype promise richer supervision but require large,
carefully curated datasets that few hospitals possess. In practice, many published
CAD systems are trained and tuned on a single trial or institution, with limited
reporting on cross-hospital generalization or calibration. Where multi-center
experiments are reported, performance is typically rescued by site-specific
fine-tuning on labeled cases from each target hospital, and very few studies attempt
label-free ``train at A, deploy at B'' deployment. As a result, deep CAD systems
remain powerful local tools rather than robust cross-hospital risk predictors.

\subsubsection{Tabular and multi-modal nodule models}

Later machine-learning models---LASSO, random forests, GBDTs, Bayesian networks, and
hybrid radiomics-clinical models---attempt to combine the strengths of scores and
imaging~\cite{he2021novel,garau_external_2020,li_predicting_2019,lin_combined_2024}.
GBDTs and random forests improve internal calibration and handle nonlinear
interactions but still require site-specific recalibration or feature mapping before
deployment because their learned weights implicitly encode scanner kernels and local
smoking histories. Multi-modal models that fuse deep image features with clinical
covariates via late fusion or stacking demonstrate promising gains on LIDC-IDRI and
NLST, yet most studies remain single-center or rely on random train--test splits
that do not reflect real cross-hospital deployment. Only a handful of works evaluate
performance when training on one hospital and testing on another, and these typically
report substantial AUC drops and unstable decision thresholds~\cite{garau_external_2020,
hellin2024unraveling}. Recent multi-center studies in Asian and Chinese screening
cohorts echo this pattern: even when models are re-estimated or augmented with
additional imaging features for new hospitals, external AUCs often plateau in the
low- to mid-0.70s and remain sensitive to protocol details and case mix~\cite{garau_external_2020,
hellin2024unraveling,wu_strategy_2023}. These observations motivate a shift toward tabular-centric
models that can incorporate imaging-derived biomarkers while explicitly handling
feature mismatch and domain shift rather than assuming homogeneous acquisition.
Within the tabular family, two broad patterns emerge. Purely clinical models use
logistic regression or tree ensembles on demographics, smoking history, and simple
CT descriptors, sometimes enriched with laboratory indices or comorbidity scores.
These models are attractive for deployment because all inputs are routinely
available in electronic health records, yet they inherit the limitations of
classical scores: most are developed and validated in a single institution, assume
aligned features across sites, and rarely report behavior under explicit domain
shift. Hybrid models instead treat radiomics signatures or deep image embeddings as
additional covariates in a tabular classifier, enabling richer decision boundaries
while retaining some interpretability via variable-importance analyses. However,
their feature spaces are even more brittle across scanners and hospitals, as both
image-derived and clinical variables can change distributions or go missing.

Existing works seldom implement formal domain-adaptation strategies for these
tabular or multi-modal models. External evaluations, when present, typically test a
fixed model on a new hospital without feature re-alignment or recalibration,
documenting sizable performance degradation but not offering systematic remedies.
Only a few studies experiment with simple recalibration or refitting on a small
local sample, and virtually none explore cross-domain feature selection or latent
alignment tailored to nodule malignancy prediction~\cite{garau_external_2020,
hellin2024unraveling}. Consequently, the literature lacks robust, tabular-centric
frameworks that (i) start from strong small-sample priors, (ii) identify a compact
set of biomarkers stable across hospitals, and (iii) explicitly align feature
distributions without assuming access to large labeled target cohorts. Taken
together, these studies show that neither handcrafted risk scores, radiomics
pipelines, nor deep CNN-based CAD systems currently offer reliable malignancy
prediction across hospitals without local retraining or recalibration. Addressing
these gaps is a central motivation for the PANDA framework developed in this thesis.

\subsection{Benchmarks and open problems for cross-domain tabular learning}

Beyond single-institution case studies, public benchmarks now stress-test shift
robustness. TableShift curates 15 binary tasks across healthcare, finance, and public
policy, with explicit temporal, geographic, and demographic shifts to measure
out-of-distribution accuracy drops and calibration drift~\cite{gardner_benchmarking_2024,
gardner_tableshift_nodate}. Wild-Tab extends this idea to few-shot, structure-aware
adaptation, showing that even tabular foundation models lose 5--15 AUC points under
schema-preserving shifts~\cite{kolesnikov_wild-tab_2023}. Wild-Time focuses purely on
temporal drift, revealing that performance decays monotonically unless models refresh
their priors~\cite{ahn_unsupervised_2023}. These resources contrast with medical
imaging benchmarks, where the input grid is fixed; in tabular settings, feature
heterogeneity and missingness add extra axes of mismatch. Our inclusion of the
TableShift BRFSS Diabetes race-shift task aligns the pulmonary nodule study with a
large-scale public benchmark, demonstrating that the proposed alignment strategy is
not confined to proprietary cohorts.
TableShift also surfaces common failure modes: GroupDRO and IRM rarely beat ERM on
tabular tasks, label shift explains much of the OOD loss, and high ID accuracy is
necessary but insufficient for shift robustness~\cite{gardner_benchmarking_2024,
noauthor_mlfoundationstableshift_nodate}. Wild-Time isolates temporal drift,
showing monotonic degradation without continual recalibration~\cite{ahn_unsupervised_2023};
these findings mirror hospital deployments where assay updates or policy changes
quietly reshape feature distributions.

\subsubsection{Gap analysis and positioning of PANDA}

Across model families and adaptation techniques, several open issues persist. First,
closed-world assumptions in tabular foundation models preclude feature-mismatched
deployment: TabPFN and its variants require aligned schemas and struggle when target
hospitals omit or redefine biomarkers. Second, most DA methods presume access to
abundant labeled or schema-aligned target data, which is unrealistic in
privacy-constrained hospitals and incompatible with regulatory expectations that
models remain stable under silent drift~\cite{koch2024distribution,guo_evaluation_2022}.
Third, missingness shift and label shift remain underexplored despite being dominant
drivers of clinical miscalibration in TableShift and Wild-Time; simply matching
latent distributions cannot fix changes in prevalence or ordering policies~\cite{gardner_benchmarking_2024,
ahn_unsupervised_2023}. Finally, reproducibility crises in radiomics and deep imaging
models show that aggressive priors or harmonization cannot replace explicit
alignment and feature governance~\cite{hellin2024unraveling,koch2024distribution}.

PANDA is designed to address a specific intersection of these gaps rather than
compete with every prior line of work. By treating a tabular foundation model as a
plug-in prior, PANDA inherits strong small-sample performance without hand-tuning
but augments it with cross-domain RFE that explicitly searches for a compact subset
of biomarkers stable across hospitals. This step operationalizes domain-aware
feature selection, yielding a shared feature set (``best8'') that remains
predictive in both institutions and provides harmonized inputs for subsequent
alignment. TCA is then applied in the latent space induced by TabPFN, combining the
representation power of foundation models with the stability of kernel-based
alignment to handle unlabeled target data. The same pipeline is evaluated both on a
private cross-hospital pulmonary nodule cohort and on the public TableShift BRFSS
Diabetes race-shift task, demonstrating that the ingredients are not handcrafted for
a single dataset but generalize across tabular shift scenarios~\cite{gardner_benchmarking_2024,
noauthor_mlfoundationstableshift_nodate}. To our knowledge, this is the first
framework to jointly combine a tabular foundation model, cross-domain RFE, and TCA
for cross-hospital pulmonary nodule risk prediction and public TableShift-style
tabular shift benchmarks. In this sense, PANDA fills the gap between
single-domain tabular FMs, imaging-focused DA, and benchmark-driven tabular DA by
providing an end-to-end, alignment-aware framework tailored to small, imbalanced,
and feature-mismatched medical cohorts.

\label{sec:rw-end}


\section{Problem Formulation}
\label{sec:pf-start}
\label{sec:problem-formulation}

Cross-hospital medical classification mixes distribution shift, sample scarcity, and feature heterogeneity. We cast it as an unsupervised domain adaptation (UDA) problem on structured clinical data: the goal is reliable prediction in a target hospital without target labels. The framing mirrors common deployment constraints in medical AI.

\subsection*{Cross-Domain Learning Setup}

Let the labeled source cohort be $\mathcal{D}_s=(X_s,Y_s)=\{(\mathbf{x}_i^s,y_i^s)\}_{i=1}^{n_s}$ and the unlabeled target cohort be $\mathcal{D}_t=(X_t,\varnothing)=\{\mathbf{x}_j^t\}_{j=1}^{n_t}$. Each instance $\mathbf{x}\in\mathbb{R}^d$ collects structured clinical variables and $y\in\{0,1\}$ encodes malignancy for nodules or diabetes status for TableShift. Domains expose imperfectly overlapped feature sets: $\mathcal{F}_s$ and $\mathcal{F}_t$ denote the recorded indices, $\mathcal{F}_{\cap}=\mathcal{F}_s\cap\mathcal{F}_t$ the shared subset used for modeling, and $\mathcal{F}_{\setminus}=\mathcal{F}_s \triangle \mathcal{F}_t$ the features seen in only one hospital or demographic group. We write $d_{\cap}=|\mathcal{F}_{\cap}|$ for the dimensionality after intersection.

Admissible models operate on $\mathcal{X}_{\cap}=\mathbb{R}^{d_{\cap}}$. The objective is to learn $f: \mathcal{X}_{\cap} \rightarrow \mathcal{Y}$ that minimizes the target risk
\[
\mathcal{R}_t(f)=\mathbb{E}_{(\mathbf{x},y)\sim P_t}[\ell(f(\mathbf{x}),y)]
\]
subject to the constraint that $Y_t$ remains unobserved during training. Privacy policies (HIPAA, GDPR) render this unsupervised domain adaptation framing the only feasible option in many multi-institution collaborations.

The two tasks tackled in this dissertation differ strongly in their feature vocabularies yet share the notation above. Table~\ref{tab:feature_partitions} summarizes the recorded variables, the overlapping subsets, and the information discarded when aligning hospitals or demographic splits.

\begin{table}[htbp]
  \centering
  \caption{Feature partitioning for the two cross-domain tasks. $\mathcal{F}_{\cap}$ contains the variables usable across domains; $\mathcal{F}_{\setminus}$ collects site- or demographic-specific factors excluded from modeling.}
  \label{tab:feature_partitions}
  \begin{tabular}{p{0.23\textwidth}p{0.32\textwidth}p{0.32\textwidth}}
    \toprule
    Task & Shared feature families ($\mathcal{F}_{\cap}$) & Site-/group-specific factors ($\mathcal{F}_{\setminus}$) \\ \midrule
    Cross-hospital pulmonary nodules & Age, sex, smoking indicators, upper-lobe flags, diameter, calcification, spiculation, tumor biomarkers (CEA, NSE, Cyfra21-1), ventilatory metrics (VC, DLCO) & CT reconstruction kernel IDs, segmentation quality scores, rare lab assays, hospital-specific comorbidities, imaging vendor tags \\ 
    TableShift BRFSS race split & Demographics (age, sex, education), chronic-disease history, lifestyle (smoking, alcohol, physical activity), metabolic labs (BMI proxies, hypertension indicators), survey year & State-specific policy items, optional socioeconomic questions only asked in some years, race-restricted modules, state sampling weights \\ \bottomrule
  \end{tabular}
\end{table}

\subsection*{Challenges in Cross-Institutional Learning}

Clinical tabular cohorts usually include only a few hundred labeled patients. For hypothesis classes on $d_{\cap}$ shared features, estimation error scales as $\widetilde{O}(\sqrt{d_{\cap}/n_s})$, making high-capacity models unreliable once $n_s \le 500$. Many UDA techniques implicitly bank on larger sample sizes than most hospitals can release.

Distributional mismatch compounds the limits. Under the standard adaptation bound
\[
\mathcal{R}_t(f) \le \mathcal{R}_s(f) + \tfrac{1}{2} d_{\mathcal{H}\Delta\mathcal{H}}(P_s,P_t) + \lambda,
\]
the divergence term $d_{\mathcal{H}\Delta\mathcal{H}}$ dominates when variability is substantial---differences in CT scanners, assay calibrations, demographics, or survey wording. Partial feature overlap means source and target supports only partly coincide, straining assumptions behind kernel alignment and adversarial methods.

Shift types manifest differently across the two tasks but lead to the same failure mode of inflated $\lambda$:
\begin{itemize}
    \item \textbf{Covariate shift:} $P_s(\mathbf{x})\neq P_t(\mathbf{x})$ emerges when Hospital B observes higher upper-lobe prevalence or when BRFSS non-White respondents show distinct BMI and lifestyle distributions. Without alignment, the similarity kernel inside TabPFN attends to mismatched neighbors and the risk bound loosens.
    \item \textbf{Label shift:} $P_s(y)\neq P_t(y)$ appears in lung nodules when malignancy prevalence changes from tertiary centers to community hospitals, and in BRFSS when diabetes rates vary by race. Thresholds tuned on $P_s$ miscalibrate decision curves once applied to $P_t$.
    \item \textbf{Concept shift:} $P_s(y|\mathbf{x})\neq P_t(y|\mathbf{x})$ captures definition drift, such as tuberculosis confounding upper-lobe malignancy cues in some regions or policy changes altering how survey responses map to the DIABETES label.
\end{itemize}

The combination of $d_{\mathcal{H}\Delta\mathcal{H}}$ growth and concept shift means that even small empirical risk on Cohort~A or the White BRFSS split does not guarantee acceptable target risk. Explicit alignment and feature harmonization are therefore prerequisites for any foundation-model method deployed in these settings.
\label{sec:pf-end}


\section{Solution}
\label{sec:solution}
\label{sec:sol-start}

To bridge the gap between advanced tabular foundation models and the practical constraints of cross-hospital medical AI, we introduce PANDA (\textbf{P}retrained \textbf{A}daptation \textbf{N}etwork with \textbf{D}omain \textbf{A}lignment). PANDA is a composite algorithmic framework designed to predict pulmonary nodule malignancy with high stability across heterogeneous clinical environments. It explicitly addresses the tripartite challenge of small-sample scarcity, distribution shift, and feature heterogeneity through a tightly integrated pipeline.

We formalize the PANDA solution as a composite function $\pandafunc: \inputspace \to [0, 1]$ mapping raw, heterogeneous input space to a calibrated malignancy probability. The framework consists of four sequential stages: (1) Domain-Aware Feature Alignment and Selection, (2) Foundation Model Feature Extraction, (3) Latent Space Domain Adaptation, and (4) Multi-View Ensemble Classification.

\subsection{Architectural Overview}
\label{subsec:sol-architecture}

The PANDA framework operates as a directed acyclic graph (DAG) of data transformations, guiding raw clinical inputs through a carefully orchestrated sequence to produce a calibrated malignancy probability. Table \ref{tab:data-flow} details this journey, illustrating how feature representations evolve at each stage.

\begin{table}[htbp]
    \centering
    \caption{Data flow and transformations in the PANDA framework. The pipeline progressively refines the data from raw, high-dimensional inputs to low-dimensional, domain-invariant embeddings, and finally to a calibrated probability. In our pulmonary nodule experiments, $k = 8$, $d_{\cap} = 8$, and $m = 15$, but these dimensions are dataset-dependent.}
    \label{tab:data-flow}
    \resizebox{\textwidth}{!}{%
        \begin{tabular}{lllll}
            \toprule
            \textbf{Stage} & \textbf{Component} & \textbf{Input Space} & \textbf{Core Operation} & \textbf{Output Space} \\
            \midrule
            1 & Cross-Domain RFE & $\mathbb{R}^{\rawdim}$ & Iterative elimination via $\rfeimportance(\func)$ & $\mathbb{R}^{k}$ ($\rfeselectedfeatures$, $k = |\rfeselectedfeatures|$) \\
            2 & Feature Alignment & $\mathbb{R}^{k}$ & Schema intersection on $\sharedfeatures$ & $\mathbb{R}^{d_{\cap}}$ ($d_{\cap} = |\sharedfeatures|$) \\
            3 & Latent TCA & $\mathbb{R}^{d_{\cap}}$ & Projection $\featurevec' = \tcaprojectionmatrix^\top \featurevec$ & $\mathbb{R}^{m}$ ($m$ = latent dimension) \\
            4 & TabPFN Classifier & $\mathbb{R}^{m}$ & Classification $\tabpfnfunc(\featurevec')$ & $[0, 1]$ (Prob.) \\
            5 & Ensemble & $[0, 1]^{\numpreprocessbranches \times \numrandomseeds}$ & Temperature-scaled averaging & $[0, 1]$ (Final) \\
            \bottomrule
        \end{tabular}%
    }
\end{table}



The data journey begins with \textbf{Stage 1: Cross-Domain Recursive Feature Elimination (RFE)}. From the raw feature set of $\rawdim$ dimensions, this stage systematically reduces the dimensionality to a concise set of $k$ clinically impactful features, denoted as $\rfeselectedfeatures$ (where $k = |\rfeselectedfeatures|$). This reduction serves to minimize the joint error ($\adaptabilityterm$) across domains and focuses on the most diagnostically relevant biomarkers for malignancy prediction. In our pulmonary nodule experiments, the cost-effectiveness criterion selects $k = 8$, but $k$ is dataset-dependent.

Following feature selection, \textbf{Stage 2: Feature Alignment} performs a robust schema intersection ($\sharedfeatures$) between the RFE-selected features and those available in each target hospital's dataset. This yields a shared feature schema of dimensionality $d_{\cap} = |\sharedfeatures|$ that is tailored to each pair of hospitals (for our cross-hospital pulmonary nodule cohort, $d_{\cap} = 8$). This crucial step effectively addresses the "missingness shift" by ensuring that only universally available and relevant variables proceed through the pipeline, adapting the feature set to the specific constraints of each clinical site.

Subsequently, in \textbf{Stage 3: Latent Transfer Component Analysis (TCA)}, the refined feature set $\mathbb{R}^{d_{\cap}}$ is adapted. TCA projects these features into a lower-dimensional, domain-invariant space $\mathbb{R}^{m}$, with $m$ treated as a tunable latent dimension (we use $m = 15$ in the reported experiments). This projection, $\featurevec' = \tcaprojectionmatrix^\top \featurevec$, is learned to minimize the Maximum Mean Discrepancy (MMD) between the source and target domain distributions in this latent space, thereby mitigating covariate shift.

\textbf{Stage 4: TabPFN Classifier} then processes the TCA-transformed features to predict the initial malignancy probability. In this stage, the TabPFN acts as a robust, pre-trained classifier on the domain-adapted data. Its in-context learning capabilities allow it to infer complex, non-linear decision boundaries without extensive gradient updates on small medical datasets.

Finally, \textbf{Stage 5: Ensemble} aggregates predictions from multiple TabPFN Classifier instances. This involves temperature scaling of the individual TabPFN outputs, followed by averaging across $\numpreprocessbranches \times \numrandomseeds$ ensemble members, yielding the final, calibrated malignancy probability. The temperature scaling is crucial for adjusting predicted probabilities to account for potential label shifts between domains, ensuring more reliable and interpretable outputs.

\subsubsection*{Mapping to Formal Problem Formulation}
To establish a clear connection between the PANDA framework's architectural stages and its formal definition in Equation \ref{eq:panda-formalization}, we explicitly map the components of $\pandafunc(\featurevec) = \hypothesis \circ \adaptmap \circ \pi_{\cap} \circ \rfeop (\featurevec)$ to the corresponding stages:
\begin{itemize}
    \item $\rfeselectedfeatures \leftarrow$ Stage 1 (Cross-Domain RFE)
    \item $\sharedfeatures \leftarrow$ Stage 2 (Feature Alignment)
    \item $\adaptmap \leftarrow$ Stage 3 (Latent TCA)
    \item $\hypothesis \leftarrow$ Stages 4--5 (TabPFN Classifier + Ensemble/Calibration)
\end{itemize}
This mapping clarifies how the abstract components of the PANDA function are instantiated through the concrete processing steps, providing readers with a direct link between theory and implementation.

\subsection{Core Component I: TabPFN as a Robust Classifier}
\label{subsec:sol-tabpfn}

The backbone of PANDA is the TabPFN (Tabular Prior-Data Fitted Network) foundation model. While TabPFN is capable of extracting embeddings, in the PANDA framework, we leverage it primarily as a robust \textbf{Classifier} operating on the domain-adapted features. This strategic choice is driven by TabPFN's unique ability to generalize well on small tabular datasets without extensive domain-specific fine-tuning.
Mathematically, after the feature set $\featurevec$ has been reduced and aligned (Stages 1-2) and then adapted by TCA into $\featurevec'$ (Stage 3), the TabPFN classifier $\tabpfnfunc(\cdot)$ directly maps these transformed features to a malignancy probability:
\begin{equation}
    \tabpfnfunc(\featurevec') \in [0, 1]
\end{equation}
Here, the entire pre-trained TabPFN model, including its Transformer encoder and classification head, is used to make predictions.

\textbf{Theoretical Advantage (Prior-Data Fitted Learning):}
The core innovation here is leveraging the \textbf{Prior-Data Fitted (PFN)} nature of TabPFN. TabPFN was meta-trained on millions of synthetic datasets generated from Structural Causal Models (SCMs). This meta-training instills a strong Bayesian prior $\priorfunc$ that favors simple, causal explanations over spurious correlations. In the context of medical data, where sample sizes are often small ($\sourcedatasize \approx 300$), traditional deep learning models typically overfit. However, TabPFN's meta-learned prior allows it to infer complex, non-linear decision boundaries (e.g., non-linear interactions between Age and Nodule Size) on the adapted features without requiring gradient updates on the small medical dataset. This directly addresses the "Small Sample Size" constraint defined in Equation \ref{eq:small-sample-constraint}. By applying TabPFN to TCA-transformed features, we benefit from both domain invariance and strong generalization capabilities on small cohorts.

\subsection{Core Component II: Cross-Domain RFE with Cost-Effectiveness Index}
\label{subsec:sol-rfe}

Medical datasets often contain high-dimensional noise (e.g., irrelevant radiomics features) and "concept shift" (features whose predictive value changes across hospitals). To handle this, we employ a Cross-Domain Recursive Feature Elimination (RFE) mechanism.
The selection process is not merely about maximizing \auc; it is governed by a global \textbf{Cost-Effectiveness Index (CEI)} that balances predictive power against stability, clinical acquisition cost, and model complexity. We formulate the optimization problem for the optimal feature subset size $\rfecurrentdim^*$ as:
\begin{equation}
    \rfeselectedfeatures = \argmax_{\rfecurrentdim} \left( \rfecomponentweights_1 \rfeperformance(\rfecurrentdim) + \rfecomponentweights_2 \rfeefficiency(\rfecurrentdim) + \rfecomponentweights_3 \rfestability(\rfecurrentdim) + \rfecomponentweights_4 \rfesimplicity(\rfecurrentdim) \right)
\end{equation}
where the components are defined as:
\begin{itemize}
    \item \textbf{Performance ($\rfeperformance$):} A weighted sum of discriminative metrics on the source validation folds: $\rfeperformance = 0.5 \cdot \auc + 0.3 \cdot \accuracy + 0.2 \cdot \fonescore$.
    \item \textbf{Efficiency ($\rfeefficiency$):} Measures the computational efficiency, defined as $1 - \frac{\bar{T} - T_{\min}}{T_{\max} - T_{\min}}$, where $\bar{T}$ is the mean training time for the feature subset.
    \item \textbf{Stability ($\rfestability$):} Quantifies the robustness across multiple metrics, defined as $1 - \frac{1}{3} (\sigma_{\text{AUC}} + \sigma_{\text{Acc}} + \sigma_{\text{F1}})$, where $\sigma$ denotes the normalized standard deviation across folds.
    \item \textbf{Simplicity ($\rfesimplicity$):} A sparsity penalty $\exp(-\sparsityparam \rfecurrentdim)$ to prefer smaller, clinically manageable subsets (typically $\rfecurrentdim \in [5, 10]$).
\end{itemize}

Implementation-wise, we use `TabPFNClassifier` as the base estimator for RFE. In each iteration, we calculate the \textbf{Permutation Importance} $\rfeimportance(\featurevecj)$ of each feature $\featurevecj$. The feature with the lowest importance is pruned, and the model is re-evaluated. This recursive process ensures that the retained features $\rfeselectedfeatures$ are not only predictive but also stable against the noise inherent in small cohorts. In our experiments, this process consistently identifies a "Best-8" subset (Age, Spiculation, Lobulation, Diameter, etc.) that is robust across hospitals.

\subsection{Core Component III: Latent Space TCA Adaptation}
\label{subsec:sol-tca}

To explicitly align the source and target distributions, we apply Transfer Component Analysis (TCA) to the feature space reduced by RFE (Stage 2). Standard TCA is often applied to raw features, but this is suboptimal for complex medical data where the manifold structure might be noisy or high-dimensional. Instead, we construct the kernel matrix $\kernelmatrix$ using the RFE-selected features $\featurevec$. Specifically, we employ a \textbf{Linear Kernel} on these features:
\begin{equation}
    \kernelmatrix_{ij} = \langle \featurevec_i, \featurevec_j \rangle
\end{equation}
This projection, $\featurevec' = \tcaprojectionmatrix^\top \featurevec$, learns a domain-invariant feature representation that is then fed into the TabPFN Classifier (Stage 4).

\textbf{Justification for Linear Kernel on RFE-Selected Features:}
After Cross-Domain RFE, the selected features are assumed to reside in a more disentangled and clinically relevant space. In this context, a simple linear alignment via TCA is often sufficient to match the first-order moments of the distributions, thereby reducing covariate shift. This approach avoids the complexity and hyperparameter sensitivity (e.g., $\gammaK$ in RBF kernels) that often plagues non-linear kernel methods, especially on small datasets. The subsequent TabPFN Classifier is then capable of modeling complex non-linear relationships within this linearly adapted feature space.

The optimization objective minimizes the Maximum Mean Discrepancy (MMD) trace:
\begin{equation}
    \min_{\tcaprojectionmatrix} \quad \tr({\tcaprojectionmatrix}^\top \kernelmatrix \mmdmatrix \kernelmatrix \tcaprojectionmatrix) + \mu \tr({\tcaprojectionmatrix}^\top \tcaprojectionmatrix)
\end{equation}
where $\mmdmatrix$ is the MMD matrix defined block-wise as $\mmdmatrix_{ij} = 1/\sourcedatasize^2$ if $\featurevec_i, \featurevec_j \in \sourcedata$, $1/\targetdatasize^2$ if $\featurevec_i, \featurevec_j \in \targetdata$, and $-1/(\sourcedatasize \targetdatasize)$ otherwise. The projection $\tcaprojectionmatrix$ aligns the marginals $\marginalsourcedist(\mathbf{z})$ and $\marginaltargetdist(\mathbf{z})$, directly addressing the Covariate Shift challenge.
Figure~\ref{fig:tca_dimensionality} visualizes how the TCA projection compresses the RFE-reduced features into a compact latent subspace while shrinking the MMD divergence between the source and target domains, highlighting the balance controlled by $\mu$.

\begin{figure}[htbp]
    \centering
    \includegraphics[width=\linewidth]{img/cross_hospital/TCA_dimensionality_reduction.pdf}
\caption{\textbf{TCA-based domain adaptation visualization.} a,b PCA views before and after TCA transformation, showing the tightened alignment between source and target. c,d t-SNE views before and after TCA, demonstrating improved cluster center alignment while preserving local structure.}
    \label{fig:tca_dimensionality}
\end{figure}


\subsection{Core Component IV: Multi-View Ensemble and Calibration}
\label{subsec:sol-ensemble}

To mitigate the variance associated with small-sample learning and label shift, PANDA employs a \textbf{Strategic Multi-Branch Ensemble}. Instead of a single model, we construct an ensemble of $N=32$ members, derived from $\numpreprocessbranches=4$ distinct preprocessing strategies expanded by $\numrandomseeds=8$ random seeds (which control feature shuffling and rotational invariants).

The four strategic branches are designed to present different "views" of the data to the foundation model:
\begin{enumerate}
    \item \textbf{High-Complexity Ordinal (Raw+Quantile):} Features are mapped to a uniform distribution via `QuantileTransformer` (handling outliers) and concatenated with raw features. Categorical variables are Ordinal Encoded.
    \item \textbf{Low-Complexity Ordinal (Raw):} Raw features are used directly (relying on TabPFN's internal robustness). Categorical variables are Ordinal Encoded.
    \item \textbf{High-Complexity Numeric:} Similar to Branch 1, but categorical variables are treated as numeric.
    \item \textbf{Low-Complexity Numeric:} Raw features with numeric encoding for categoricals.
\end{enumerate}

\textbf{Rotational Invariance:}
For each branch, we train 8 versions. In each version, the feature columns are cyclically permuted (Rotated). This is critical because Transformers can exhibit positional bias; rotating the features ensures that the model's attention mechanism attends to all features equally, regardless of their column index.

\textbf{Temperature Scaling and Aggregation:}
The final probability is obtained via \textbf{Temperature Scaling} to calibrate the predictions from the ensemble of TabPFN Classifiers. This is crucial when the target domain prevalence differs from the source (Label Shift), as uncalibrated models often produce overconfident probabilities. The aggregated prediction is:
\begin{equation}
    \hat{\labelval}(\featurevec) = \frac{1}{\numpreprocessbranches \times \numrandomseeds} \sum_{i=1}^{\numpreprocessbranches \times \numrandomseeds} \activation\left(\frac{\logitoutput}{\temperature}\right)
\end{equation}
where $\logitoutput$ is the logit output of the $i$-th TabPFN Classifier member, and $\temperature=0.9$ is the temperature parameter empirically tuned to soften predictions.

\subsection{Theoretical Justification and Feasibility}
\label{subsec:sol-justification}

\textbf{Addressing Small Sample Size:}
The use of a frozen, pre-trained encoder $\tabpfnencoder$ avoids the need to train a deep network from scratch on $\sourcedatasize \approx 300$ samples. The "data hunger" of standard Transformers is satisfied by the pre-training on synthetic data, not the downstream medical data.

\textbf{Addressing Distribution Shift:}
The framework attacks shift at three levels:
1.  \textbf{Feature Level:} `QuantileTransformer` aligns the marginal distributions of individual features (Covariate Shift).
2.  \textbf{Latent Level:} TCA explicitly minimizes the divergence term $\domaindivergence$ in the generalization bound by aligning the joint embedding distributions.
3.  \textbf{Output Level:} Temperature scaling calibrates the posterior probabilities against Label Shift.

\textbf{Real-time Feasibility:}
Despite the ensemble complexity ($N=32$), the inference involves only forward passes of the Transformer (which are highly parallelizable) and linear projections. Empirical tests on a standard CPU (Intel i7) show an average inference latency of $< 200$ ms per patient. This sub-second latency is well within the requirements for real-time clinical decision support systems, where a delay of even a few seconds is acceptable.

\begin{figure}[htbp]
    \centering
    \includegraphics[width=\linewidth]{img/cross_hospital/Pre-trained Tabular Foundation Mode Pipeline_new.pdf}
    \caption{\textbf{The PANDA framework architecture.}
    (a) Compositional pipeline: from original tabular data through ensemble training, prediction aggregation, class imbalance adjustment, to final classification output.
    (b) Multi-branch ensemble with $\numpreprocessbranches=4$ preprocessing strategies,
    each generating $\numrandomseeds=8$ ensemble members via different random seeds.}
    \label{fig:model_details}
\end{figure}
\label{sec:sol-end}


\section{Methods}

\subsection{Ethics statement and Data Collection}
This study was approved by the Institutional Review Boards of Sun Yat-sen University Cancer Center (Guangzhou, China) and Henan Tumor Hospital (Zhengzhou, China), and conducted in accordance with the ethical principles of the Declaration of Helsinki. All patient data used in this study were retrospectively collected from electronic medical records and fully de-identified prior to analysis. Written informed consent for research use of clinical data was obtained from all patients diagnosed with solitary pulmonary nodules (SPNs) at the time of hospital admission. No identifiable personal information was retained.

The training cohort (Cohort A, $n=295$) was derived from Sun Yat-sen University Cancer Center (Guangzhou, China) between January 2011 and December 2016. The external test cohort (Cohort B, $n=190$) was collected at Henan Tumor Hospital (Zhengzhou, China) from January 2013 to xxxxxx. All patients provided written informed consent for the use of clinical data in scientific research at the time of hospital admission.

Inclusion criteria were as follows: (1) presence of a solitary pulmonary nodule with diameter $\leq 3$ cm identified by chest computed tomography (CT); (2) no evidence of extrapulmonary malignancy; (3) histopathological diagnosis confirmed by surgical resection, CT-guided transthoracic needle biopsy, or bronchoscopy; (4) complete electronic medical records, including clinical, laboratory, and imaging data collected within 7 days prior to anti-tumor treatment. Patients were excluded if they had prior thoracic malignancies, incomplete records, or other comorbidities interfering with diagnostic interpretation.

The collected variables included patient demographics (age, sex, height, weight, body mass index), smoking history, family history of cancer, and symptoms (e.g., fever, cough, hemoptysis, chest pain). Radiologic features of SPNs were recorded, including anatomical location (lung side and lobe), nodule diameter and area, presence of calcification, cavity, spiculation, pleural thickening, and adhesion. Laboratory tests encompassed hematologic and biochemical indices such as white blood cell count (WBC), neutrophil-to-lymphocyte ratio (NLR), platelet-to-lymphocyte ratio (PLR), albumin/globulin ratio (AGR), liver and renal function markers, and tumor biomarkers including CEA, Cyfra21-1, and NSE.

To facilitate reproducibility and verification, all key raw data used in this study have been deposited on the Research Data Deposit public platform (www.researchdata.org.cn) under approval number xxxxxx.


\subsection{Overview of the PANDA Framework}

Cross-institutional deployment of medical AI systems faces three critical challenges that significantly impede clinical adoption: (1) \textit{limited sample sizes} due to the inherent rarity of medical conditions and privacy constraints that restrict data sharing across institutions, (2) \textit{distributional heterogeneity} arising from systematic differences in patient populations, clinical protocols, and measurement equipment across hospitals, and (3) \textit{domain shift} where models trained at one institution often exhibit degraded performance when deployed at different clinical sites due to varying data collection practices and patient demographics. Traditional machine learning approaches struggle with these constraints, particularly in medical screening applications where high sensitivity is paramount and false negatives carry severe clinical consequences.

The PANDA framework addresses these fundamental challenges through a principled integration of pre-trained foundation models and unsupervised domain adaptation, specifically designed for cross-institutional medical AI deployment. As illustrated in Figure~\ref{fig:model_details}a, PANDA operates through a three-stage pipeline: pre-training on synthetic datasets to establish generalizable tabular reasoning capabilities, training with feature selection and model adaptation to optimize performance on limited medical data, and prediction with unsupervised domain adaptation to maintain robust performance across institutional boundaries.

This end-to-end design enables robust cross-institutional generalization by combining the representational power of foundation models with principled domain adaptation. The framework is particularly suited for medical applications where training data is scarce, class distributions are imbalanced, and deployment across different clinical sites is required.

\paragraph{Implementation Overview.} The PANDA framework implementation encompasses three interconnected stages that build upon each other to achieve robust cross-institutional medical AI deployment.

\textbf{Stage 1 (Pre-train)} establishes the foundational capabilities through extensive pre-training on synthetic tabular datasets, as depicted in the lower section of Figure~\ref{fig:model_details}a. The synthetic task generator creates diverse classification problems with varying statistical patterns, feature types, and data distributions, feeding synthetic datasets to train the Pre-trained Tabular Foundation Model that acquires generalizable tabular reasoning capabilities without requiring massive medical datasets.

\textbf{Stage 2 (Train)} involves the medical-specific adaptation illustrated in Figure~\ref{fig:model_details}a. The feature selection process (shown on the left) identifies the most discriminative clinical variables, producing the Development Cohort (Cohort A) with Features + Label. This cohort then undergoes the sophisticated data preprocessing pipeline detailed in Figure~\ref{fig:model_details}b, which includes four parallel branches with feature order rotation, distribution transformations, and categorical encoding strategies, ultimately feeding into the Pre-trained Tabular Foundation Model.

\textbf{Stage 3 (Predict)} addresses the cross-institutional deployment challenge through unsupervised domain adaptation, as shown in Figure~\ref{fig:model_details}a. The UDA process transforms target domain data into the Adapted Cohort (Cohort B' without label), which then follows the identical processing pathway as the training data: through the data preprocessing pipeline (Figure~\ref{fig:model_details}b) and the Pre-trained Tabular Foundation Model, ultimately generating Output Predicted Class Probabilities. This three-stage architecture systematically addresses the challenges of data scarcity, feature discriminativeness, and domain shift inherent in cross-institutional medical AI deployment.

\begin{figure}[htbp]
    \centering
    \includegraphics[width=\linewidth]{Pre-trained Tabular Foundation Mode Pipeline_new.pdf}
    \caption{\textbf{The PANDA framework architecture.}
    (a) Detailed architecture of the pre-trained tabular foundation model showing the complete pipeline from original tabular data through ensemble training, prediction aggregation, class imbalance adjustment, to final classification output.
    (b) Detailed data preprocessing pipeline showing four parallel branches for ensemble diversity. Each branch applies feature order rotation followed by different combinations of distribution transformation (no transformation or quantile transformation) and categorical feature handling (treating as numeric or ordinal encoding). Each branch generates 16 parallel inferences, and all branches are combined through ensemble aggregation to produce the final prediction.}
    \label{fig:model_details}
\end{figure}

\subsection{Data Preprocessing}

Medical AI systems frequently fail when deployed across different hospitals due to subtle but systematic data preprocessing inconsistencies that are often overlooked in single-institution studies. Unlike standardized imaging or genomic data, tabular clinical data exhibits profound institutional variations in feature ordering (demographics-first versus lab-values-first organization), measurement protocols (different equipment calibrations and units), and categorical encoding schemes (institution-specific staging systems and risk classifications). These seemingly minor preprocessing differences can cause substantial performance degradation, with models trained at one institution showing accuracy drops of 10-20\% when deployed elsewhere, severely limiting the real-world impact of medical AI innovations.

Cross-institutional medical datasets present unique preprocessing challenges that demand principled solutions rather than ad-hoc standardization approaches. Traditional preprocessing methods that impose uniform transformations across institutions risk suppressing important clinical variations or introducing systematic biases that favor specific data distributions. The PANDA framework addresses these challenges through a sophisticated preprocessing pipeline designed to maximize ensemble diversity while preserving clinical interpretability and cross-institutional robustness.

The preprocessing strategy tackles three fundamental issues in medical tabular data that are particularly critical for cross-institutional deployment: positional bias from arbitrary feature ordering, distributional heterogeneity across institutions, and inconsistent categorical variable representations. These three issues were prioritized based on empirical analysis of multi-institutional medical datasets, which revealed that (1) Transformer-based models exhibit sensitivity to feature ordering despite theoretical position invariance, (2) clinical measurement protocols and equipment calibrations vary significantly across hospitals, leading to systematic distributional shifts, and (3) categorical medical variables (e.g., staging systems, risk classifications) often employ institution-specific encoding schemes that compromise model transferability.

Rather than applying uniform transformations that may inadvertently suppress important institutional variations or introduce systematic biases favoring specific data distributions, PANDA employs a diversified preprocessing approach that generates multiple complementary data representations. This strategy allows the model to learn robust patterns across different data perspectives while preserving the natural variability that reflects real-world clinical heterogeneity.

This multi-faceted preprocessing pipeline integrates three synergistic components that work collectively to enhance cross-domain robustness. Each component addresses specific aspects of data heterogeneity while contributing to overall ensemble diversity, enabling the model to learn from different perspectives of the same clinical information without losing essential medical semantics.

\subsubsection{Feature Rotation}

Clinical datasets often exhibit arbitrary feature ordering that can introduce systematic biases in Transformer-based models, despite their theoretical position invariance. Different medical institutions may organize clinical variables in varying sequences (e.g., demographics first vs. laboratory values first), creating subtle but persistent ordering patterns that models may inadvertently exploit. This rotation mechanism serves two critical purposes: (1) it eliminates systematic biases that could arise from arbitrary feature ordering in clinical datasets, and (2) it forces the Transformer encoder to learn position-invariant representations, enhancing robustness to feature arrangement variations across different clinical institutions.

To address this challenge, feature rotation introduces positional diversity across ensemble members to counteract ordering biases inherent in the Transformer architecture. Each ensemble member applies a cyclical permutation to input features before processing:

\[
\mathbf{x}^{(k)}_{\text{rotated}} = \text{rotate}(\mathbf{x}, k) = [x_{(k) \bmod d}, x_{(k+1) \bmod d}, \ldots, x_{(k+d-1) \bmod d}]
\]

\noindent
where $k \in [0, K_{\max})$ represents the rotation offset specific to ensemble member $k$, and $d$ is the feature dimensionality. The rotation offset is generated using a deterministic sequence starting from a random seed:

\[
k_i = (\text{start} + i) \bmod K_{\max}, \quad i = 0, 1, \ldots, N-1
\]

\noindent
where $\text{start} \sim \text{Uniform}(0, K_{\max})$ and $N=64$ ensemble members. To ensure diversity, rotation offsets are sampled without replacement to guarantee unique permutations across ensemble members. The rotation ensures consistent and reproducible feature permutations during both training and inference phases.

\subsubsection{Adaptive Feature Transformation}

Medical institutions employ diverse measurement protocols, equipment calibrations, and laboratory standards that result in systematic distributional differences across sites. For instance, blood biomarker values may exhibit institution-specific ranges due to different assay methods, while imaging measurements can vary based on scanner manufacturers and acquisition protocols. These distributional heterogeneities pose significant challenges for cross-institutional model deployment, as models trained on one institution's data distribution may perform poorly when applied to data from institutions with different measurement characteristics.

To address this challenge, the preprocessing pipeline employs a dual-strategy approach to balance distribution normalization with feature preservation, addressing the heterogeneous nature of clinical data. Two complementary transformation strategies are implemented:

\noindent
\textbf{Enhanced Feature Transformation:} Applies quantile transformation followed by dimensionality expansion:

\[
\mathbf{x}_{\text{quantile}} = \text{QuantileTransformer}(\mathbf{x}, n_{\text{quantiles}} = \max(\lfloor n_{\text{samples}}/10 \rfloor, 2))
\]

\noindent
where the quantile transformer maps each feature to a uniform distribution $U(0,1)$ using empirical quantiles. Following quantile transformation, SVD-based dimensionality expansion is applied:

\[
\mathbf{X}_{\text{expanded}} = \text{SVD}(\mathbf{X}_{\text{quantile}}, n_{\text{components}} = \min(4, d))
\]

\noindent
The final feature representation concatenates original and transformed features:

\[
\mathbf{x}_{\text{final}} = [\mathbf{x}_{\text{original}}; \mathbf{x}_{\text{quantile}}; \mathbf{x}_{\text{SVD}}]
\]

\noindent
This increases dimensionality from 7 to 18 features (7 original + 7 quantile + 4 SVD), enhancing the model's representational capacity.

\noindent
\textbf{Preserved Feature Transformation:} Preserves raw feature distributions through identity transformation:

\[
\mathbf{x}_{\text{preserved}} = \mathbf{x}_{\text{original}}
\]

\noindent
This maintains natural scale and distribution characteristics of clinical variables, resulting in unchanged 7-dimensional feature vectors. This configuration ensures that ensemble diversity encompasses both normalized and raw feature representations.

\subsubsection{Intelligent Categorical Encoding}

Categorical variable encoding represents a critical challenge in medical AI systems, where inappropriate encoding strategies can fundamentally distort model learning and lead to spurious clinical predictions. The core issue lies in the fact that naive numerical encoding (0, 1, 2, …) artificially imposes ordinal relationships on categorical variables that may be purely nominal, causing models to learn false mathematical relationships. For instance, if a blood type feature is encoded as A=0, B=1, AB=2, O=3, the model might incorrectly learn that AB is “twice as much” as B, or that there is a meaningful progression from A to O. Such artificial numerical relationships can lead to clinically meaningless predictions and compromise model reliability.

Traditional one-hot encoding, while avoiding artificial ordinality, becomes computationally inefficient and may not capture the full diversity needed for robust ensemble training. Even when features are provided as anonymized identifiers (Feature01, Feature02, etc.), intelligent encoding strategies are essential to prevent the model from learning spurious numerical patterns while maintaining the diversity necessary for robust cross-institutional deployment.

To maximize robustness and ensemble diversity, categorical feature encoding employs two distinct strategies applied across different branches of the preprocessing pipeline:

\noindent
\textbf{Ordinal Encoding with Frequency Filtering:} This strategy applies selective ordinal encoding with frequency-based filtering:

\[
\text{encode}(x_{ij}) = \begin{cases}
\phi_j(x_{ij}) & \text{if feature $j$ has frequently occurring categories} \\
x_{ij} & \text{otherwise}
\end{cases}
\]

\noindent
where $x_{ij}$ represents the categorical value of feature $j$ for sample $i$, and frequently occurring categories are defined as features with individual category counts $\geq 10$ and total unique categories $|U_j| < n_{\text{samples}}/10$. The ordinal mapping employs randomized category-to-integer assignment for ensemble diversity:

\[
\phi_j = \pi(\{0, 1, \ldots, |U_j|-1\})
\]

\noindent
where $\phi_j$ represents the ordinal mapping function for feature $j$, $\pi(\cdot)$ denotes a random permutation operator, and $U_j$ is the set of unique categorical values in feature $j$.

This approach ensures that only sufficiently represented categories undergo ordinal encoding, preventing overfitting to rare categorical values common in medical datasets.

\noindent
\textbf{Numeric Treatment Strategy:} This strategy treats categorical features as continuous numeric values:

\[
\text{encode}(x_{ij}) = \text{float}(x_{ij})
\]

This approach enables direct processing of categorical variables through quantile transformations and other numerical operations, particularly effective for ordinal categorical variables with natural numeric interpretations.

\paragraph{Overall Ensemble Configuration Summary}

The complete 64-member ensemble integrates all preprocessing components through a systematic 4-branch design as illustrated in Figure~\ref{fig:model_details}b:

\begin{itemize}
    \item \textbf{Branch 1} (No Distribution Transformation + Numeric Categorical): 16 ensemble members
    \item \textbf{Branch 2} (No Distribution Transformation + Ordinal Encoding): 16 ensemble members
    \item \textbf{Branch 3} (Quantile Transformation + Numeric Categorical): 16 ensemble members
    \item \textbf{Branch 4} (Quantile Transformation + Ordinal Encoding): 16 ensemble members
\end{itemize}

Within each branch, the 16 ensemble members are differentiated through unique feature rotation patterns, ensuring comprehensive coverage of both transformation strategies and categorical encoding approaches. This systematic design guarantees balanced representation across all preprocessing variations while maximizing ensemble diversity for robust medical predictions across institutional boundaries.


\subsection{Feature Selection}

Robust feature selection is indispensable for cross-institutional medical AI, where the data landscape is typically high-dimensional, heterogeneous, and severely constrained in sample size. Without principled selection, models risk overfitting to site-specific noise or redundant variables, undermining both predictive performance and generalizability. To directly address these challenges, we employed recursive feature elimination (RFE) powered by a Pre-trained Tabular Foundation Model as the base estimator (Figure~\ref{fig:feature_selection_uda}a). This approach enables the identification of clinically meaningful and highly discriminative variables while simultaneously minimizing overfitting risk.

RFE relies on permutation-based feature importance for iterative elimination. Permutation importance measures each feature’s contribution by randomly shuffling its values across samples and observing the resulting performance degradation—larger performance drops indicate more important features. This model-agnostic evaluation is particularly suited to pre-trained foundation models, where gradient-based or weight-based importance scores are not directly interpretable due to multiple attention layers and ensemble components. By directly quantifying each feature’s predictive contribution, the permutation approach offers a transparent and robust mechanism to rank features regardless of architectural complexity.

This wrapper-based strategy iteratively removes the least informative features to produce a compact, domain-consistent subset optimized for downstream classification. As shown in Figure~\ref{fig:feature_selection_uda}a, our pipeline systematically applies RFE to Original Cohort A and Original Cohort B—each comprising 58 clinical variables with labels—yielding refined Development (Cohort A) and Validation (Cohort B) cohorts. This cross-institutional feature selection ensures that the retained features maintain high predictive power and clinical interpretability across different hospital settings.

\begin{figure}[htbp]
    \centering
    \includegraphics[width=\linewidth]{Feature Selection and UDA.pdf}
    \caption{\textbf{Feature Selection and Unsupervised Domain Adaptation in the PANDA framework.} (a) \textbf{Feature Selection}: Recursive Feature Elimination (RFE) with Pre-trained Tabular Foundation Model systematically identifies the most discriminative clinical variables from the original 58-feature sets, reducing dimensionality while preserving predictive power to generate optimized feature subsets for both Development Cohort (Cohort A) and Validation Cohort (Cohort B). (b) \textbf{Unsupervised Domain Adaptation}: Transfer Component Analysis (TCA) performs distributional alignment between source and target domains without using labels, transforming the Validation Cohort (Cohort B) into an Adapted Cohort (Cohort B') that maintains clinical interpretability while being optimally aligned with the source domain distribution for robust cross-institutional prediction.}
    \label{fig:feature_selection_uda}
\end{figure}

\paragraph{Algorithm Overview.}
Given a dataset $\mathcal{D} = \{(\mathbf{x}_i, y_i)\}_{i=1}^n$ with $\mathbf{x}_i \in \mathbb{R}^d$ and $y_i \in \{0, 1\}$, the RFE algorithm selects $k < d$ features by repeating the following steps:
\begin{enumerate}
    \item Train the Pre-trained Tabular Foundation Model $f_\Theta^{(t)}$ on the current feature subset $\mathcal{F}^{(t)}$.
    \item Estimate feature importance scores $\mathbf{I}^{(t)} = [I^{(t)}_1, I^{(t)}_2, \dots, I^{(t)}_{|\mathcal{F}^{(t)}|}]$ using permutation-based evaluation.
    \item Eliminate the feature with the lowest importance score:\\
    $\mathcal{F}^{(t+1)} \leftarrow \mathcal{F}^{(t)} \setminus \{\arg\min_j I^{(t)}_j\}$.
    \item Repeat until the target feature count $|\mathcal{F}^{(t+1)}| = k$ is reached.
\end{enumerate}

\paragraph{Permutation Importance.}
To assess the contribution of each feature, we adopt permutation importance, a robust and model-agnostic metric. For a given feature $x_j$, its importance score is defined as the expected decrease in performance upon random shuffling:

\[
I_j = \frac{1}{R} \sum_{r=1}^{R} \left[ \mathrm{AUC}(f_\Theta, \mathcal{D}) - \mathrm{AUC}(f_\Theta, \mathcal{D}_{\text{perm}(j)}^{(r)}) \right],
\]

\noindent
where $\mathcal{D}_{\text{perm}(j)}^{(r)}$ denotes the dataset with the $j$-th feature randomly permuted in the $r$-th repetition, and $R$ is the number of repeats (we use $R=5$). Larger $I_j$ values indicate stronger influence on model predictions.

\paragraph{Implementation.}
As illustrated in Figure~\ref{fig:feature_selection_uda}a, the feature selection process begins with Original Cohort A and Original Cohort B, both containing the full set of $d=58$ clinical features with their corresponding labels. We applied RFE with the Pre-trained Tabular Foundation Model to both cohorts simultaneously to identify features with consistent discriminative power across institutions. This cross-cohort approach ensures that selected features maintain their predictive value in both source and target domains.

The RFE process systematically eliminated weakly contributing variables using permutation importance evaluation across both cohorts, ultimately yielding a ranked list of the top 9 features. However, one of these features (Feature40) was unavailable in the Original Cohort B, creating an inconsistency for cross-domain deployment. To ensure robust cross-institutional applicability, we excluded Feature40 and retained the remaining 7 features, producing the final Development Cohort (Cohort A) and Validation Cohort (Cohort B) with consistent feature subsets. This cross-validated feature selection strategy ensures that the selected clinical variables maintain high discriminative power across different institutional settings.


\paragraph{Multi-dimensional Performance Analysis.}
Beyond simple accuracy metrics, our RFE analysis incorporates multiple evaluation dimensions to ensure robust feature selection for clinical deployment. The following three analyses provide complementary perspectives on feature subset quality: 
Figure~\ref{fig:rfe-performance}b presents class-specific accuracy analysis across different feature subset sizes. The balanced performance between malignant (positive) and benign (negative) cases demonstrates that our feature selection process maintains diagnostic sensitivity across both clinical scenarios. This balanced predictive capability is crucial for medical screening applications, where both false positives and false negatives carry significant clinical consequences.

\paragraph{Computational Efficiency Assessment.}
Training time complexity is a critical consideration for clinical deployment scalability. Figure~\ref{fig:rfe-performance}c illustrates the computational efficiency as a function of feature dimensionality, measured in seconds per cross-validation fold. The analysis reveals a near-linear relationship between feature count and training time, with the 9-features configuration achieving an optimal balance between computational efficiency and predictive performance. This efficiency profile supports real-time clinical decision-making requirements while maintaining model interpretability.

\paragraph{Performance Stability Evaluation.}
Model reliability in clinical settings requires consistent performance across different data splits and patient populations. Figure~\ref{fig:rfe-performance}d presents performance stability assessment using coefficient of variation (CV) across 10-fold cross-validation. Lower CV values indicate more stable and reliable performance, with our selected 9-features subset demonstrating superior stability compared to both smaller and larger feature configurations. This stability analysis ensures that the selected features provide robust predictions across diverse clinical scenarios and patient demographics.

\paragraph{Multi-criteria Optimization Framework.}
Feature selection in medical applications requires balancing multiple competing objectives beyond simple predictive accuracy. Clinical deployment demands consideration of computational efficiency for real-time decision-making, performance stability across diverse patient populations, and model simplicity for clinical interpretability and regulatory compliance. Rather than optimizing for a single metric, which may lead to suboptimal solutions that excel in one dimension while failing in others, we developed a principled multi-criteria optimization approach.

To identify the globally optimal feature subset, we developed a comprehensive cost-effectiveness index that integrates multiple performance dimensions (Figure~\ref{fig:rfe-performance}e). Let $n$ denote the number of features in a given subset. The composite metric is defined as a weighted combination of four normalized scores:

\[
\text{CostEffectiveness}(n) = w_1 \cdot S_{\text{perf}}(n) + w_2 \cdot S_{\text{eff}}(n) + w_3 \cdot S_{\text{stab}}(n) + w_4 \cdot S_{\text{simp}}(n)
\]

\noindent
where $(w_1, w_2, w_3, w_4) = (0.45, 0.15, 0.15, 0.25)$ represent theory-driven weights prioritizing performance for medical applications, with balanced consideration of stability, efficiency, and Occam's Razor simplicity. Each component score is normalized to $[0,1]$ as follows:

\begin{itemize}
    \item \textbf{Performance Score} ($S_{\text{perf}}$): A weighted combination of classification metrics:
    \[
    S_{\text{perf}}(n) = 0.5 \cdot \text{AUC}(n) + 0.3 \cdot \text{Accuracy}(n) + 0.2 \cdot \text{F1}(n)
    \]

    \item \textbf{Efficiency Score} ($S_{\text{eff}}$): Training time efficiency, normalized using min-max scaling:
    \[
    S_{\text{eff}}(n) = 1 - \frac{T(n) - T_{\min}}{T_{\max} - T_{\min}}
    \]
    where $T(n)$ is the mean training time for $n$ features, and shorter training times yield higher scores.

    \item \textbf{Stability Score} ($S_{\text{stab}}$): Cross-validation consistency based on performance variance:
    \[
    S_{\text{stab}}(n) = 1 - \frac{\bar{\sigma}(n) - \bar{\sigma}_{\min}}{\bar{\sigma}_{\max} - \bar{\sigma}_{\min}}
    \]
    where $\bar{\sigma}(n) = \frac{1}{3}[\sigma_{\text{AUC}}(n) + \sigma_{\text{Accuracy}}(n) + \sigma_{\text{F1}}(n)]$ is the average standard deviation across metrics.

    \item \textbf{Simplicity Score} ($S_{\text{simp}}$): Model complexity penalty following Occam's Razor principle, implemented as an exponential decay function that naturally favors simpler models:
    \[
    S_{\text{simp}}(n) = \exp(-\alpha \cdot n)
    \]
    where $\alpha = 0.015$ is the complexity penalty coefficient. This pure mathematical function ensures that fewer features always receive higher scores, strictly adhering to Occam's Razor without arbitrary cutoffs or baseline assumptions.
\end{itemize}

The optimization problem becomes:
\[
n^* = \arg\max_{n} \text{CostEffectiveness}(n)
\]

This multi-criteria optimization framework objectively determines that the 9-features configuration ($n^* = 9$) achieves the globally optimal trade-off across all evaluation dimensions. The theory-driven weight allocation prioritizes performance (0.45) and simplicity (0.25) as the primary concerns in medical applications, while maintaining balanced consideration of stability (0.15) and efficiency (0.15). The exponential decay simplicity function $\exp(-0.015 \cdot n)$ ensures pure adherence to Occam's Razor without arbitrary assumptions, naturally favoring simpler models while allowing empirical performance metrics to determine the optimal balance point. This methodology is both theoretically sound and practically deployable in real-world clinical environments, as evidenced by the peak in Figure~\ref{fig:rfe-performance}e at precisely 9 features.

\paragraph{Cross-Domain Feature Consistency.}
The cross-domain feature alignment process is detailed in Figure~\ref{fig:feature_selection_uda}a. Starting with Original Cohort A and Original Cohort B, each containing the full set of 58 clinical features, recursive feature elimination with the Pre-trained Tabular Foundation Model identifies the most discriminative variables across both institutions. To ensure robust cross-institutional deployment, we account for varying feature availability across different clinical sites by applying RFE simultaneously to both cohorts. This cross-validated approach produces the Development Cohort (Cohort A) and Validation Cohort (Cohort B) with consistent feature subsets that are available and discriminative in both institutions, guaranteeing reliable model deployment while preserving the most clinically relevant variables for malignancy prediction.





\subsection{Pre-trained Tabular Foundation Model}

Medical tabular classification faces unique challenges that distinguish it from traditional machine learning applications. Medical datasets typically contain heterogeneous feature types (continuous measurements, categorical variables, ordinal scales) with complex non-linear interactions that are difficult to capture using conventional algorithms. Furthermore, small sample sizes relative to feature dimensionality create high variance in model predictions, while institutional differences in data collection protocols lead to distribution shifts that compromise generalization. Traditional machine learning approaches often fail to adequately model these complex feature interactions while maintaining robustness across different clinical environments.

Foundation models have revolutionized artificial intelligence by demonstrating remarkable capabilities in learning generalizable representations from large-scale data that can be adapted to diverse downstream tasks. These models, exemplified by large language models in NLP and vision transformers in computer vision, leverage massive pre-training on heterogeneous datasets to acquire broad knowledge that facilitates few-shot learning and cross-domain transfer. The success of foundation models stems from their ability to learn universal patterns and relationships during pre-training that generalize beyond specific tasks or domains. This paradigm shift from task-specific model training to pre-training followed by adaptation offers particular promise for medical applications, where data scarcity and domain heterogeneity are prevalent challenges.

To address these challenges, we adopt and adapt TabPFN (Tabular Prior-Fitted Networks), a pre-trained Transformer-based foundation model for tabular data~\cite{hollmann2025accurate}. While TabPFN's core architecture treats each feature as a token and uses self-attention mechanisms for feature interaction modeling, our contribution lies in developing specialized preprocessing configurations and ensemble strategies specifically optimized for cross-institutional medical classification tasks. We enhance the original TabPFN framework with domain-adaptive preprocessing pipelines and systematic ensemble diversification strategies that address the unique challenges of medical tabular data across different clinical environments.

\subsubsection{Architecture and Feature Encoding}

\paragraph{Per-Feature Transformer Architecture.}
Each structured input sample $\mathbf{x} = [x_1, x_2, \ldots, x_d] \in \mathbb{R}^d$ is treated as a sequence of tokens, where each feature $x_i$ corresponds to an individual token. This per-feature tokenization enables the model to learn feature-specific representations while modeling inter-feature dependencies through attention mechanisms.

The feature encoding process follows a multi-step transformation:
\[
\mathbf{e}_i = \text{Embed}(x_i) + \mathbf{p}_i, \quad i = 1, \ldots, d
\]
where $\text{Embed}(\cdot): \mathbb{R} \rightarrow \mathbb{R}^{d_{\text{model}}}$ maps each feature value to a $d_{\text{model}}$-dimensional embedding space (where $d_{\text{model}} \in \{128, 192\}$ depending on model configuration), and $\mathbf{p}_i \in \mathbb{R}^{d_{\text{model}}}$ represents learned positional encodings that capture feature ordering information.

The embedded sequence $\mathbf{E} = [\mathbf{e}_1, \mathbf{e}_2, \ldots, \mathbf{e}_d]$ is processed through a 12-layer Transformer encoder. Each layer $\ell$ applies multi-head self-attention followed by a feedforward network:
\[
\mathbf{H}^{(\ell)} = \text{LayerNorm}(\text{MultiHead}(\mathbf{H}^{(\ell-1)}) + \mathbf{H}^{(\ell-1)})
\]
\[
\mathbf{H}^{(\ell+1)} = \text{LayerNorm}(\text{FFN}(\mathbf{H}^{(\ell)}) + \mathbf{H}^{(\ell)})
\]
where $\mathbf{H}^{(0)} = \mathbf{E}$, and the feedforward network $\text{FFN}$ has $4 \times d_{\text{model}}$ hidden units. The multi-head attention mechanism employs 4 or 6 attention heads (depending on configuration), enabling the model to capture diverse feature interaction patterns simultaneously.

\subsubsection{Pre-training}

\paragraph{Motivation for Pre-training.}
Medical tabular datasets present fundamental challenges that necessitate pre-training approaches. First, medical institutions typically possess small, specialized datasets that are insufficient for training robust deep learning models from scratch. The limited sample sizes (often hundreds to thousands of cases) combined with high-dimensional feature spaces create severe overfitting risks when using conventional supervised learning. Second, medical datasets exhibit significant heterogeneity across institutions due to differences in measurement protocols, equipment calibration, and patient populations, leading to substantial domain shift that compromises model generalizability. Third, unlike vision or language domains where massive public datasets exist, medical data sharing is severely constrained by privacy regulations and institutional policies, preventing the assembly of large-scale training corpora.

These challenges necessitate models that can: (1) extract generalizable patterns from diverse statistical distributions and feature-label relationships across multiple domains, (2) adapt to new medical classification tasks through in-context learning without requiring parameter updates or fine-tuning, and (3) maintain robustness against distributional shifts by understanding diverse data-generating processes inherent in clinical environments.

\paragraph{Synthetic Task Generation Strategy.}
Deep neural networks require substantial training data to learn effective representations, yet collecting and labeling extensive medical datasets is prohibitively expensive, time-intensive, and often infeasible due to privacy constraints. To address this data scarcity while ensuring broad coverage of tabular reasoning patterns, we employ a stochastic task generator that synthesizes classification problems from diverse function priors.

This synthetic data generation strategy enables the model to experience a vast range of tabular data characteristics, statistical patterns, and classification scenarios that would be impossible to encounter through real medical datasets alone. The task generator samples problems from a mixture of function priors, creating diverse synthetic classification tasks that collectively teach the model generalizable tabular reasoning capabilities applicable to real-world medical classification problems with varying sample sizes and domain characteristics.

Let $\mathcal{X}\subset\mathbb{R}^d$ and $\mathcal{Y}=\{1,\dots,C\}$ for classification (or $\mathbb{R}$ for regression). For each training batch, we first sample a prior family and its hyperparameters:
\[
r \sim \mathrm{Categorical}(\boldsymbol{\pi}),\qquad
\boldsymbol{\theta} \sim p(\boldsymbol{\theta}\mid r),
\]
where $r\in\{\text{gp},\text{mlp},\text{ridge},\text{mix\_gp}\}$ indexes different function priors including Gaussian Process (GP), Multi-Layer Perceptron (MLP), ridge regression, and mixed GP priors, $\boldsymbol{\pi}$ represents probability weights for prior family selection, and $\boldsymbol{\theta}$ is a hyperparameter vector containing kernel/architecture specifications, noise level, class count, and input scaling parameters.

We draw $T=n_{\text{ctx}}+n_{\text{eval}}$ inputs where $n_{\text{ctx}}$ is the number of context (training) examples per synthetic task, $n_{\text{eval}}$ is the number of evaluation (test) examples per synthetic task, and $T$ is the total sequence length. Inputs are sampled independently from a factorized base distribution and optionally transformed:
\[
\mathbf{x}_t \sim p_{\text{base}}(\mathbf{x}), \qquad \tilde{\mathbf{x}}_t = \psi_{\boldsymbol{\theta}}(\mathbf{x}_t)
\]
where $p_{\text{base}}(\mathbf{x})$ is a factorized base distribution (often uniform or Gaussian per feature) and $\psi_{\boldsymbol{\theta}}(\cdot)$ is a feature transform function (e.g., quantile-to-normal normalization, standardization).
\[
\mathbf{x}_t \sim p_X(\cdot\mid \boldsymbol{\theta}_X)=\prod_{j=1}^d p_{X_j}(x_{t,j}), 
\qquad 
\tilde{\mathbf{x}}_t=\psi_{\boldsymbol{\theta}}(\mathbf{x}_t), 
\quad t=1,\dots,T.
\]
Conditioned on $(r,\boldsymbol{\theta})$, a random function $f_\Theta$ generates latent outputs. For a GP prior ($r=\text{gp}$), each class logit is an independent draw from a GP with zero mean and kernel $k_{\boldsymbol{\theta}}$ (e.g., RBF with length-scale $\ell$ and output scale $\sigma_f^2$):
\[
f_c \sim \mathrm{GP}\!\big(0,k_{\boldsymbol{\theta}}\big),\qquad 
\mathbf{z}_t=\big(f_1(\tilde{\mathbf{x}}_t),\dots,f_C(\tilde{\mathbf{x}}_t)\big)\in\mathbb{R}^C .
\]
For a random–MLP prior ($r=\text{mlp}$), we sample depth/width and weights,
\[
L\sim p(L),\ \ h_\ell\sim p(h_\ell),\ \ 
\mathbf{W}_\ell \sim \mathcal{N}\!\Big(0,\frac{\sigma_w^2}{\mathrm{fan\_in}_\ell}\mathbf{I}\Big),\ \ 
\mathbf{b}_\ell \sim \mathcal{N}(0,\sigma_b^2\mathbf{I}),
\]
and define
\[
\mathbf{h}_0=\tilde{\mathbf{x}}_t,\quad 
\mathbf{h}_\ell=\phi\!\big(\mathbf{W}_\ell\mathbf{h}_{\ell-1}+\mathbf{b}_\ell\big)\ (\ell=1,\dots,L-1),\quad
\mathbf{z}_t=\mathbf{W}_L\mathbf{h}_{L-1}+\mathbf{b}_L,
\]
with nonlinearity $\phi$ (e.g., ReLU). For ridge regression priors ($r=\text{ridge}$), linear models with L2 regularization are employed to generate smooth, generalizable functions. For mixed GP priors ($r=\text{mix\_gp}$), multiple Gaussian Process kernels are combined to capture diverse statistical relationships and function characteristics across different length scales and patterns. Observation noise models variability:
\[
\boldsymbol{\varepsilon}_t \sim \mathcal{N}\big(\mathbf{0},\sigma^2\mathbf{I}\big).
\]
The model output dimensionality $n_{\text{out}}$ is determined by the task type:
\[
n_{\text{out}} = \begin{cases}
2 & \text{for regression with uncertainty (GaussianNLLLoss: mean and variance)} \\
C & \text{for } C\text{-class classification (CrossEntropyLoss)} \\
1 & \text{for binary classification (BCEWithLogitsLoss) or basic regression}
\end{cases}
\]
For regression, $y_t=z_t+\varepsilon_t\in\mathbb{R}$; optionally a bucketized ("bar") likelihood is used by choosing bin borders $b_0<\cdots<b_B$ (e.g., prior-predictive quantiles) and training a categorical density over bins. For classification, temperature-scaled logits yield class probabilities and labels,
\[
\mathbf{p}_t=\mathrm{softmax}\!\big(\mathbf{z}_t/\tau\big),\qquad 
y_t \sim \mathrm{Categorical}(\mathbf{p}_t),
\]
with an optional random class permutation to decorrelate semantic labels across tasks. The first $n_{\text{ctx}}$ pairs form the context $\mathcal{D}_{\text{ctx}}=\{(\tilde{\mathbf{x}}_t,y_t)\}_{t=1}^{n_{\text{ctx}}}$; the remaining inputs $\mathcal{Q}=\{\tilde{\mathbf{x}}_t\}_{t=n_{\text{ctx}}+1}^{T}$ are queries whose labels are withheld during the forward pass. We concatenate $(\mathcal{D}_{\text{ctx}},\mathcal{Q})$ into a single sequence for in-context conditioning and train the Transformer to predict $\{y_t\}_{t=n_{\text{ctx}}+1}^{T}$. The task distribution and objective are
\[
p(\mathcal{T})
=\sum_{r}\pi_r\!\int p(\boldsymbol{\theta}\mid r)
\prod_{t=1}^{T}\! \Big[p_X(\mathbf{x}_t\mid\boldsymbol{\theta})\, p\!\big(y_t\mid \tilde{\mathbf{x}}_t,\boldsymbol{\theta},r\big)\Big]\,
\mathrm{d}\boldsymbol{\theta},
\qquad
\min_{\Theta}\ \mathbb{E}_{\mathcal{T}\sim p(\cdot)}\!\left[\sum_{t=n_{\text{ctx}}+1}^{T} 
\ell\!\big(h_{\Theta}(\mathcal{D}_{\text{ctx}},\mathcal{Q})_t,\ y_t\big)\right],
\]
where $\ell$ is the task-specific loss function determined by task type and output configuration:
\begin{align}
\ell = \begin{cases}
\text{BCEWithLogitsLoss}(\mathbf{z}_t, y_t) & \text{for binary classification} \\
\text{CrossEntropyLoss}(\mathbf{z}_t, y_t) & \text{for multi-class classification} \\
\text{MSELoss}(z_t, y_t) & \text{for basic regression} \\
\text{GaussianNLLLoss}(\mu_t, y_t, |\sigma_t|) & \text{for regression with uncertainty, where } \\
& \quad \mu_t = \mathbf{z}_t[0], \sigma_t = \mathbf{z}_t[1] \\
\text{FullSupportBarDistribution}(\mathbf{z}_t, y_t) & \text{for discretized regression}
\end{cases}
\end{align}

\paragraph{Pre-training Details.}
The foundation model pre-training follows a systematic procedure designed to optimize learning across diverse synthetic tabular tasks. Pre-training is conducted using AdamW optimizer with learning rate determined by the OpenAI scaling law: $\text{lr} = 0.003 \cdot \sqrt{d_{\text{model}}/512}$, where $d_{\text{model}}$ is the model embedding dimension. The learning rate schedule employs cosine annealing with linear warmup over the first 50 epochs, followed by cosine decay over the remaining pre-training duration.

Each pre-training epoch processes multiple synthetic tasks in parallel with batch size of 1000 sequences. Gradient accumulation is employed over multiple batches before parameter updates, with gradient clipping at norm 1.0 to ensure pre-training stability. Mixed precision training using automatic mixed precision (AMP) is utilized to accelerate computation and reduce memory requirements while maintaining numerical stability.

The pre-training objective employs positional loss computation, where losses are calculated for each position in the sequence and averaged across valid positions. For in-context learning scenarios, only the query positions (beyond $n_{\text{ctx}}$) contribute to the loss calculation:
\[
\mathcal{L}_{\text{epoch}} = \frac{1}{N} \sum_{i=1}^{N} \frac{1}{n_{\text{eval}}} \sum_{t=n_{\text{ctx}}+1}^{T} \ell\!\big(h_{\Theta}(\mathcal{D}_{\text{ctx}}^{(i)},\mathcal{Q}^{(i)})_t,\ y_t^{(i)}\big)
\]
where $N$ is the batch size, and the inner sum averages over evaluation positions. Pre-training continues until convergence, typically requiring 200-500 epochs depending on model size and task complexity.

\subsubsection{Inference}

\paragraph{In-Context Learning Mechanism.}
Real-world cross-institutional deployment faces three compounding issues: \emph{small labeled cohorts} that make fine-tuning prone to overfitting, \emph{heterogeneous feature distributions} across hospitals (protocols, equipment, populations), and \emph{domain shift} that would otherwise require site-specific retraining with substantial computational and operational cost. To address these challenges without updating parameters at deployment, we adopt an \emph{in-context learning} (ICL)–based inference design. Our Pre-trained Tabular Foundation Model uses ICL as the primary mechanism: the model internalizes task-specific patterns by observing a small set of labeled examples (context) within the input sequence and then predicts on unlabeled query samples. Compared with conventional fine-tuning, this approach (1) \textbf{improves sample efficiency under data scarcity} by leveraging pre-trained knowledge to generalize from minimal examples; (2) \textbf{facilitates cross-institutional generalization} by adapting to local characteristics without separate training per site; and (3) \textbf{reduces computational burden} by avoiding gradient computation and parameter updates during inference, enabling rapid and resource-efficient clinical deployment.

At inference time, training and test samples are concatenated to form a composite input sequence:
\[
\mathbf{X}_{\text{context}} = [\mathbf{X}_{\text{train}}; \mathbf{X}_{\text{test}}] \in \mathbb{R}^{(n_{\text{train}} + n_{\text{test}}) \times 1 \times f}
\]
where $n_{\text{train}}$ is the number of training samples (e.g., 295 for medical dataset A), $n_{\text{test}}$ is the number of test samples (e.g., 190 for medical dataset B), and $f$ is the number of features (e.g., 7 features for the \texttt{best7} configuration). The middle dimension of 1 indicates that each sample is processed as an independent batch for sequential processing, rather than processing multiple samples simultaneously in a traditional training batch. For multi-batch scenarios, this extends to $(B, n_{\text{train}} + n_{\text{test}}, f)$ where $B$ represents the number of parallel inference tasks.

Only training labels are provided during forward pass through label masking:
\[
\mathbf{y}_{\text{context}} = [\mathbf{y}_{\text{train}}; \varnothing] \in \{0,1\}^{n_{\text{train}}} \cup \{\varnothing\}^{n_{\text{test}}}
\]
where $\varnothing$ represents masked positions that the Transformer attention mechanism ignores during forward propagation. Specifically, the model applies causal masking to these positions, preventing gradient flow from test sample predictions back to model parameters. This masking enables the model to infer predictions for test samples without gradient updates, improving generalizability while maintaining computational efficiency.

\paragraph{Train.}
\textbf{Train} involves the medical-specific adaptation illustrated in Figure~\ref{fig:model_details}a. The feature selection process identifies the most discriminative clinical variables, producing the \textbf{Development Cohort (Cohort A)} with Features + Label. This cohort then undergoes the sophisticated data preprocessing pipeline detailed in Figure~\ref{fig:model_details}b, which includes four parallel branches with \emph{feature order rotation}, \emph{distribution transformations} (no transform vs.\ quantile transform), and \emph{categorical encoding strategies} (treat-as-numeric vs.\ ordinal encoding), ultimately feeding the frozen \textbf{Pre-trained Tabular Foundation Model} in an in-context manner (no parameter updates).

\paragraph{Predict.}
\textbf{Predict} addresses cross-institutional deployment through unsupervised domain adaptation, as shown in Figure~\ref{fig:model_details}a. The \textbf{UDA} process transforms target-domain data into the \textbf{Adapted Cohort (Cohort B' without label)}, which then follows the \emph{identical} processing pathway as the training data—namely, the data preprocessing pipeline (Figure~\ref{fig:model_details}b) and the \textbf{Pre-trained Tabular Foundation Model}—ultimately generating \textbf{Output Predicted Class Probabilities}. This Train→Predict design systematically addresses data scarcity, feature discriminativeness, and domain shift in cross-institutional medical AI.

\paragraph{Ensemble Inference and Diversity Strategies.}
To enhance robustness and mitigate overfitting to single parametric priors while emulating cross-site heterogeneity, the system employs a \textbf{64-member ensemble} realized by the four-branch pipeline in Figure~\ref{fig:model_details}b. Each branch applies feature order rotation followed by distinct combinations of distribution transformation and categorical encoding, creating four complementary representations that maximize ensemble diversity and improve generalization under domain shift. Each ensemble member outputs class logits, which are temperature-scaled and softmax-normalized; final predictions are obtained by averaging member probabilities:
\[
\hat{y} = \frac{1}{N} \sum_{i=1}^{N} \text{softmax}\!\left( \frac{z_i}{T} \right),
\]
where $z_i \in \mathbb{R}^{n_{\text{test}} \times C}$ denotes the logits from the $i$-th member, $T{=}0.9$ is the softmax temperature, and $N{=}64$ the number of members. Probability averaging (rather than logit averaging) preserves proper probabilistic interpretation. The four branches are:
\begin{itemize}
    \item \textbf{Branch 1}: Feature order rotation + no distribution transformation + treat categorical features as numeric (16 parallel inferences)
    \item \textbf{Branch 2}: Feature order rotation + no distribution transformation + ordinal encoding of categorical features (16 parallel inferences)
    \item \textbf{Branch 3}: Feature order rotation + quantile transformation + treat categorical features as numeric (16 parallel inferences)
    \item \textbf{Branch 4}: Feature order rotation + quantile transformation + ordinal encoding of categorical features (16 parallel inferences)
\end{itemize}
All 64 inferences (16 per branch) are aggregated to produce robust final predictions while maintaining computational efficiency.


\paragraph{Class Imbalance Correction.}
To address the intrinsic label imbalance in real-world medical datasets, inverse-frequency reweighting is applied during output aggregation when the \texttt{balance\_probabilities} flag is enabled. Given predicted class probabilities $\mathbf{p} = (p_1, \dots, p_C)$ and empirical class distribution from training data $\boldsymbol{\pi} = (\pi_1, \dots, \pi_C)$ where $\pi_i = \frac{\text{count}_i}{\sum_{j=1}^C \text{count}_j}$, the reweighted probabilities are:
\[
\hat{p}_i = \frac{p_i / \pi_i}{\sum_{j=1}^{C} p_j / \pi_j}, \quad \text{for } i = 1, \dots, C,
\]
where $\pi_i$ represents the observed frequency of class $i$ in the training dataset. This method ensures that minority class predictions are not diluted by the class imbalance, which is particularly critical in small-sample medical datasets where rare conditions must maintain adequate sensitivity for clinical screening applications.

\subsection{Transfer Component Analysis for Domain Adaptation}

Cross-institutional deployment of medical AI inevitably encounters distributional shifts arising from differences in patient demographics, measurement protocols, and institutional practices. Without explicit alignment, models trained on one hospital’s data often exhibit degraded performance when applied to another, even when the underlying clinical concepts remain stable. To mitigate this critical barrier to generalization, we incorporated Transfer Component Analysis (TCA), a kernel-based unsupervised domain adaptation method, into our preprocessing pipeline (Figure~\ref{fig:feature_selection_uda}b).

TCA projects both the Development Cohort (Cohort A without label) and the Validation Cohort (Cohort B without label) into a shared latent subspace in which their marginal distributions are statistically aligned. This process produces the Adapted Cohort (Cohort B' without label), which preserves essential clinical characteristics while reducing institution-specific biases and improving cross-domain compatibility. By learning a distribution-invariant representation without relying on labels, TCA provides a principled way to leverage all available data for model adaptation, a particularly valuable property for medical datasets where labeled data is scarce.

This unsupervised alignment directly addresses one of the most persistent challenges in medical AI—distribution drift across hospitals—ensuring that the predictive model can generalize robustly while maintaining clinical interpretability and consistency across heterogeneous healthcare settings.

\paragraph{Objective and Kernel Construction.}
Let $X_s \in \mathbb{R}^{n \times d}$ and $X_t \in \mathbb{R}^{m \times d}$ denote the source and target domain feature matrices, with $n + m$ total samples and $d$-dimensional features. TCA constructs a combined kernel matrix $K \in \mathbb{R}^{(n+m) \times (n+m)}$ using a linear kernel:
\[
K(x_i, x_j) = x_i^\top x_j.
\]
The composite kernel $K$ is partitioned as:
\[
K = 
\begin{bmatrix}
K_{ss} & K_{st} \\
K_{ts} & K_{tt}
\end{bmatrix},
\]
where $K_{ss} = X_s X_s^\top$, $K_{tt} = X_t X_t^\top$, and $K_{st} = X_s X_t^\top$. A projection matrix $W \in \mathbb{R}^{(n+m) \times k}$ is then learned by solving:
\[
\min_W \; \mathrm{tr}(W^\top K L K^\top W) + \mu \cdot \mathrm{tr}(W^\top K H K^\top W),
\]
where $L$ is a domain alignment matrix based on maximum mean discrepancy (MMD), $H$ is a centering matrix, and $\mu > 0$ is a regularization coefficient.

The alignment matrix $L$ is constructed as:
\[
L = 
\begin{bmatrix}
\frac{1}{n^2} \mathbf{1}_{n \times n} & -\frac{1}{nm} \mathbf{1}_{n \times m} \\
-\frac{1}{nm} \mathbf{1}_{m \times n} & \frac{1}{m^2} \mathbf{1}_{m \times m}
\end{bmatrix},
\]
which encourages samples from different domains to align while preserving within-domain relationships. The centering matrix is defined as:
\[
H = I - \frac{1}{n+m} \mathbf{1} \mathbf{1}^\top,
\]
ensuring that the projected features are zero-centered in the kernel space.

\paragraph{Projection and Prediction.}
The optimization problem is solved via eigen-decomposition. Specifically, the generalized eigensystem:
\[
(I + \mu K L K) S = K H K S
\]
is decomposed into eigenvectors $S = U \Lambda U^\top$, and the top-$k$ eigenvectors are used to construct the projection matrix $W = U_{[:,1:k]}$.

Source and target samples are then projected via:
\[
Z_s = K_s W, \quad Z_t = K_t W,
\]
where $K_s$ and $K_t$ are kernel submatrices corresponding to $X_s$ and $X_t$. A logistic regression classifier is trained on $(Z_s, y_s)$ and applied to $Z_t$ for prediction. This subspace alignment significantly improves performance under covariate shift, particularly in cross-institutional settings where the label space remains shared but marginal distributions differ.



\subsection{Evaluation Metrics}

To comprehensively evaluate the effectiveness of the proposed Pre-trained Tabular Foundation Model with TCA-based domain adaptation, we constructed a multidimensional assessment framework covering classification metrics, statistical confidence intervals, visualization-based analysis, and domain discrepancy measures.

\subsubsection*{1. Classification Performance Metrics}

Five widely used metrics were adopted to assess classification accuracy: AUC, accuracy, F1 score, sensitivity, and specificity. All results were averaged over 10-fold stratified cross-validation to ensure robustness against label imbalance. Let $TP$, $TN$, $FP$, and $FN$ denote the number of true positives, true negatives, false positives, and false negatives, respectively. The metrics are defined as:

\[
\begin{aligned}
&\text{True Positive Rate:} && TPR(\tau) = \frac{TP(\tau)}{TP(\tau) + FN(\tau)} \\
&\text{False Positive Rate:} && FPR(\tau) = \frac{FP(\tau)}{FP(\tau) + TN(\tau)} \\
&\text{AUC:} && AUC = \int_0^1 TPR(\tau)\, d(FPR(\tau)) \\
&\text{Accuracy:} && \frac{TP + TN}{TP + TN + FP + FN} \\
&\text{Precision:} && \frac{TP}{TP + FP} \\
&\text{Recall:} && \frac{TP}{TP + FN} \\
&\text{F1 Score:} && \frac{2 \cdot \text{Precision} \cdot \text{Recall}}{\text{Precision} + \text{Recall}} = \frac{2TP}{2TP + FP + FN} \\
&\text{Specificity:} && \frac{TN}{TN + FP}
\end{aligned}
\]

Let $\mathcal{D} = \{(\mathbf{x}_i, y_i)\}_{i=1}^n$ denote the full dataset, and $\mathcal{D}_k$ be the $k$-th fold. For metric $M$, the mean and standard deviation over $K=10$ folds are:

\[
\bar{M} = \frac{1}{K}\sum_{k=1}^K M_k, \quad \sigma_M = \sqrt{\frac{1}{K-1} \sum_{k=1}^K (M_k - \bar{M})^2}
\]

\subsubsection*{2. Confidence Interval Estimation}

To quantify uncertainty in AUC, we employed non-parametric bootstrap resampling ($B=1000$). For each bootstrap iteration $b$, a dataset $\mathcal{D}_b^*$ was sampled with replacement:

\[
\mathcal{D}_b^* = \{(\mathbf{x}_{i_j}^*, y_{i_j}^*)\}_{j=1}^n, \quad i_j \sim \text{Uniform}(1, \dots, n)
\]

The model was retrained on each $\mathcal{D}_b^*$ to compute $\text{AUC}_b^*$, forming an empirical distribution $\{\text{AUC}_1^*, \dots, \text{AUC}_B^*\}$. The 95\% confidence interval was then defined as:

\[
CI_{95\%} = \left[ Q_{2.5\%},\, Q_{97.5\%} \right]
\quad \text{where} \quad Q_\alpha := \text{quantile at } \alpha
\]

\subsubsection*{3. Visualization-Based Evaluation}

\begin{itemize}
    \item \textbf{ROC Curves:} Plot $TPR(\tau)$ versus $FPR(\tau)$ for $\tau \in [0,1]$ to visualize sensitivity-specificity trade-off. An ideal curve approaches the point $(0,1)$, while a random model lies along the diagonal $TPR = FPR$.

    \item \textbf{Calibration Curves:} Assess the agreement between predicted probability $\hat{p}_i$ and observed frequency $y_i$. For $K$ equal-width bins $B_k = [k/K, (k+1)/K)$:

    \[
    \bar{p}_k = \frac{1}{|B_k|} \sum_{i \in B_k} \hat{p}_i, \quad \bar{y}_k = \frac{1}{|B_k|} \sum_{i \in B_k} y_i
    \]

    \item \textbf{Decision Curve Analysis (DCA):} Evaluate net benefit $NB(p_t)$ under clinical cost-benefit assumptions:

    \[
    NB(p_t) = \frac{TP(p_t)}{n} - \frac{FP(p_t)}{n} \cdot \frac{p_t}{1 - p_t}
    \]

    With benchmark strategies:
    \[
    NB_{all}(p_t) = \text{Prevalence} - (1 - \text{Prevalence}) \cdot \frac{p_t}{1 - p_t}, \quad NB_{none} = 0
    \]
    where $\text{Prevalence} = \frac{1}{n} \sum_{i=1}^n y_i$
\end{itemize}

\subsubsection*{4. Domain Adaptation Evaluation}

To assess the effectiveness of TCA in aligning source and target distributions, we used both qualitative and quantitative tools:

\paragraph{(a) Dimensionality Reduction.}
PCA and t-SNE were applied to visualize domain overlap before and after adaptation.

\paragraph{(b) Normalized Domain Distance Metrics.}
Let $\mu(\cdot)$ and $\sigma(\cdot)$ denote feature-wise mean and std. The standardized features are:

\[
\hat{\mu} = \frac{\mu(\mathbf{X}_s) + \mu(\mathbf{X}_t)}{2}, \quad
\hat{\sigma} = \frac{\sigma(\mathbf{X}_s) + \sigma(\mathbf{X}_t)}{2}
\]
\[
\mathbf{X}_s^{\text{norm}} = \frac{\mathbf{X}_s - \hat{\mu}}{\hat{\sigma}}, \quad \mathbf{X}_t^{\text{norm}} = \frac{\mathbf{X}_t - \hat{\mu}}{\hat{\sigma}}
\]

Then, compute the following:

\begin{itemize}
    \item \textbf{Wasserstein Distance:}
    \[
    W_{\text{norm}}(\mathbf{X}_s, \mathbf{X}_t) = \frac{1}{d} \sum_{i=1}^d W_1(X_{s,i}^{\text{norm}}, X_{t,i}^{\text{norm}})
    \]

    \item \textbf{Symmetric KL Divergence:}
    \[
    KL_{\text{norm}}(\mathbf{X}_s, \mathbf{X}_t) = \frac{1}{d} \sum_{i=1}^d \frac{KL(P_{s,i}^{\text{norm}} || P_{t,i}^{\text{norm}}) + KL(P_{t,i}^{\text{norm}} || P_{s,i}^{\text{norm}})}{2}
    \]

    \item \textbf{MMD with RBF Kernel:}
    \[
    \text{MMD}^2(\mathbf{X}_s, \mathbf{X}_t) =
    \frac{1}{n_s(n_s-1)} \sum_{i \neq j} k(x_i^s, x_j^s)
    + \frac{1}{n_t(n_t-1)} \sum_{i \neq j} k(x_i^t, x_j^t)
    - \frac{2}{n_s n_t} \sum_{i,j} k(x_i^s, x_j^t)
    \]
    where $k(\mathbf{x}, \mathbf{y}) = \exp(-\gamma ||\mathbf{x} - \mathbf{y}||^2)$
\end{itemize}

Notably, while TCA uses a linear kernel for domain projection, RBF-kernel MMD offers a nonlinear complementary perspective.

\vspace{1em}
This comprehensive evaluation protocol enables rigorous, multidimensional validation of the proposed domain-adaptive foundation model across both predictive accuracy and cross-domain generalization.



\subsection{Baseline Methods}

To evaluate the effectiveness of the proposed Pre-trained Tabular Foundation Model with TCA-based domain adaptation, we designed a series of comparative experiments involving the following baseline groups:

\paragraph{(1) Foundation Model without Domain Adaptation.}  
This baseline directly applied the Pre-trained Tabular Foundation Model trained on the source domain to the target domain, without any domain alignment. It served as a foundation-only benchmark to isolate the contribution of domain adaptation.

\paragraph{(2) Conventional Clinical Risk Models.}  
We included several widely used rule-based clinical scoring systems for comparison, including the PKUPH model, the Mayo Clinic score, and a previously published logistic regression model (Paper\_LR). These methods reflect existing clinical heuristics based on handcrafted variables and were implemented following their original published formulas. Since these models do not involve data-driven training, their generalization relies solely on the stability of clinical rule transfer.

\paragraph{(3) Classical Machine Learning Algorithms.}  
To assess the generalization of standard supervised learners, we implemented several representative classifiers:
\begin{itemize}
    \item \textbf{Support Vector Machine (SVM)} using an RBF kernel, with hyperparameters tuned via grid search on the source domain.
    \item \textbf{Decision Tree (DT)} and \textbf{Random Forest (RF)} models using Gini impurity for splitting and evaluated under varying maximum depths and tree counts.
    \item \textbf{Gradient Boosted Decision Tree (GBDT)} and \textbf{XGBoost}, optimized with respect to learning rate, number of estimators, maximum tree depth, and subsample ratio. All models were trained on the source domain and tested directly on the target domain.
\end{itemize}

\paragraph{(4) Proposed Domain Adaptation Method: Pre-trained Tabular Foundation Model + TCA.}  
Our proposed method integrates the Pre-trained Tabular Foundation Model with Transfer Component Analysis (TCA) to mitigate domain shift. Specifically, TCA projects both source and target data into a shared latent space, after which predictions are made using the ensemble inference structure of the pre-trained model. This approach enables cross-domain generalization by aligning marginal distributions while retaining the rich semantic representations learned during pretraining.

\paragraph{Hyperparameter Tuning and Evaluation Protocol.}  
For all trainable baselines, hyperparameters were optimized using 10-fold stratified cross-validation within the source domain to avoid data leakage. Final model performance was assessed on the target domain using the same metrics and evaluation procedures as applied to our proposed method.

This comprehensive set of baselines enables a rigorous comparative analysis, revealing the relative contributions of domain alignment and model architecture to cross-institutional performance in medical tabular data.


\subsection{Computational resource}

\subsection{Reporting summary}


\section{Analysis}

\label{sec:analysis}

This chapter moves beyond descriptive reporting of performance metrics to develop a rigorous theoretical account of \textit{why} the PANDA framework succeeds where traditional methods fail. The failure of classical models (e.g., GBDTs, CNNs) in cross-hospital deployment is interpreted not as an engineering artifact, but as a violation of a core assumption in statistical learning theory, namely that training and test data are independently and identically distributed (i.i.d.) samples from the same joint distribution $P(\inputspace, \labelspace)$.

By analyzing PANDA under the generalization bound of Ben-David et al. for domain adaptation, we show that its architecture constitutes a direct algorithmic response to the theoretical decomposition of target error.

\subsection{Theoretical Foundation: The Generalization Bound}
\label{subsec:generalization_bound}

To study the generalization properties of PANDA, we adopt the seminal learning-theoretic framework of Shai Ben-David et al. (2010), which is a cornerstone of domain adaptation theory \cite{ben2010theory}. This framework provides a rigorous upper bound on the target domain error and decomposes it into observable and optimizable components. The bound clarifies why minimizing the source error alone is insufficient and why explicit domain-alignment mechanisms such as TCA are required.

\begin{theorem}[Ben-David et al., 2010]
Let $\hypothesisclass$ be a hypothesis space of VC-dimension $\vcdim$. If $\sourcedata$ and $\targetdata$ are samples of size $n$ drawn from source distribution $\sourcedomaindist$ and target distribution $\targetdomaindist$ respectively, then for any $\confidence \in (0, 1)$, with probability at least $1 - \confidence$, for every $\hypothesis \in \hypothesisclass$:
\begin{equation}
\label{eq:ben_david_bound}
\targeterror \leq \sourceerror + \frac{1}{2} \domaindivergence(\sourcedomaindist, \targetdomaindist) + \adaptabilityterm + \mathcal{O}\left(\sqrt{\frac{\vcdim \log n}{n} + \log \frac{1}{\confidence}}\right)
\end{equation}
\end{theorem}

Inequality~\eqref{eq:ben_david_bound} shows that reducing the target error $\targeterror$ requires the simultaneous control of three terms:
\begin{enumerate}
\item \textbf{Source Risk} $\sourceerror$: The expected error on the source domain.
\item \textbf{Domain Divergence} $\domaindivergence(\sourcedomaindist, \targetdomaindist)$: The $\mathcal{H}\Delta\mathcal{H}$-divergence, which measures the discrepancy between the marginal feature distributions.
\item \textbf{Adaptability Term} $\adaptabilityterm$: The error of the ideal joint hypothesis, $\adaptabilityterm = \min_{\hypothesis \in \hypothesisclass} (\sourceerror + \targeterror)$, which captures irreducible error due to concept shift, that is, mismatch in the conditional distributions $P(Y\mid X)$.
\end{enumerate}

\subsubsection{The $\mathcal{H}\Delta\mathcal{H}$-Divergence}
A central quantity in the theory of Ben-David et al. is the $\mathcal{H}\Delta\mathcal{H}$-divergence, which quantifies domain distance with respect to the hypothesis class $\mathcal{H}$. It is defined as
\begin{equation}
    \label{eq:hdhd-divergence}
    \domaindivergence(\sourcedomaindist, \targetdomaindist) = 2 \sup_{h, h' \in \mathcal{H}} \left| \Pr_{x \sim \sourcedomaindist}[h(x) \neq h'(x)] - \Pr_{x \sim \targetdomaindist}[h(x) \neq h'(x)] \right|.
\end{equation}
Intuitively, this metric measures the maximal discrepancy in classifier disagreement across the two domains. If there exist two classifiers $h, h'$ that exhibit low disagreement on the source domain but high disagreement on the target domain, the divergence is large. In this case, the source domain does not sufficiently constrain classifier behavior on the target domain, which can lead to negative transfer. Alignment methods such as TCA and CORAL aim to transform the feature space so that this divergence is reduced.

\subsection{Minimizing Source Risk $\sourceerror$: The TabPFN Mechanism}
\label{subsec:analysis_source_risk}

The first challenge is the small sample size ($\sourcedatasize \approx 295$). In this regime, standard Empirical Risk Minimization (ERM) is prone to high variance. Deep neural networks typically require $n > 10^4$ to generalize reliably, while GBDTs often overfit, effectively memorizing noise in $\sourcedata$ rather than learning the structure of $\sourcedomaindist$.

\subsubsection{Prior-Data Fitted Networks vs. Parametric Learning}
Traditional parametric learning optimizes weights $\modelparams$ to minimize loss on $\sourcedata$,
\begin{equation}
    \label{eq:erm}
    \hat{\modelparams} = \arg\min_{\modelparams} \sum_{({\featurevec},{\labelval}) \in \sourcedata} \loss(f_{\modelparams}({\featurevec}), {\labelval}),
\end{equation}
starting from largely uninformative priors and therefore requiring substantial data.

In contrast, PANDA uses TabPFN, a \textbf{Prior-Data Fitted Network (PFN)}. It reduces $\sourceerror$ not by performing gradient descent on $\sourcedata$, but by approximating the \textbf{Posterior Predictive Distribution (PPD)} using a Transformer pre-trained on millions of synthetic causal models:
\begin{equation}
    \label{eq:ppd}
    P({\labelval}_{\text{query}} \mid \queryvec, \sourcedata) \approx \int P({\labelval} \mid \featurevec, M) P(M \mid \sourcedata) , dM.
\end{equation}
TabPFN treats the source dataset $\sourcedata$ as context for Bayesian inference rather than as a training set for parameter optimization, and thus acts as a strong regularizer that imposes inductive biases favoring sparsity and piecewise smoothness.

\textbf{Empirical Consequence:} As shown in our results, TabPFN attains a source AUC of \textbf{0.829}, substantially higher than Random Forest (0.752) and XGBoost (0.742). This establishes a strictly lower starting point for the $\sourceerror$ term in the bound.

\subsection{Minimizing Divergence $\domaindivergence$: Latent Space TCA}
\label{subsec:analysis_divergence}

The second term, $\domaindivergence$, measures the discrepancy between domains. In our setting, scanner heterogeneity induces \textbf{covariate shift}, so that $\marginalsourcedist(\featurevec) \neq \marginaltargetdist(\featurevec)$.

\subsubsection{Why Feature Space Alignment?}
Aligning raw features is often suboptimal because medical variables exhibit complex, nonlinear dependencies. PANDA therefore applies TCA directly to $\rfeselectedfeatures$. After RFE, the retained features form a more robust and relevant subset, which makes them more amenable to linear alignment. The goal of this step is to align the marginal distributions of these preselected features between the source and target domains.

\subsubsection{The TCA Optimization Objective}
We employ Transfer Component Analysis (TCA) to find a projection $\tcaprojectionmatrix \in \mathbb{R}^{\rfecurrentdim \times \tcaprojectdim}$ that minimizes the Maximum Mean Discrepancy (MMD) between source and target features (after RFE). The objective is
\begin{equation}
    \label{eq:tca-objective}
    \min_{\tcaprojectionmatrix} \tr({\tcaprojectionmatrix}^\top \kernelmatrix \mmdmatrix \kernelmatrix \tcaprojectionmatrix) + \regularizationparam\, \tr({\tcaprojectionmatrix}^\top \tcaprojectionmatrix) \quad \text{s.t.} \quad {\tcaprojectionmatrix}^\top \kernelmatrix \centeringmatrix \kernelmatrix \tcaprojectionmatrix = \identitymatrix,
\end{equation}
where $\kernelmatrix$ is the kernel matrix of the RFE-selected features ($K_{ij} = \langle \featurevec_i, \featurevec_j \rangle$), $\mmdmatrix$ is the MMD indicator matrix, and $\centeringmatrix$ is the centering matrix. The constraint ${\tcaprojectionmatrix}^\top \kernelmatrix \centeringmatrix \kernelmatrix \tcaprojectionmatrix = \identitymatrix$ preserves data variance, ensuring that the projection does not collapse informative signal while aligning means. Here, $\rfecurrentdim$ denotes the dimensionality of the RFE-selected features.

\subsection{Minimizing Adaptability Error $\adaptabilityterm$: Cross-Domain RFE}
\label{subsec:analysis_adaptability}

The third term, $\adaptabilityterm$, represents the irreducible error of the best joint hypothesis. A large $\adaptabilityterm$ indicates \textbf{concept shift} ($\conditionalps({\labelval}\mid\featurevec) \neq \conditionalpt({\labelval}\mid\featurevec)$), where the same feature value corresponds to different risks across hospitals (for example, ``spiculation'' due to cancer in Hospital A versus tuberculosis in Hospital B).

PANDA reduces $\adaptabilityterm$ through \textbf{Cross-Domain Recursive Feature Elimination (RFE)}. By intersecting feature-importance rankings from the source with availability and stability constraints, it effectively restricts the hypothesis class $\hypothesisclass$ to a subspace $\rfeselectedfeatures$ in which the conditional distributions are approximately invariant:
\begin{equation}
    \label{eq:concept-invariance}
    \conditionalps({\labelval} \mid \featurevec_{\rfeselectedfeatures}) \approx \conditionalpt({\labelval} \mid \featurevec_{\rfeselectedfeatures}).
\end{equation}
Discarding unstable features (such as subjective morphological scores) can slightly increase the intrinsic source error $\sourceerror$ (bias), but it substantially reduces $\adaptabilityterm$, yielding a tighter overall bound on the target error.

\subsection{Synthesis: Linking Theory to Empirical Results}
\label{subsec:analysis_synthesis}

Table~\ref{tab:theory_empirics} summarizes the connection between the theoretical components and our experimental findings, and illustrates how each PANDA component targets a specific term in the Ben-David bound.

\begin{table}[htbp]
\centering
\caption{Mapping theoretical error terms to PANDA components and empirical results.}
\label{tab:theory_empirics}
\begin{tabular}{llll}
\toprule
\textbf{Error Term} & \textbf{Statistical Challenge} & \textbf{PANDA Component} & \textbf{Empirical Impact} \\
\midrule
$\sourceerror$ & Small sample size & TabPFN backbone & Source AUC: $0.829$ vs $0.742$ (XGBoost) \\
& Overfitting & (Meta-learned priors) & (High sample efficiency) \\
\midrule
$\domaindivergence$ & Covariate shift & TCA on RFE-selected features & Target recall: $0.944$ vs $0.888$ (No-TCA) \\
& (Scanner variation) & (MMD minimization) & (Boundary correction) \\
\midrule
$\adaptabilityterm$ & Concept shift & Cross-domain RFE & Target AUC gap: $<0.01$ (TableShift) \\
& (TB vs cancer) & (Stability filtering) & (Robustness to population shift) \\
\bottomrule
\end{tabular}
\end{table}

\subsection{Summary of Theoretical Insights}
The PANDA framework is not an ad hoc ensemble of heuristics, but a theoretically grounded response to the domain adaptation bound. TabPFN controls the source risk $\sourceerror$ through informative priors; TCA reduces divergence $\domaindivergence$ via spectral alignment; and RFE decreases the adaptability error $\adaptabilityterm$ through stability-based feature pruning. Together, these components support reliable generalization in the challenging regime of small, heterogeneous medical datasets.


\section{Evaluation}
\label{sec:eval-start}
\label{sec:evaluation}

We evaluate PANDA from an artificial intelligence and deployment perspective,
focusing on its ability to (i) learn from small, heterogeneous tabular cohorts,
(ii) remain robust under cross-hospital and cross-population distribution shifts, and
(iii) provide interpretable, clinically useful predictions.
To this end, we adopt a unified evaluation protocol across two complementary
experiments: \textbf{Experiment 1 (E1)}, which targets cross-hospital pulmonary
nodule malignancy prediction, and \textbf{Experiment 2 (E2)}, which targets
race-driven cross-population shift in the TableShift BRFSS Diabetes benchmark.

\subsection{Evaluation Protocols for Cross-Domain Diagnostic Models}

We adopt a common set of discrimination, calibration, and clinical-utility metrics
to enable consistent comparison across hospitals and benchmarks.
Unless otherwise specified, all reported metrics are computed over 10-fold
stratified cross-validation on the source domain and once on the external or
out-of-distribution (OOD) domain.

\subsubsection{Classification Performance Metrics}

We report all results as averages over 10-fold stratified cross-validation to mitigate label imbalance. The metrics are defined as follows:

\begin{equation}
    \label{eq:tpr}
    \text{True Positive Rate:} \quad \tpr(\tau) = \frac{\tp(\tau)}{\tp(\tau) + \fn(\tau)}
\end{equation}

\begin{equation}
    \label{eq:fpr}
    \text{False Positive Rate:} \quad \fpr(\tau) = \frac{\fp(\tau)}{\fp(\tau) + \tn(\tau)}
\end{equation}

\begin{equation}
    \label{eq:auc}
    \text{AUC:} \quad \auc = \int_0^1 \tpr(\tau)\, d(\fpr(\tau))
\end{equation}

\begin{equation}
    \label{eq:accuracy}
    \text{Accuracy:} \quad \frac{\tp + \tn}{\tp + \tn + \fp + \fn}
\end{equation}

\begin{equation}
    \label{eq:precision}
    \text{Precision:} \quad \frac{\tp}{\tp + \fp}
\end{equation}

\begin{equation}
    \label{eq:recall}
    \text{Recall (Sensitivity):} \quad \frac{\tp}{\tp + \fn}
\end{equation}

\begin{equation}
    \label{eq:f1}
    \text{F1 Score:} \quad \frac{2 \cdot \text{Precision} \cdot \text{Recall}}{\text{Precision} + \text{Recall}}
\end{equation}

\begin{equation}
    \label{eq:specificity}
    \text{Specificity:} \quad \frac{\tn}{\tn + \fp}
\end{equation}

Let $\mathcal{D} = \{(\featurevec_i, \labelval_i)\}_{i=1}^n$ denote the full dataset, and let $\mathcal{D}_k$ be the $k$-th fold. For a metric $M$, the mean and standard deviation over $\numfolds=10$ folds are

\begin{equation}
    \label{eq:cross-validation-stats}
    \meanmetric = \frac{1}{\numfolds}\sum_{k=1}^{\numfolds} \metrick, \quad \stdmetric = \sqrt{\frac{1}{\numfolds-1} \sum_{k=1}^{\numfolds} (\metrick - \meanmetric)^2}.
\end{equation}

\subsubsection{Visualization-Based Evaluation}

\begin{itemize}
\item \textbf{ROC Curves:} We plot $\tpr(\tau)$ versus $\fpr(\tau)$ for $\tau \in [0,1]$ to characterize the trade-off between sensitivity and specificity across operating thresholds.

\item \textbf{Calibration Curves:} We assess the agreement between the predicted probability $\predprob$ and the observed frequency $\labelval_i$. For $\numfolds$ equal-width bins $\calibbin = [k/\numfolds, (k+1)/\numfolds)$, we compute
\begin{equation}
    \label{eq:calibration-bins}
    \meanpredprob = \frac{1}{|\calibbin|} \sum_{i \in \calibbin} \predprob, \quad \meanobservedfreq = \frac{1}{|\calibbin|} \sum_{i \in \calibbin} \labelval_i.
\end{equation}
These quantities summarize both over- and underestimation of risk across the probability range.

\item \textbf{Decision Curve Analysis (DCA):} For a given probability threshold $\probthreshold$, the net benefit is
\begin{equation}
    \label{eq:net-benefit}
    \netbenefit = \frac{\tp(\probthreshold)}{n} - \frac{\fp(\probthreshold)}{n} \cdot \frac{\probthreshold}{1 - \probthreshold},
\end{equation}
with benchmark strategies
\begin{equation}
    \label{eq:benchmark-strategies}
    \nball(\probthreshold) = \text{Prevalence} - (1 - \text{Prevalence}) \cdot \frac{\probthreshold}{1 - \probthreshold}, \quad \nbnone = 0.
\end{equation}
DCA therefore quantifies the clinical utility of a model across plausible decision thresholds relative to treating all or no patients.
\end{itemize}

\subsection{Experiment 1 (E1): Cross-Hospital Pulmonary Nodule Malignancy Prediction}

% --- 这一小段可作为"数据/设置/挑战"描述 ---
Experiment 1 evaluates PANDA in a realistic cross-hospital deployment scenario
for pulmonary nodule malignancy prediction.
Structured clinical data from two cancer centers in China provide a labeled
source cohort (Cohort A, $\sourcedatasize=295$) and an external target cohort
(Cohort B, $\targetdatasize=190$).
Cohort A contains 63 structured features and Cohort B contains 58, reflecting
differences in local documentation and biomarker panels
(Table~\ref{tab:cohort_summary}).
Both cohorts exhibit moderate class imbalance, with malignant nodules comprising
64.1\% of patients in Cohort A and 65.8\% in Cohort B, and key covariates such as
upper-lobe location, age, pulmonary function indices (DLCO, VC), and serum CEA
show distributional differences between the two hospitals.
These discrepancies capture the covariate shift that motivates domain adaptation
in this setting.

\paragraph{Prevalence and label shift.}
Real-world LDCT screening cohorts typically exhibit malignancy prevalences of approximately 5\%. In contrast, both of our internal pulmonary nodule cohorts are enriched case-control datasets with a higher malignancy proportion (around 65\%), constructed to ensure sufficient malignant samples for model development and cross-hospital comparison. Consequently, these cohorts do not constitute a canonical low-prevalence label-shift scenario. Instead, our pulmonary experiment focuses on covariate shift (e.g., shifts in age, smoking history, and acquisition protocols) and concept shift arising from differences in local labeling and management practices across hospitals.

\begin{table}[htbp]
\centering
\caption{Training (Cohort A) and testing (Cohort B) cohorts.}
\label{tab:cohort_summary}
\begin{tabular}{lcc}
\toprule
\textbf{Characteristic} & \textbf{Cohort A (n = 295)} & \textbf{Cohort B (n = 190)} \\
\midrule
Upper lobe & & \\
\quad Yes/Positive & 121 (41.0\%) & 98 (51.6\%) \\
\quad No/Negative & 174 (59.0\%) & 92 (48.4\%) \\
Age (years) & 56.95 $\pm$ 11.03 & 58.26 $\pm$ 9.57 \\
Lobe location (upper) & & \\
\quad Category 1 & 161 (54.6\%) & 98 (51.6\%) \\
\quad Category 2 & 29 (9.8\%) & 18 (9.5\%) \\
\quad Category 3 & 105 (35.6\%) & 74 (38.9\%) \\
DLCO1 & 5.90 $\pm$ 2.89 & 6.31 $\pm$ 1.55 \\
VC & 3.33 $\pm$ 0.80 & 2.92 $\pm$ 0.73 \\
CEA & 4.23 $\pm$ 6.90 & 4.15 $\pm$ 10.61 \\
Outcome (Malignant) & & \\
\quad Yes/Positive & 189 (64.1\%) & 125 (65.8\%) \\
\quad No/Negative & 106 (35.9\%) & 65 (34.2\%) \\
\bottomrule
\end{tabular}
\end{table}

% --- 把原来的第二个 subection 降一级 ---
\subsubsection{Internal and Cross-Hospital Generalization}

\subsubsection{Source and Target Domain Performance}

Figure~\ref{fig:performance-heatmaps} summarizes relative performance trends
across methods, and Table~\ref{tab:main_results_table} reports the corresponding
numerical metrics.
On the source domain (10-fold cross-validation on Cohort A), PANDA (TabPFN) achieves the
highest \auc of 0.8287, demonstrating the superior performance of the tabular
foundation model on small, heterogeneous medical datasets.
LASSO LR and Random Forest follow closely, with \auc values of 0.7631 and 0.7515,
respectively.
XGBoost and GBDT show competitive performance (\auc 0.7416 and 0.7212), while
traditional clinical scores (Mayo, PKUPH) and classical machine learning baselines
exhibit lower discrimination.
All source-domain experimental results are reproducible using the provided
codebase and configuration scripts.

On the external target domain (Cohort B), the benefits of unsupervised domain
adaptation become evident.
The TCA-enhanced PANDA model attains the highest \auc\ of 0.7046 and an
exceptionally high \recall\ of 0.9440, indicating strong sensitivity for clinical
screening applications.
PANDA\_NoUDA follows closely with \auc of 0.6980, demonstrating the inherent
transferability of the tabular foundation model.
In contrast, traditional machine learning methods experience substantial
performance drops under domain shift, with Random Forest declining to \auc 0.6324
and tree ensembles dropping below 0.600.
The clinical scores continue to show poor transportability across institutions.

\begin{table}[htbp]
\centering
\caption{Comprehensive performance comparison. Source results are from 10-fold cross-validation; target results are from external validation on Cohort B. Best values are bolded.}
\label{tab:main_results_table}
\begin{tabular}{lccccc}
\toprule
\textbf{Model} & \textbf{\auc} & \textbf{\accuracy} & \textbf{\fonescore} & \textbf{\precision} & \textbf{\recall} \\
\midrule
\multicolumn{6}{l}{\textit{Source Domain (Internal CV)}} \\
\textbf{PANDA (TabPFN)} & \textbf{0.8287} & 0.7460 & \textbf{0.8102} & 0.7864 & 0.8462 \\
LASSO LR & 0.7631 & 0.7224 & 0.8101 & 0.7227 & \textbf{0.9254} \\
Random Forest & 0.7515 & 0.6983 & 0.7792 & 0.7351 & 0.8415 \\
XGBoost & 0.7416 & 0.6778 & 0.7520 & 0.7325 & 0.7871 \\
GBDT & 0.7212 & 0.6911 & 0.7690 & 0.7394 & 0.8140 \\
SVM & 0.7175 & 0.6645 & 0.7197 & \textbf{0.7794} & 0.6769 \\
PKUPH & 0.6640 & \textbf{0.6782} & 0.7672 & 0.7148 & 0.8354 \\
Mayo & 0.6049 & 0.3592 & 0.0000 & 0.0000 & 0.0000 \\
Decision Tree & 0.5764 & 0.6099 & 0.6929 & 0.6997 & 0.6974 \\
\midrule
\multicolumn{6}{l}{\textit{Target Domain (External Validation)}} \\
\textbf{TCA} & \textbf{0.7046} & \textbf{0.7053} & \textbf{0.8082} & 0.7066 & \textbf{0.9440} \\
PANDA\_NoUDA & 0.6980 & 0.6632 & 0.7762 & 0.6894 & 0.8880 \\
Random Forest & 0.6324 & 0.6789 & 0.7753 & 0.7128 & 0.8538 \\
PKUPH & 0.6356 & 0.6947 & 0.7838 & \textbf{0.7329} & 0.8474 \\
LASSO LR & 0.6678 & 0.6737 & 0.7911 & 0.6825 & 0.9429 \\
SVM & 0.6285 & 0.5684 & 0.6468 & 0.6950 & 0.6064 \\
GBDT & 0.5906 & 0.5842 & 0.6834 & 0.6689 & 0.7109 \\
XGBoost & 0.5672 & 0.5947 & 0.6937 & 0.6701 & 0.7244 \\
Decision Tree & 0.5090 & 0.5684 & 0.6759 & 0.6650 & 0.6942 \\
Mayo & 0.5837 & 0.3421 & 0.0000 & 0.0000 & 0.0000 \\
\bottomrule
\end{tabular}
\end{table}

\begin{figure}[htbp]
\centering
\includegraphics[width=\linewidth]{img/cross_hospital/combined_heatmaps_nature.pdf}
\caption{\textbf{Performance comparison across domains.} \textbf{a} Source-domain 10-fold cross-validation heatmap over five metrics, showing PANDA’s leading performance in AUC, accuracy, and precision. \textbf{b} Cross-domain external validation heatmap; the TCA-enhanced PANDA model maintains the highest AUC and recall, highlighting its stability under domain shift.}
\label{fig:performance-heatmaps}
\end{figure}

\subsubsection{Stratified Analysis}

To examine potential biases and subgroup robustness, we evaluated PANDA's performance across key subgroups (Table~\ref{tab:stratified_analysis}).

\begin{itemize}
\item \textbf{Nodule Size}: Performance remains strong for large nodules ($>8$ mm, \auc 0.74) but declines for sub-centimeter nodules (\auc 0.65), reflecting the inherent difficulty of radiological characterization for small lesions.
\item \textbf{Smoking Status}: The model performs better in smokers (\auc 0.72) than in non-smokers (\auc 0.68), likely because smoking provides a strong prior for malignancy that the model can exploit.
\item \textbf{Gender}: We observe comparable performance across gender (\auc 0.70 vs 0.71), suggesting no substantial gender-specific bias at the current sample size.
\end{itemize}

\begin{table}[htbp]
\centering
\caption{Stratified performance of PANDA+TCA on the target cohort.}
\label{tab:stratified_analysis}
\begin{tabular}{lccc}
\toprule
\textbf{Subgroup} & \textbf{n} & \textbf{\auc} & \textbf{Sensitivity} \\
\midrule
\textbf{Nodule Size} & & & \\
$\le 8$ mm & 72 & 0.65 & 0.88 \\
$> 8$ mm & 118 & 0.74 & 0.96 \\
\midrule
\textbf{Smoking History} & & & \\
Never Smoker & 105 & 0.68 & 0.92 \\
Current/Former & 85 & 0.72 & 0.97 \\
\midrule
\textbf{Gender} & & & \\
Male & 110 & 0.71 & 0.95 \\
Female & 80 & 0.70 & 0.93 \\
\bottomrule
\end{tabular}
\end{table}

\subsection{Experiment 2 (E2): Cross-Population Validation on TableShift}

We further evaluated PANDA on the TableShift BRFSS Diabetes benchmark, which introduces a race-driven shift by training on one demographic group and evaluating on another (White $\rightarrow$ Non-White). This benchmark provides a complementary stress test to the cross-hospital shift in the pulmonary nodule experiment and mimics realistic deployment across heterogeneous populations in population-health applications.

In this setting, our goal is to assess whether a tabular foundation model with lightweight domain alignment can remain stable when the covariate distribution changes along demographic axes. Qualitatively, PANDA (with and without TCA) achieves discrimination and calibration that are competitive with well-tuned tree ensembles, while avoiding the severe performance collapse that would be concerning in a cross-population deployment. This behavior supports the view that the inductive biases learned by the foundation model transfer beyond the pulmonary nodule task and remain useful under a distinct form of distribution shift.

\begin{table}[htbp]
\small
\centering
\caption{Source (ID) vs target (OOD) cohorts for the BRFSS Diabetes race shift.}
\label{tab:tableshift_cohort_summary}
\begin{tabular}{p{0.34\linewidth}p{0.28\linewidth}p{0.28\linewidth}}
\toprule
\textbf{Characteristic} & \textbf{Source / ID (PRACE1 = 1)} & \textbf{Target / OOD (PRACE1 $\in \{2,\dots,6\}$)} \\
\midrule
Sample size (full) & Train 969{,}229; Val 121{,}154; Test 121{,}154 & Val 23{,}264; Test 209{,}375 \\
Diabetes positive rate & 12.5\% (train) & 17.4\% (ood\_test) \\
Years & 2015: 245{,}675; 2016: 5{,}789; 2017: 244{,}996; 2018: 6{,}403; 2019: 221{,}847; 2020: 9{,}630; 2021: 223{,}088; 2022: 11{,}801 & 2015: 49{,}216; 2016: 1{,}507; 2017: 52{,}150; 2018: 1{,}424; 2019: 48{,}012; 2020: 3{,}147; 2021: 50{,}595; 2022: 3{,}324 \\
Domain shift variable & PRACE1 = 1 (non-Hispanic White) & PRACE1 $\in \{2,3,4,5,6\}$ (other races) \\
Modeling sample (seed 42) & 1{,}024 sampled for training & 2{,}048 sampled for evaluation \\
Sampled diabetes rate & 13.2\% & 17.3\% \\
Sampled pos / neg counts & 135 / 889 & 355 / 1{,}693 \\
Sampled years (seed 42) & 2015: 245; 2016: 8; 2017: 278; 2018: 2; 2019: 232; 2020: 7; 2021: 241; 2022: 11 & 2015: 497; 2016: 17; 2017: 488; 2018: 17; 2019: 445; 2020: 30; 2021: 525; 2022: 29 \\
\bottomrule
\end{tabular}
\end{table}

The BRFSS \texttt{DIABETES} label is coded 1 (Yes) versus 0 (No/Prediabetes/Borderline), with NA rows removed. After preprocessing, all 142 features are numerical, cross-year aligned, and retain the survey year (\texttt{IYEAR}). Preprocessing steps drop \texttt{DRNK\_PER\_WEEK=99900}, map \texttt{SEX$\rightarrow\{0,1\}$}, set health-day 88 to 0, and fill remaining non-asked entries with \texttt{NOTASKED\_MISSING}.

Figure~\ref{fig:tableshift-heatmaps} summarizes the behavior of different models across the training and race-shifted evaluation splits, and Figure~\ref{fig:brfss-roc} reports ROC, calibration, and decision curves. Together, these visualizations indicate that PANDA’s score distributions and decision boundaries can be adapted to new demographic groups without sacrificing overall screening utility.

\begin{figure}[htbp]
\centering
\includegraphics[width=0.95\textwidth]{img/tableshift/combined_heatmaps_nature.pdf}
\caption{\textbf{Performance comparison on the TableShift BRFSS Diabetes benchmark.} Heatmaps summarize multiple metrics for PANDA and baseline models across the training split and the race-shifted evaluation split. PANDA with TCA remains competitive with strong tree ensembles and does not exhibit pronounced degradation under race shift.}
\label{fig:tableshift-heatmaps}
\end{figure}

\textbf{Discussion on Precision/Recall}: Readers may observe low $\fonescore$ values in BRFSS despite high $\accuracy$ (Fig.~\ref{fig:brfss-roc}). This pattern arises from the low positive-class prevalence $\pi = P(Y=1)$ (17.4\%) and the default 0.5 threshold: the model correctly identifies most negatives (high $\accuracy$) but, without class re-weighting, yields moderate $\precision$ on the minority positive class. For screening purposes, the ROC curves indicate adequate discriminative ability; the operating point can be adjusted via threshold tuning to prioritize $\recall$ when desired.

\begin{figure}[htbp]
\centering
\includegraphics[width=0.95\textwidth]{img/tableshift/combined_analysis_figure.pdf}
\caption{\textbf{TableShift BRFSS Diabetes analysis.} \textbf{a,b} ROC curves showing PANDA's robustness under race-driven shift. \textbf{c,d} Calibration curves assessing probability estimates. \textbf{e,f} Decision curves illustrating net benefit across clinical thresholds.}
\label{fig:brfss-roc}
\end{figure}

\subsection{Interpretability and Stability}

Figure~\ref{fig:combined_analysis} compares ROC curves, calibration, and decision curves for PANDA and baseline models on the source and target hospitals. It shows that TCA-based alignment preserves screening-level sensitivity while improving net clinical benefit relative to traditional scores and tree ensembles. The same figure also demonstrates better calibration and higher net clinical benefit in decision-curve analysis, particularly in the clinically relevant risk range.

From an interpretability perspective, recursive feature elimination (RFE) identified a stable subset of eight features (Age, Spiculation, etc.) that maximized the cost-effectiveness index (Fig.~\ref{fig:rfe-performance}). This \texttt{best8} set performed within 1\% of the full 63-feature set while providing substantially greater cross-center stability. In source-domain evaluation, the RFE curves (Fig.~\ref{fig:rfe-performance}) track AUC, accuracy, and F1 as functions of the retained subset size, together with stability and cost-effectiveness metrics. Performance plateaus around 9--13 features, consistent with the subset used in the final experiments.

\begin{figure}[htbp]
\centering
\includegraphics[width=1\linewidth]{img/cross_hospital/feature_performance_comparison_comprehensive.pdf}
\caption{\textbf{Comprehensive feature selection and performance analysis using recursive feature elimination (RFE).} (a) AUC, accuracy, and F1 curves as functions of the number of selected features; performance plateaus around 9--13 features, aligning with the preference for simpler models. Shaded regions show variance across 10-fold cross-validation. (b) Class-specific accuracy for malignant and benign cases across subset sizes, illustrating how predictive balance shifts as features are removed. (c) Training-time analysis (seconds per fold) as a function of feature dimensionality, highlighting the computational gain from smaller subsets. (d) Stability assessment using the coefficient of variation across folds; lower values indicate steadier performance. (e) Cost-effectiveness index combining multiple criteria (Performance$\times$0.45 + Simplicity$\times$0.25 + Efficiency$\times$0.15 + Stability$\times$0.15) to identify a feature count that balances accuracy with practical deployment considerations.}
\label{fig:rfe-performance}
\end{figure}

\begin{figure}[htbp]
\centering
\includegraphics[width=1\linewidth]{img/cross_hospital/combined_analysis_figure.pdf}
\caption{\textbf{Performance and utility across source and target domains.} \textbf{a,b} ROC curves comparing source (Cohort A) and external (Cohort B) behavior. \textbf{c,d} Calibration plots for the same splits. \textbf{e,f} Decision curves illustrating the net clinical benefit of PANDA and its TCA extension.}
\label{fig:combined_analysis}
\end{figure}

\label{sec:eval-end}


\section{Conclusion}
\label{sec:concl-start}

\subsection{Summary of Contributions}

This dissertation introduces PANDA, a framework that connects pre-trained tabular foundation models to the practical constraints of medical tabular data, including small sample sizes, heterogeneous feature schemas, and distribution shifts. Experiments on private cross-hospital cohorts and public benchmarks show that:

\begin{enumerate}
\item \textbf{Foundation Models as Robust Priors}: The pre-trained TabPFN backbone substantially outperforms traditional baselines (Random Forest, XGBoost) on small datasets ($n < 300$) by exploiting priors learned from millions of synthetic tasks. This in-context learning capability provides a strong initialization that is more resistant to overfitting than standard empirical risk minimization.

\item \textbf{Stability via Selection}: The Cross-Domain Recursive Feature Elimination (RFE) protocol is essential for removing site-specific artifacts. By converging on a compact set of eight stable predictors, it reduces the dimensionality of the adaptation problem and enables linear alignment methods to succeed where non-linear approaches fail.

\item \textbf{Latent Space Alignment}: Transfer Component Analysis (TCA) applied in the Transformer embedding space effectively reduces the Maximum Mean Discrepancy (MMD) between domains. This alignment yields a consistent performance gain (AUC +0.007) and, more importantly, improves calibration in the target domain.
\end{enumerate}

\subsection{Limitations}

Although PANDA advances the state of the art, several limitations remain.

\subsubsection{Closed-World Assumption}

PANDA assumes that the source and target domains share a common feature schema given by their intersection. It does not address open-world shifts in which the target domain introduces entirely new, highly predictive features that are absent in the source. For example, if a new hospital collects a molecular biomarker (such as DNA methylation) that is not available in the training cohort, PANDA cannot exploit this information without retraining. This lowest-common-denominator strategy in feature selection promotes stability but may limit performance relative to models trained on richer, site-specific schemas.

\subsubsection{Missing Data Mechanisms}

The current implementation of PANDA uses a complete-case strategy: it removes samples with missing values before downstream preprocessing and modeling. Consequently, the core framework does not rely on explicit imputation. This approach implicitly assumes that the retained cohort remains representative under Missing Completely At Random (MCAR) or, more weakly, Missing At Random (MAR). The \texttt{best8} feature set was selected partly for its high completeness, which reduces the number of excluded cases.

However, clinical data are often Missing Not At Random (MNAR); for instance, a clinician may not order a test because they believe the patient is either too healthy or too sick. Under such informative missingness, complete-case filtering may introduce selection bias and limit the generalizability of the current results. Future extensions of PANDA could incorporate missingness indicators or explicitly model informative missingness to better address MNAR settings.

\subsubsection{Computational Resource Requirements}

In contrast to decision trees, which can run on embedded central processing units, TabPFN requires a graphical processing unit for efficient inference (approximately 20 ms per patient). While this cost is negligible for a cloud-based service, it creates a barrier to deployment on edge devices, such as older hospital workstations without dedicated hardware acceleration. The $O(N^2)$ complexity of the Transformer attention mechanism also limits the context size and necessitates subsampling strategies for larger datasets such as BRFSS.

\subsubsection{Scope of Distribution Shift in the Pulmonary Experiment}

While PANDA is in principle applicable under label shift, our cross-hospital pulmonary nodule experiment is not a canonical low-prevalence screening setting. The internal cohorts are enriched case-control datasets with higher malignancy rates (around 65\%), and the dominant sources of shift are covariate and concept differences between hospitals. Label-shift robustness is instead assessed on the TableShift BRFSS Diabetes benchmark. Future work could incorporate additional screening cohorts with realistic prevalences to further stress-test PANDA under extreme label imbalance.

\subsection{Future Directions}

\subsubsection{Federated Domain Adaptation}

Privacy regulations (such as GDPR and HIPAA) often prevent the centralization of medical data. A natural extension of PANDA is federated domain adaptation, in which the feature extractor (TabPFN) remains frozen and shared, while the alignment matrix (TCA) is learned through secure multi-party computation. Because TCA only requires second-order statistics (covariance matrices), participating hospitals can aggregate these sufficient statistics without sharing patient-level records.

\subsubsection{Multimodal Integration}

Pulmonary nodule diagnosis inherently combines imaging (computed tomography scans) with clinical data. Future work should investigate a multimodal variant of PANDA that aligns tabular embeddings from TabPFN with visual embeddings from a convolutional neural network or Vision Transformer. A cross-attention mechanism could then weight the relative contributions of clinical history and radiological appearance in a domain-aware manner, placing greater emphasis on the modality that remains stable when the other is affected by artifacts.

\subsection{Final Remarks}

The deployment gap in medical AI seldom arises from a lack of sophisticated architectures; it more often results from a failure to accommodate the noisy and shifted nature of real-world data. PANDA provides a pragmatic blueprint for this challenge: \textit{do not learn everything from scratch, select only what is stable, and align what remains}. By treating pre-trained representations as portable priors and statistical alignment as a safety mechanism, this framework moves the field closer to reliable cross-institutional AI systems that can safely scale beyond their initial training sites.

\label{sec:concl-end}


\section*{Acknowledgements}

I thank the clinical teams at the participating hospitals for sharing de-identified data and domain expertise, my advisor Wenqi Fan for steady guidance, Bobo for patient, practical advice. Any remaining mistakes are mine.


% References section

\bibliographystyle{unsrt}       % ✅ Use numerical references only --- conforms to:
                                % ``Please use numerical references only for citations''

\bibliography{refs}             % ⚠️ For writing convenience only.
                                % Before submission, you must:
                                % 1. Run BibTeX to generate a `.bbl` file.
                                % 2. Copy the contents of the `.bbl` file into this .tex file.
                                % 3. Delete both `\bibliography{}` and `\bibliographystyle{}`.
                                % → As per: ``If you wish to use BibTeX, please copy the reference list from the .bbl file...''


\end{document}
