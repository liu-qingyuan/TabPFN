\section{Introduction}

Early and accurate prediction of pulmonary nodule malignancy still shapes lung cancer outcomes, yet many decision support tools struggle in everyday clinical use. Nodules discovered on routine CT scans create tricky triage decisions, with malignancy estimates ranging anywhere from 5\% to 70\% depending on setting and patient mix~\cite{swensen1997chest}. Classic risk scores like the Mayo Clinic and Veterans Affairs models remain helpful, though their performance often drops once they leave the cohorts they were built on~\cite{swensen1997chest,gould2007clinical,cui_comparison_2019}. The gap is familiar by now: we need methods that can travel between hospitals while still meeting the sensitivity expectations of screening workflows.

Recent tabular foundation models—most notably TabPFN—have changed small-sample learning by relying on large-scale pre-training to perform well with limited data~\cite{hollmann2025accurate}. Yet these models implicitly assume that feature distributions remain aligned across sites, which makes them surprisingly fragile when real cross-hospital shifts come into play~\cite{zech_variable_2018}. Domain adaptation techniques that succeed in imaging remain thin for structured clinical data~\cite{guan2021domain}, so algorithmic progress has not translated into predictable bedside gains.

Three knots entangle cross-hospital deployment. First, datasets are small; many hospitals share only a few hundred patients with complete records, too little to train deep models from scratch~\cite{borisov2022deep}. Second, domain shift magnifies the issue because patient mixes, workflows, and collection protocols differ enough to drag external AUC down by 20--30\%~\cite{guo_evaluation_2022}. Third, feature heterogeneity shows up when sites record variables inconsistently, use different codes, or leave gaps, so models do not transfer cleanly~\cite{zhou2025representationlearningadvancemultiinstitutional}. Tackling each problem alone has not proved durable.

Tabular foundation models shine in small-sample regimes yet come without built-in domain adaptation~\cite{schneider2024foundation}. Domain adaptation work mostly targets images, not structured clinical data~\cite{guan2021domain}. We have not seen a single approach that marries pre-trained tabular models with cross-domain feature selection and unsupervised alignment, which keeps reliable deployment out of reach in mixed healthcare settings.

We present \emph{PANDA} (Pretrained Adaptation Network with Domain Alignment), a pragmatic attempt to pair a pre-trained tabular foundation model with unsupervised domain adaptation for cross-hospital pulmonary nodule malignancy prediction. The recipe mixes three pieces: (1) TabPFN's small-sample modeling via meta-training on millions of synthetic tasks; (2) Transfer Component Analysis (TCA) to align feature distributions across hospitals while keeping signal; and (3) Recursive Feature Elimination (RFE) to surface stable clinical variables across sites, softening feature heterogeneity. The goal is cross-hospital prediction while holding the high sensitivity (94.4\%) screening usually demands.

The promise rests on four points that seem worth testing in practice: it appears to be the first time a pre-trained tabular foundation model is paired with domain adaptation in a medical setting; external validation hints the approach might travel across hospital systems without falling apart; the sensitivity target of 94.4\% lines up with what screening workflows expect, with calibration and decision-curve gains to match; and the alignment path tries to juggle small sample sizes, class imbalance, and distribution shift in one pass.
