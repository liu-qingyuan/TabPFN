\section{Theoretical Analysis of Cross-Hospital Generalization}
\label{sec:analysis}

This chapter moves beyond the phenomenological reporting of performance metrics to establish a rigorous theoretical understanding of \textit{why} the PANDA framework succeeds where traditional methods falter. We posit that the failure of classical models (e.g., GBDTs, CNNs) in cross-hospital deployment is not merely an engineering oversight but a violation of the fundamental assumptions of statistical learning theory—specifically, the assumption that training and test data are drawn independent and identically distributed (i.i.d.) from the same joint distribution $P(\inputspace, \labelspace)$.

By dissecting PANDA through the lens of the Ben-David et al. generalization bound for domain adaptation, we demonstrate that the architecture is a direct algorithmic response to the theoretical decomposition of target error.

\subsection{Theoretical Foundation: The Generalization Bound}
\label{subsec:generalization_bound}

To analyze the generalization capability of PANDA, we adopt the seminal learning-theoretic framework established by Shai Ben-David et al. (2010), a cornerstone of domain adaptation theory \cite{ben2010theory}. Their work provides a rigorous upper bound on the target domain error, decomposing it into observable and optimizable quantities. This bound explains why minimizing source error alone is insufficient and necessitates explicit domain alignment strategies like TCA.

\begin{theorem}[Ben-David et al., 2010]
Let $\hypothesisclass$ be a hypothesis space of VC-dimension $\vcdim$. If $\sourcedata$ and $\targetdata$ are samples of size $n$ drawn from source distribution $\sourcedomaindist$ and target distribution $\targetdomaindist$ respectively, then for any $\confidence \in (0, 1)$, with probability at least $1 - \confidence$, for every $\hypothesis \in \hypothesisclass$:
\begin{equation}
    \label{eq:ben_david_bound}
    \targeterror \leq \sourceerror + \frac{1}{2} \domaindivergence(\sourcedomaindist, \targetdomaindist) + \adaptabilityterm + \mathcal{O}\left(\sqrt{\frac{\vcdim \log n}{n} + \log \frac{1}{\confidence}}\right)
\end{equation}
\end{theorem}

This inequality dictates that minimizing target error $\targeterror$ requires simultaneously minimizing three distinct terms:
\begin{enumerate}
    \item \textbf{Source Risk} $\sourceerror$: The expected error on the source domain.
    \item \textbf{Domain Divergence} $\domaindivergence(\sourcedomaindist, \targetdomaindist)$: The $\mathcal{H}\Delta\mathcal{H}$-divergence, measuring the discrepancy between the marginal feature distributions.
    \item \textbf{Adaptability Term} $\adaptabilityterm$: The error of the ideal joint hypothesis, $\adaptabilityterm = \min_{\hypothesis \in \hypothesisclass} (\sourceerror + \targeterror)$, which captures irreducible error due to concept shift (mismatch in conditional distributions $P(Y|X)$).
\end{enumerate}

\subsubsection{The $\mathcal{H}\Delta\mathcal{H}$-Divergence}
A key contribution of Ben-David's theory is the $\mathcal{H}\Delta\mathcal{H}$-divergence, which quantifies domain distance with respect to the hypothesis class $\mathcal{H}$. It is defined as:
\begin{equation}
    \domaindivergence(\sourcedomaindist, \targetdomaindist) = 2 \sup_{h, h' \in \mathcal{H}} \left| \Pr_{x \sim \sourcedomaindist}[h(x) \neq h'(x)] - \Pr_{x \sim \targetdomaindist}[h(x) \neq h'(x)] \right|
\end{equation}
Intuitively, this metric measures the maximal discrepancy in the "disagreement" between any two classifiers in the hypothesis space across the two domains. If there exist two classifiers $h, h'$ that make similar predictions on the source domain (low disagreement) but make widely different predictions on the target domain (high disagreement), the divergence is high. This implies that the source domain does not adequately constrain the behavior of classifiers on the target domain, leading to potential negative transfer. Alignment methods like TCA and CORAL aim to transform the feature space such that this divergence is minimized.

\subsection{Minimizing Source Risk $\sourceerror$: The TabPFN Mechanism}
\label{subsec:analysis_source_risk}

The first challenge is the small sample size ($\sourcedatasize \approx 295$). In this regime, standard Empirical Risk Minimization (ERM) is prone to high variance. Deep neural networks typically require $n > 10^4$ to generalize, while GBDTs often overfit, effectively "memorizing" the noise in $\sourcedata$ rather than learning the structure of $\sourcedomaindist$.

\subsubsection{Prior-Data Fitted Networks vs. Parametric Learning}
Traditional parametric learning optimizes weights $\modelparams$ to minimize loss on $\sourcedata$: $\hat{\modelparams} = \arg\min_{\modelparams} \sum_{({\featurevec},{\labelval}) \in \sourcedata} \loss(f_{\modelparams}(\featurevec), {\labelval})$. This process is data-hungry because it starts with uninformative priors.

In contrast, PANDA uses TabPFN, a \textbf{Prior-Data Fitted Network (PFN)}. It minimizes $\sourceerror$ not by gradient descent on $\sourcedata$, but by approximating the \textbf{Posterior Predictive Distribution (PPD)} using a Transformer pre-trained on millions of synthetic causal models:
\begin{equation}
    P({\labelval}_{\text{query}} \mid \queryvec, \sourcedata) \approx \int P({\labelval} \mid \featurevec, M) P(M \mid \sourcedata) \, dM
\end{equation}
By treating the source dataset $\sourcedata$ as a \textit{context} for Bayesian inference rather than a training set for parameter optimization, TabPFN acts as a massive regularizer. It enforces inductive biases favoring sparsity and piecewise smoothness.

\textbf{Empirical Consequence:} As shown in our results, TabPFN achieves a source AUC of \textbf{0.829}, significantly outperforming Random Forest (0.752) and XGBoost (0.742). This establishes a strictly lower starting point for the $\sourceerror$ term in the bound.

\subsection{Minimizing Divergence $\domaindivergence$: Latent Space TCA}
\label{subsec:analysis_divergence}

The second term, $\domaindivergence$, measures the discrepancy between domains. In our setting, scanner heterogeneity causes \textbf{Covariate Shift}: $\marginalsourcedist(\featurevec) \neq \marginaltargetdist(\featurevec)$.

\subsubsection{Why Feature Space Alignment?}
Aligning raw features is often suboptimal due to the complexity and non-linear dependencies of medical variables. Instead, PANDA applies TCA directly to the $\rfeselectedfeatures$. This is motivated by the observation that after RFE, the features are already a more robust and relevant subset, making them more amenable to linear alignment. The goal of this step is to directly align the marginal distributions of these pre-selected features between the source and target domains.


\subsubsection{The TCA Optimization Objective}
We employ Transfer Component Analysis (TCA) to find a projection $\tcaprojectionmatrix \in \mathbb{R}^{\rfecurrentdim \times \tcaprojectdim}$ that minimizes the Maximum Mean Discrepancy (MMD) between source and target features (after RFE). The objective is:
\begin{equation}
    \begin{aligned}
    \min_{\tcaprojectionmatrix} \quad & \tr({\tcaprojectionmatrix}^\top \kernelmatrix \mmdmatrix \kernelmatrix \tcaprojectionmatrix) + \regularizationparam \tr({\tcaprojectionmatrix}^\top \tcaprojectionmatrix) \\
    \text{s.t.} \quad & {\tcaprojectionmatrix}^\top \kernelmatrix \centeringmatrix \kernelmatrix \tcaprojectionmatrix = \identitymatrix
    \end{aligned}
\end{equation}
where $\kernelmatrix$ is the kernel matrix of RFE-selected features ($K_{ij} = \langle \featurevec_i, \featurevec_j \rangle$), $\mmdmatrix$ is the MMD indicator matrix, and $\centeringmatrix$ is the centering matrix. The constraint ${\tcaprojectionmatrix}^\top \kernelmatrix \centeringmatrix \kernelmatrix \tcaprojectionmatrix = \identitymatrix$ preserves data variance, ensuring the projection does not collapse informative signals while aligning means. Here, $\rfecurrentdim$ represents the dimensionality of the RFE-selected features.

\subsection{Minimizing Adaptability Error $\adaptabilityterm$: Cross-Domain RFE}
\label{subsec:analysis_adaptability}

The third term, $\adaptabilityterm$, represents the irreducible error of the best joint hypothesis. A large $\adaptabilityterm$ implies \textbf{Concept Shift} ($\conditionalps({\labelval}|\featurevec) \neq \conditionalpt({\labelval}|\featurevec)$), where the same feature value implies different risks across hospitals (e.g., "spiculation" caused by cancer in Hospital A vs. tuberculosis in Hospital B).

PANDA minimizes $\adaptabilityterm$ via \textbf{Cross-Domain Recursive Feature Elimination (RFE)}. By intersecting feature importance rankings from the source with availability and stability constraints, we effectively restrict the hypothesis class $\hypothesisclass$ to a subspace $\rfeselectedfeatures$ where the conditional distributions are approximately invariant:
\begin{equation}
    \conditionalps({\labelval} \mid \featurevec_{\rfeselectedfeatures}) \approx \conditionalpt({\labelval} \mid \featurevec_{\rfeselectedfeatures})
\end{equation}
Discarding unstable features (like subjective morphological scores) might slightly increase intrinsic source error $\sourceerror$ (bias), but it drastically reduces $\adaptabilityterm$, resulting in a tighter overall bound on target error.

\subsection{Synthesis: Linking Theory to Empirical Results}
\label{subsec:analysis_synthesis}

Table~\ref{tab:theory_empirics} synthesizes the theoretical components with our experimental findings, demonstrating how each PANDA component targets a specific term in the Ben-David bound.

\begin{table}[htbp]
  \centering
  \caption{Mapping Theoretical Error Terms to PANDA Components and Empirical Results}
  \label{tab:theory_empirics}
  \resizebox{\textwidth}{!}{
  \begin{tabular}{llll}
    \toprule
    \textbf{Error Term} & \textbf{Statistical Challenge} & \textbf{PANDA Component} & \textbf{Empirical Impact} \\
    \midrule
    $\sourceerror$ & Small Sample Size & TabPFN Backbone & Source AUC: $0.829$ vs $0.742$ (XGBoost) \\
    & Overfitting & (Meta-learned Priors) & (High sample efficiency) \\
    \midrule
    $\domaindivergence$ & Covariate Shift & TCA on RFE-Selected Features & Target Recall: $0.944$ vs $0.888$ (No-TCA) \\
    & (Scanner variation) & (MMD Minimization) & (Boundary correction) \\
    \midrule
    $\adaptabilityterm$ & Concept Shift & Cross-Domain RFE & Target AUC Gap: $<0.01$ (TableShift) \\
    & (TB vs Cancer) & (Stability Filtering) & (Robustness to population shift) \\
    \bottomrule
  \end{tabular}%
  }
\end{table}

\subsection{Conclusion}
 The PANDA framework is not merely an ensemble of heuristics but a theoretically grounded solution to the domain adaptation bound. TabPFN secures the source risk $\sourceerror$ via priors; TCA minimizes divergence $\domaindivergence$ via spectral alignment; and RFE minimizes adaptability error $\adaptabilityterm$ via stability-based pruning. Together, they ensure generalization in the challenging regime of small, heterogeneous medical datasets.
