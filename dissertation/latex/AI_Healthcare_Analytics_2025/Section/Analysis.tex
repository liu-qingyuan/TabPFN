\section{Theoretical Analysis of Cross-Hospital Generalization}
\label{sec:analysis}

This chapter moves beyond the phenomenological reporting of performance metrics to establish a rigorous theoretical understanding of \textit{why} the PANDA framework succeeds where traditional methods falter. We posit that the failure of classical models (e.g., GBDTs, CNNs) in cross-hospital deployment is not merely an engineering oversight but a violation of the fundamental assumptions of statistical learning theory—specifically, the assumption that training and test data are drawn independent and identically distributed (i.i.d.) from the same joint distribution $P(X, Y)$.

By dissecting PANDA through the lens of the Ben-David et al. generalization bound for domain adaptation, we demonstrate that the architecture is a direct algorithmic response to the theoretical decomposition of target error.

\subsection{Theoretical Foundation: The Generalization Bound}
\label{subsec:generalization_bound}

To analyze the generalization capability of PANDA, we adopt the seminal learning-theoretic bound proposed by Ben-David et al. (2010). This bound decomposes the error on the target domain into three observable and optimizable quantities.

\begin{theorem}[Ben-David et al., 2010]
Let $\mathcal{H}$ be a hypothesis space of VC-dimension $d_{VC}$. If $S$ and $T$ are samples of size $n$ drawn from source distribution $\mathcal{D}_S$ and target distribution $\mathcal{D}_T$ respectively, then for any $\delta \in (0, 1)$, with probability at least $1 - \delta$, for every $h \in \mathcal{H}$:
\begin{equation}
    \label{eq:ben_david_bound}
    \epsilon_T(h) \leq \epsilon_S(h) + \frac{1}{2} d_{\mathcal{H}\Delta\mathcal{H}}(\mathcal{D}_S, \mathcal{D}_T) + \lambda + \mathcal{O}\left(\sqrt{\frac{d_{VC} \log n}{n} + \log \frac{1}{\delta}}\right)
\end{equation}
\end{theorem}

This inequality dictates that minimizing target error $\epsilon_T(h)$ requires simultaneously minimizing three distinct terms:
\begin{enumerate}
    \item \textbf{Source Risk} $\epsilon_S(h)$: The expected error on the source domain.
    \item \textbf{Domain Divergence} $d_{\mathcal{H}\Delta\mathcal{H}}(\mathcal{D}_S, \mathcal{D}_T)$: A measure of discrepancy between the marginal feature distributions $P_S(\mathbf{x})$ and $P_T(\mathbf{x})$.
    \item \textbf{Adaptability Term} $\lambda$: The error of the ideal joint hypothesis, $\lambda = \min_{h \in \mathcal{H}} (\epsilon_S(h) + \epsilon_T(h))$, which captures irreducible error due to concept shift.
\end{enumerate}

Standard approaches often fail because they optimize only one term (e.g., GBDTs minimize $\epsilon_S(h)$ but ignore divergence) or optimize naively (e.g., aligning raw features reduces divergence but may increase $\lambda$ by distorting semantic boundaries). PANDA is explicitly engineered to address all three, as detailed below.

\subsection{Minimizing Source Risk $\epsilon_S(h)$: The TabPFN Mechanism}
\label{subsec:analysis_source_risk}

The first challenge is the small sample size ($n_s \approx 295$). In this regime, standard Empirical Risk Minimization (ERM) is prone to high variance. Deep neural networks typically require $n > 10^4$ to generalize, while GBDTs often overfit, effectively "memorizing" the noise in $S$ rather than learning the structure of $\mathcal{D}_S$.

\subsubsection{Prior-Data Fitted Networks vs. Parametric Learning}
Traditional parametric learning optimizes weights $\theta$ to minimize loss on $S$: $\hat{\theta} = \arg\min_{\theta} \sum_{(\mathbf{x},y) \in S} \mathcal{L}(f_\theta(\mathbf{x}), y)$. This process is data-hungry because it starts with uninformative priors.

In contrast, PANDA uses TabPFN, a \textbf{Prior-Data Fitted Network (PFN)}. It minimizes $\epsilon_S(h)$ not by gradient descent on $S$, but by approximating the \textbf{Posterior Predictive Distribution (PPD)} using a Transformer pre-trained on millions of synthetic causal models:
\begin{equation}
    P(y_{\text{query}} \mid \mathbf{x}_{\text{query}}, S) \approx \int P(y \mid \mathbf{x}, M) P(M \mid S) \, dM
\end{equation}
By treating the source dataset $S$ as a \textit{context} for Bayesian inference rather than a training set for parameter optimization, TabPFN acts as a massive regularizer. It enforces inductive biases favoring sparsity and piecewise smoothness.

\textbf{Empirical Consequence:} As shown in our results, TabPFN achieves a source AUC of \textbf{0.829}, significantly outperforming Random Forest (0.752) and XGBoost (0.742). This establishes a strictly lower starting point for the $\epsilon_S(h)$ term in the bound.

\subsection{Minimizing Divergence $d_{\mathcal{H}\Delta\mathcal{H}}$: Latent Space TCA}
\label{subsec:analysis_divergence}

The second term, $d_{\mathcal{H}\Delta\mathcal{H}}$, measures the discrepancy between domains. In our setting, scanner heterogeneity causes \textbf{Covariate Shift}: $P_S(\mathbf{x}) \neq P_T(\mathbf{x})$.

\subsubsection{Why Feature Space Alignment?}
Aligning raw features is often suboptimal due to the complexity and non-linear dependencies of medical variables. Instead, PANDA applies TCA directly to the RFE-selected features. This is motivated by the observation that after RFE, the features are already a more robust and relevant subset, making them more amenable to linear alignment. The goal of this step is to directly align the marginal distributions of these pre-selected features between the source and target domains.


\subsubsection{The TCA Optimization Objective}
We employ Transfer Component Analysis (TCA) to find a projection $\mathbf{W} \in \mathbb{R}^{k \times m}$ that minimizes the Maximum Mean Discrepancy (MMD) between source and target features (after RFE). The objective is:
\begin{equation}
    \begin{aligned}
    \min_{\mathbf{W}} \quad & \text{tr}(\mathbf{W}^\top \mathbf{K} \mathbf{L} \mathbf{K} \mathbf{W}) + \mu \text{tr}(\mathbf{W}^\top \mathbf{W}) \\
    \text{s.t.} \quad & \mathbf{W}^\top \mathbf{K} \mathbf{H} \mathbf{K} \mathbf{W} = \mathbf{I}
    \end{aligned}
\end{equation}
where $\mathbf{K}$ is the kernel matrix of RFE-selected features ($K_{ij} = \langle \mathbf{x}_i, \mathbf{x}_j \rangle$), $\mathbf{L}$ is the MMD indicator matrix, and $\mathbf{H}$ is the centering matrix. The constraint $\mathbf{W}^\top \mathbf{K} \mathbf{H} \mathbf{K} \mathbf{W} = \mathbf{I}$ preserves data variance, ensuring the projection does not collapse informative signals while aligning means. Here, $k$ represents the dimensionality of the RFE-selected features.

\subsection{Minimizing Adaptability Error $\lambda$: Cross-Domain RFE}
\label{subsec:analysis_adaptability}

The third term, $\lambda$, represents the irreducible error of the best joint hypothesis. A large $\lambda$ implies \textbf{Concept Shift} ($P_S(Y|\mathbf{x}) \neq P_T(Y|\mathbf{x})$), where the same feature value implies different risks across hospitals (e.g., "spiculation" caused by cancer in Hospital A vs. tuberculosis in Hospital B).

PANDA minimizes $\lambda$ via \textbf{Cross-Domain Recursive Feature Elimination (RFE)}. By intersecting feature importance rankings from the source with availability and stability constraints, we effectively restrict the hypothesis class $\mathcal{H}$ to a subspace $\mathcal{F}^*$ where the conditional distributions are approximately invariant:
\begin{equation}
    P_S(Y \mid \mathbf{x}_{\mathcal{F}^*}) \approx P_T(Y \mid \mathbf{x}_{\mathcal{F}^*})
\end{equation}
Discarding unstable features (like subjective morphological scores) might slightly increase intrinsic source error $\epsilon_S$ (bias), but it drastically reduces $\lambda$, resulting in a tighter overall bound on target error.

\subsection{Synthesis: Linking Theory to Empirical Results}
\label{subsec:analysis_synthesis}

Table~\ref{tab:theory_empirics} synthesizes the theoretical components with our experimental findings, demonstrating how each PANDA component targets a specific term in the Ben-David bound.

\begin{table}[htbp]
  \centering
  \caption{Mapping Theoretical Error Terms to PANDA Components and Empirical Results}
  \label{tab:theory_empirics}
  \resizebox{\textwidth}{!}{
  \begin{tabular}{llll}
    \toprule
    \textbf{Error Term} & \textbf{Statistical Challenge} & \textbf{PANDA Component} & \textbf{Empirical Impact} \\
    \midrule
    $\epsilon_S(h)$ & Small Sample Size & TabPFN Backbone & Source AUC: $0.829$ vs $0.742$ (XGBoost) \\
    & Overfitting & (Meta-learned Priors) & (High sample efficiency) \\
    \midrule
    $d_{\mathcal{H}\Delta\mathcal{H}}$ & Covariate Shift & TCA on RFE-Selected Features & Target Recall: $0.944$ vs $0.888$ (No-TCA) \\
    & (Scanner variation) & (MMD Minimization) & (Boundary correction) \\
    \midrule
    $\lambda$ & Concept Shift & Cross-Domain RFE & Target AUC Gap: $<0.01$ (TableShift) \\
    & (TB vs Cancer) & (Stability Filtering) & (Robustness to population shift) \\
    \bottomrule
  \end{tabular}%
  }
\end{table}

\subsection{Conclusion}
 The PANDA framework is not merely an ensemble of heuristics but a theoretically grounded solution to the domain adaptation bound. TabPFN secures the source risk $\epsilon_S$ via priors; TCA minimizes divergence $d_{\mathcal{H}\Delta\mathcal{H}}$ via spectral alignment; and RFE minimizes adaptability error $\lambda$ via stability-based pruning. Together, they ensure generalization in the challenging regime of small, heterogeneous medical datasets.
