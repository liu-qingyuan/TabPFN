\section{Conclusion}
\label{sec:conclusion}

This work links pre-trained tabular foundation models with domain adaptation to address long-standing issues in tabular learning under distribution shift. PANDA suggests that foundation-model priors and statistical alignment can reinforce one another, helping models generalize from scarce, heterogeneous samples where standard supervised approaches often stumble. The evidence is not sweeping, but it does point toward a practical recipe rather than a one-off trick.

Several methodological themes stand out. Pre-trained representations reduce the effective sample burden, letting high-capacity models behave sensibly in low-data regimes. Cross-domain feature selection pinpoints predictors that consistently transfer between sites, which makes alignment less fragile. Embedding TCA into these smoother representation spaces also seems to make domain transitions more workable. Taken together, these pieces outline a reasonable blueprint for adapting pre-trained tabular models across domains without relying on abundant labels.

Beyond pulmonary nodules, the same ingredients likely extend to other structured settings with small samples and noticeable shift--financial risk scores that change across branches, industrial monitoring when sensors drift, or hospital-adjacent analytics where coding practices evolve. PANDA is meant as a reusable template that treats pre-trained representations as portable priors rather than site-specific quirks.

The claims about smoother representations, feature-selection interactions, and reduced sample complexity align with the observed reduction in discrepancy and the improved external performance, hinting that pre-trained tabular models may broaden what is feasible in domain adaptation.

Open questions remain: scaling to larger tabular foundation models, moving toward multimodal pre-training, tightening feature selection for distributional robustness, and handling continual shift. As tabular models mature, pairing them with principled alignment may redefine how we handle shift.

In sum, PANDA frames tabular domain adaptation around pre-trained representations that support cross-domain generalization, aiming for deployments where shift is the rule rather than the exception. The same recipe now holds across private clinical cohorts and a public TableShift benchmark, hinting at dataset-agnostic, shift-resilient generalization beyond pulmonary nodules.
\label{sec:concl-end}
