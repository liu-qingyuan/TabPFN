\documentclass[10pt,aspectratio=169]{beamer}
\usepackage{poly}

\usepackage[UTF8]{ctex}
\usepackage{amsmath, amssymb}
\usepackage{graphicx}

\title{AI4Medicine:跨域表格预测研究}
\subtitle{PANDA:预训练模型与TCA联合的领域适应框架}
\author{刘青沅}
\institute[PolyU]{Department of Computing, The Hong Kong Polytechnic University}
\date{\today}

\begin{document}
\maketitle

\section{PANDA 框架概览}

\begin{frame}{PANDA 方法框架总览}

\textbf{PANDA(Pretrained Adaptation Network with Domain Alignment)}框架包括两个核心阶段:

\vspace{1em}
\textbf{阶段一:训练阶段(Source Domain)}
\begin{itemize}
    \item 使用 Cohort A(源域数据)进行训练;
    \item 特征选择 + 缺失值填补 + 类别特征处理 + 类别不平衡校正;
    \item 输入进入一个 \textbf{64 个子模型构成的预训练模型集成结构};
\end{itemize}

\vspace{1em}
\textbf{阶段二:测试阶段(Target Domain)}
\begin{itemize}
    \item 使用 TCA(Transfer Component Analysis)对源域与目标域进行特征对齐;
    \item 经过对齐后的目标样本被送入相同的 64 个模型集成结构中;
    \item 预测结果通过 Softmax 平均后输出,结合类别重权重增强罕见类别的预测能力。
\end{itemize}

\end{frame}


\begin{frame}{PANDA 模型结构图示(图像版)}

\begin{figure}[h]
    \centering
    \includegraphics[width=.85\textwidth]{PANDA_pipline.pdf}
    \caption{PANDA 方法的整体流程图,包含训练与测试阶段、TCA模块与模型内部结构}
\end{figure}

\end{frame}


\begin{frame}{PANDA 模型结构图示(图像版)}

\textbf{图像关键解读:}

\begin{itemize}
    \item \textbf{右上角:训练阶段(Source Domain)} \\
    原始表格数据经特征选择、缺失值填补与类别不平衡校正后,送入模型集成结构。

    \item \textbf{左上角:测试阶段(Target Domain)} \\
    测试数据先经过 \textbf{TCA 特征对齐},投影到与源域一致的特征空间后再进入模型。

    \item \textbf{下方:子模型结构(Pre-trained Tabular Foundation Model × 32)}
    \begin{itemize}
        \item 所有子模型共享主干结构,但采用 \textbf{不同的预处理组合}增强模型多样性;
        \item 共分为 \textbf{四种基础配置(各16个模型)},通过数值变换×类别编码组合:
        \begin{itemize}
            \item \textbf{配置1(16个):}分位数变换+SVD降维+序数编码,最终20维;
            \item \textbf{配置2(16个):}无变换+序数编码,保持8维;
            \item \textbf{配置3(16个):}分位数变换+SVD降维+数值编码,最终20维;
            \item \textbf{配置4(16个):}无变换+数值编码,保持8维;
        \end{itemize}
        \item 每种配置内的8个模型通过 \textbf{特征顺序扰动(Rotation 0-7)}增强多样性。
    \end{itemize}

    \item \textbf{输出阶段:} 所有子模型并行预测,输出通过 Softmax 聚合,并结合类别重加权,得到最终预测结果。
\end{itemize}

\end{frame}





\section{PANDA 核心组件详解}

\begin{frame}{TabPFN集成配置架构}

\textbf{四种基础配置设计}:通过\textbf{数值变换}和\textbf{类别编码}的2×2组合实现集成多样性

\vspace{1em}
\begin{table}[h]
\centering
\scriptsize
\begin{tabular}{|c|c|c|c|c|c|c|}
\hline
\textbf{配置ID} & \textbf{数值变换} & \textbf{保留原始} & \textbf{SVD降维} & \textbf{类别编码} & \textbf{维度} & \textbf{复杂度} \\
\hline
配置1 & quantile\_uni\_coarse & \checkmark & \checkmark & ordinal\_shuffled & 20维 & 高 \\
\hline
配置2 & none & \texttimes & \texttimes & ordinal\_shuffled & 8维 & 低 \\
\hline
配置3 & quantile\_uni\_coarse & \checkmark & \checkmark & numeric & 20维 & 高 \\
\hline
配置4 & none & \texttimes & \texttimes & numeric & 8维 & 低 \\
\hline
\end{tabular}
\end{table}

\textbf{64个集成成员分布}:每种基础配置分配16个集成成员,通过特征重排(rotate偏移0-7)增强多样性

\end{frame}

\begin{frame}[fragile]{TabPFN集成架构技术实现}

\textbf{源码实现}(\texttt{default\_classifier\_preprocessor\_configs()}):

\vspace{0.5em}
\scriptsize
\begin{verbatim}
def default_classifier_preprocessor_configs():
    return [
        # 配置1:高复杂度 + 序数编码 (8原始+8分位数+4SVD=20维)
        PreprocessorConfig("quantile_uni_coarse", 
                          append_original=True,
                          categorical_name="ordinal_shuffled",
                          global_transformer_name="svd"),
        # 配置2:低复杂度 + 序数编码 (8维)  
        PreprocessorConfig("none", 
                          categorical_name="ordinal_shuffled"),
        # 配置3:高复杂度 + 数值编码 (8原始+8分位数+4SVD=20维)
        PreprocessorConfig("quantile_uni_coarse", 
                          append_original=True,
                          categorical_name="numeric", 
                          global_transformer_name="svd"),
        # 配置4:低复杂度 + 数值编码 (8维)
        PreprocessorConfig("none", categorical_name="numeric"),
    ]
\end{verbatim}

\end{frame}

\begin{frame}{领域适应模块:TCA 对齐机制}

\textbf{Transfer Component Analysis(TCA)核心流程:}

\begin{itemize}
    \item 输入:源域数据 $X_s$ 与目标域数据 $X_t$;
    \item 构造核矩阵 $K$,在再生核希尔伯特空间(RKHS)中寻找一个投影矩阵 $W$;
    \item 优化目标:最小化源-目标分布差异(最大均值差异 MMD);
    \item 投影:$Z_s = K_s W,\quad Z_t = K_t W$;
    \item 在对齐空间中重用训练模型进行预测。
\end{itemize}

\vspace{1em}
\textbf{PANDA 中的作用:}
\begin{itemize}
    \item 显著减少跨医院数据分布差异;
    \item 不需要目标域标签(适用于现实无标签场景);
    \item 提升外部测试集(B医院)在 A 域训练模型下的鲁棒性。
\end{itemize}

\end{frame}

\section{集成架构优化分析}

\begin{frame}{集成配置架构对比:16:16 vs 8:8:8:8}

\textbf{架构演进}:从2种配置16:16分布优化为4种配置8:8:8:8分布

\vspace{1em}
\textbf{旧架构(8:8:8:8分布)}:
\begin{itemize}
    \item \textbf{配置1(8个)}:分位数变换+SVD降维+序数编码,20维
    \item \textbf{配置2(8个)}:无变换+序数编码,8维
    \item \textbf{配置3(8个)}:分位数变换+SVD降维+数值编码,20维
    \item \textbf{配置4(8个)}:无变换+数值编码,8维
    \item \textbf{性能表现(Best11特征集)}:源域AUC 0.8444,无UDA目标域AUC 0.6554,经TCA提升至 0.7056
\end{itemize}

\vspace{1em}
\textbf{新架构(16:16:16:16分布)}:
\begin{itemize}
    \item \textbf{配置A(16个)}:分位数变换+SVD降维+序数编码
    \item \textbf{配置B(16个)}:无变换+数值编码
    \item \textbf{性能表现(Best7特征集)}:源域AUC 0.8285,无UDA目标域AUC 0.6948,经TCA提升至 0.7013
\end{itemize}

\end{frame}

\begin{frame}{架构优化效果分析:权衡分析}

\textbf{架构对比的全面评估}:

\begin{table}[h]
\centering
\scriptsize
\begin{tabular}{|c|c|c|c|c|c|c|}
\hline
\textbf{架构类型} & \textbf{特征集} & \textbf{源域AUC} & \textbf{无UDA基线} & \textbf{目标域AUC} & \textbf{TCA改进} & \textbf{改进倍数} \\
\hline
旧架构(8:8:8:8) & best11 & \textcolor{green}{\textbf{0.8444}} & \textcolor{red}{\textbf{0.6554}} & 0.7056 & \textcolor{green}{\textbf{+0.0502}} & 1.0× \\
\hline
新架构(16:16:16:16) & best7 & 0.8285 & \textcolor{green}{\textbf{0.6948}} & 0.7013 & +0.0065 & \textcolor{red}{\textbf{0.13×}} \\
\hline
\end{tabular}
\end{table}

\vspace{1em}
\textbf{架构权衡分析}:
\begin{itemize}
    \item \textcolor{green}{\textbf{优势}}:新架构在无UDA条件下表现更稳(0.6554→0.6948,+6.0\%),同时保持三十二模型集成规模
    \item \textcolor{red}{\textbf{劣势}}:TCA增益明显下降(+0.0502→+0.0065,约降低87\%),目标域AUC略低(0.7056→0.7013)
    \item \textbf{净效果}:若更关注原始预测基线,可优先新架构;若强调域适应增益,旧架构仍具优势
\end{itemize}

\vspace{1em}
\textbf{临床决策建议}:
\begin{itemize}
    \item \textbf{单医院部署}:可考虑旧架构(更高的无UDA基线性能)
    \item \textbf{跨医院部署}:推荐新架构(更强的域适应能力)
    \item \textbf{混合策略}:根据部署场景选择合适的架构配置
\end{itemize}

\end{frame}

\section{特征选择扫描分析}

\begin{frame}{Feature Sweep Analysis:Best3-Best58完整扫描}

\textbf{扫描配置}:
\begin{itemize}
    \item 特征范围:best3 ~ best58(共56种配置)
    \item 并行工作进程:4
    \item 成功分析:56/56(100\%成功率)
\end{itemize}

\vspace{1em}
\textbf{核心发现}:
\begin{itemize}
    \item \textbf{最佳源域性能}:best14 (AUC: 0.8466)
    \item \textbf{最佳目标域基线}:best7 (AUC: 0.6868)  
    \item \textbf{最佳TCA域适应}:\textcolor{red}{\textbf{best11 (AUC: 0.7056)}}
    \item \textbf{最大TCA改进}:\textcolor{red}{\textbf{best11 (+0.0502)}}
\end{itemize}

\vspace{1em}
\textbf{性能趋势}:
\begin{itemize}
    \item 平均源域AUC:0.8221,平均目标域TCA AUC:0.6501
    \item 平均TCA改进:+0.0104,best11达到最大改进效果
\end{itemize}

\end{frame}

\section{模型性能评估}

\begin{frame}{Best11最优配置详细分析}

% \textbf{特征配置}:11个特征的最优组合,特征集ID: best11

% \vspace{1em}
\textbf{源域10折交叉验证结果}:
\begin{table}[h]
\centering
\tiny
\begin{tabular}{|c|c|c|c|c|c|}
\hline
\textbf{方法} & \textbf{AUC} & \textbf{Accuracy} & \textbf{F1} & \textbf{Precision} & \textbf{Recall} \\
\hline
\textcolor{red}{\textbf{PANDA}} & \textcolor{red}{\textbf{0.8444}} & \textcolor{red}{\textbf{0.7763}} & \textcolor{red}{\textbf{0.8332}} & \textcolor{red}{\textbf{0.8066}} & 0.8675 \\
\hline
XGBOOST & 0.8003 & 0.7663 & 0.8201 & 0.8120 & 0.8357 \\
\hline
RF & 0.7908 & 0.7525 & 0.8218 & 0.7682 & 0.8886 \\
\hline
GBDT & 0.7754 & 0.7392 & 0.8051 & 0.7837 & 0.8409 \\
\hline
Paper method & 0.7631 & 0.7224 & 0.8101 & 0.7227 & 0.9254 \\
\hline
PKUPH & 0.6640 & 0.6782 & 0.7672 & 0.7148 & 0.8354 \\
\hline
DT & 0.6283 & 0.6476 & 0.7138 & 0.7374 & 0.6994 \\
\hline
MAYO & 0.6049 & 0.3592 & 0.0000 & 0.0000 & 0.0000 \\
\hline
SVM & 0.5207 & 0.4847 & 0.3972 & 0.7755 & 0.2795 \\
\hline
\end{tabular}
\end{table}

\textbf{目标域迁移测试结果}:
\begin{table}[h]
\centering
\tiny
\begin{tabular}{|c|c|c|c|c|c|}
\hline
\textbf{方法} & \textbf{AUC} & \textbf{Accuracy} & \textbf{F1} & \textbf{Precision} & \textbf{Recall} \\
\hline
\textcolor{red}{\textbf{PANDA (UDA)}} & \textcolor{red}{\textbf{0.7056}} & \textcolor{red}{\textbf{0.7000}} & \textcolor{red}{\textbf{0.8014}} & \textcolor{red}{\textbf{0.7099}} & \textcolor{red}{\textbf{0.9200}} \\
\hline
RF & 0.6855 & 0.7000 & 0.7982 & 0.7160 & 0.9026 \\
\hline
GBDT & 0.6788 & 0.6895 & 0.7784 & 0.7346 & 0.8308 \\
\hline
Paper LR & 0.6678 & 0.6737 & 0.7911 & 0.6825 & 0.9429 \\
\hline
PANDA (no UDA) & 0.6554 & 0.6842 & 0.7794 & 0.7211 & 0.8480 \\
\hline
PKUPH & 0.6356 & 0.6947 & 0.7838 & 0.7329 & 0.8474 \\
\hline
XGBoost & 0.6233 & 0.6368 & 0.7266 & 0.7230 & 0.7353 \\
\hline
DT & 0.5933 & 0.6263 & 0.7081 & 0.7263 & 0.6962 \\
\hline
Mayo & 0.5837 & 0.3421 & 0.0000 & 0.0000 & 0.0000 \\
\hline
SVM & 0.4295 & 0.6211 & 0.7481 & 0.6589 & 0.8718 \\
\hline
\end{tabular}
\end{table}

% \vspace{0.5em}
% \textbf{关键发现}:PANDA+TCA在所有方法中AUC最高(0.7056),召回率92\%适合恶性结节筛查

\end{frame}

\begin{frame}{旧架构(8:8:8:8)可视化分析}

\begin{figure}[h]
    \centering
    \includegraphics[width=0.4\textwidth]{best11_analysis_new.pdf}
    % \caption{旧架构(Best11):综合性能分析图表,显示TCA增益显著的历史配置}
\end{figure}

\end{frame}

\begin{frame}{新架构(16:16:16:16)对比分析}

\begin{figure}[h]
    \centering
    \includegraphics[width=0.4\textwidth]{best7_analysis_old.pdf}
    % \caption{新架构(Best7):对比基准分析,显示无UDA基线更高但域适应效果有限}
\end{figure}

\end{frame}

\begin{frame}{架构性能热图对比}

\begin{columns}
\begin{column}{0.48\textwidth}
\begin{figure}[h]
    \centering
    \includegraphics[width=0.65\textwidth]{best11_heatmaps_new.pdf}
    \caption{旧架构:Best11特征集}
\end{figure}
\end{column}

\begin{column}{0.48\textwidth}
\begin{figure}[h]
    \centering
    \includegraphics[width=0.65\textwidth]{best7_heatmaps_old.pdf}
    \caption{新架构:Best7特征集}
\end{figure}
\end{column}
\end{columns}

\vspace{1em}
\textbf{热图解读}:旧架构在TCA域适应方面依旧占优;新架构在无UDA基线更稳、但跨域增益有限

\end{frame}

\section{总结与临床意义}

\begin{frame}{技术创新与架构权衡总结}

\textbf{PANDA 框架核心技术创新:}

\begin{itemize}
    \item \textbf{TabPFN集成架构}:旧架构为4种基础配置×8个成员强化多样性,新架构收敛为2种配置×16个成员以稳住基线
    \item \textbf{TCA域适应机制}:无监督跨域特征对齐,在RKHS空间最小化MMD距离
    \item \textbf{特征选择优化}:通过Best3-Best58全扫描验证,\textcolor{red}{\textbf{best7配置在新架构下表现最佳}}
    \item \textbf{小样本处理能力}:在295训练样本下实现0.8285源域AUC(新架构)
\end{itemize}

\vspace{1em}
\textbf{架构权衡的关键发现:}
\begin{itemize}
    \item \textcolor{green}{\textbf{优势}}:新架构带来更高的无UDA基线和特征稳定性
    \item \textcolor{red}{\textbf{代价}}:TCA增益下降,仅提升0.0065 AUC,整体跨域收益低于旧架构
    \item \textbf{适用场景}:跨医院部署若以基线稳定为主可采新架构;若强调域适应收益,应继续沿用旧架构
    \item \textbf{系统性验证}:56种特征配置完整对比,16:16 vs 8:8:8:8架构分析
\end{itemize}

\end{frame}


\end{document}
