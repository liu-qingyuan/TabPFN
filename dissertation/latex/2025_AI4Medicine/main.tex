\documentclass{article} 
% ✅ Standard class file (article.cls) — conforms to: 
% “Please use any of the standard class files such as article.cls, revtex.cls or amsart.cls.”

% Language setting
% Replace `english' with e.g. `spanish' to change the document language
\usepackage[english]{babel} 
% ✅ Standard language package — allowed

% Set page size and margins
% Replace `letterpaper' with `a4paper' for UK/EU standard size
\usepackage[letterpaper,top=2cm,bottom=2cm,left=3cm,right=3cm,marginparwidth=1.75cm]{geometry}
% ✅ Geometry used to control margins — acceptable since visual formatting is not essential
% “There is no need to spend time visually formatting the manuscript...”

% Useful packages
\usepackage{amsmath, amssymb} 
% ✅ Standard math support — allowed

\usepackage{graphicx} 
% ✅ Required for figures — conforms to:
% “For graphics, we recommend graphicx.sty.”
% ⚠️ Submit a single .tex file only — all figures should be referenced using \includegraphics,
% and all content must reside in this file or included in-line.
% → “All textual material should be provided as a single file...”

\usepackage[colorlinks=true, allcolors=blue]{hyperref}
% ✅ Optional, allowed if needed; bookmarks or PDF styling is not required

% ⚠️ Do NOT use any custom macros or commands
% → “Remove all personal macros before submitting.”

% ⚠️ Do NOT use non-standard fonts (e.g., \usepackage{times}, \usepackage{helvet}, etc.)
% → “All textual material should be provided as a single file in default Computer Modern fonts.”

% ✅ Ensure this file compiles without errors or warnings before submission
% → “Please ensure that the complete .tex file compiles successfully on your own system...”

\usepackage{xcolor}
\newcommand{\red}[1]{{\color{red}#1}} % comments
\definecolor{lbcolor}{RGB}{13, 151, 175}
\newcommand{\lb}[1]{{\color{lbcolor}#1}} % comments

\title{Predicting Malignancy of Pulmonary Nodules Across Hospitals Using Transferable Machine Learning on Small and Imbalanced Datasets}

\author{
    First Author\textsuperscript{1},\quad
    Qingyuan Liu\textsuperscript{2},\quad
    Wenqi Fan\textsuperscript{2,}\thanks{Corresponding author: wenqi.fan@polyu.edu.hk}
}

\date{
    \textsuperscript{1}Department of XYZ, University A, City, Country\\
    \textsuperscript{2}Department of Computing, The Hong Kong Polytechnic University, Hong Kong SAR, China
}


\begin{document}
\maketitle

\begin{abstract}
Medical AI systems face persistent challenges due to limited data, class imbalance, and inter-institutional heterogeneity. In particular, small sample sizes and skewed class distributions are common in clinical datasets, often introducing biases and undermining model reliability. Furthermore, models trained in one hospital or cohort frequently underperform when deployed at a different site, as differences in equipment, protocols, and patient demographics lead to domain shift. To address these issues, we propose a novel cross-domain adaptation framework that integrates a pre-trained Transformer-based foundation model with a classic domain alignment technique, Transfer Component Analysis (TCA). This approach leverages the rich representational power of Transformer models (pre-trained on large tabular data) and the distribution-matching capability of TCA to improve generalization under small-sample conditions. We validated our method on structured clinical datasets from two independent hospitals (295 patients in Hospital A for training, 190 patients in Hospital B for testing), focusing on malignant vs. benign pulmonary nodule classification. Our model achieved high area under the ROC curve (AUC) along with improved sensitivity and specificity on the external test set, outperforming several conventional domain adaptation baselines. These results demonstrate that the proposed framework achieves robust cross-domain prediction by aligning feature distributions and learning generalizable classification functions over structured clinical data. Interpretation: By effectively combating data scarcity and shift, the model exhibits strong generalizability and shows potential for reliable deployment in multi-center medical settings.
\end{abstract}

\section{Introduction}
\label{sec:intro-start}

Accurate diagnostic risk prediction is a canonical setting in which advances in machine learning (ML) and data analytics can have immediate clinical impact. In pulmonary nodule screening, classical clinical risk scores such as the Mayo Clinic, Veterans Affairs, Brock (PanCan), PKUPH, and Li models achieve strong internal discrimination (AUC $\approx 0.80$--$0.94$) by fitting logistic regressions to carefully curated, single-center cohorts~\cite{swensen1997chest,mcwilliams2013probability,li2011development,he2021novel,zhang_comprehensive_2022,liu_establishment_2024}. However, meta-analyses and external validations show that their performance can decline to AUCs of $0.60$--$0.75$ when transported to community-screening sites, TB-endemic regions, or demographically distinct populations~\cite{garau_external_2020,zhang_comprehensive_2022,liu_establishment_2024}. These degradations, driven by shifts in disease prevalence, acquisition protocols, and background pathology (e.g., granulomas versus tuberculosis), illustrate how non-adaptive risk calculators can become unreliable in cross-hospital practice.

From an AI perspective, these failures reflect a mismatch between the complexity of real-world deployment and the simplifying assumptions of classical supervised learning. Clinical tabular datasets are typically small, imbalanced, and heterogeneous: even high-value registries often contain only a few hundred labeled patients, with malignant nodules representing a minority class. Features are high-dimensional and only partially overlapping across sites, as institutions log different biomarker panels, coding schemes, and acquisition protocols. This combination of small sample size, distribution shift, and feature-space mismatch violates the closed-world assumptions underlying many standard models and exposes the limits of purely local training.

The algorithmic trajectory for structured data in healthcare mirrors this tension. Gradient-boosted decision trees (GBDTs), led by XGBoost, LightGBM, and related ensembles, remain the workhorses of tabular ML because they tolerate heterogeneous scales, missingness, and noisy categorical codes~\cite{chen2016xgboost,gorishniy2021revisiting}. Neural ``deep tabular'' architectures—including TabNet, TabTransformer, SAINT, FT-Transformer, NODE, and other attention- or gating-based variants—extend differentiability to structured data and enable multimodal fusion, but they require substantial data, are sensitive to hyperparameters, and often lag well-tuned trees on clinical benchmarks when effective sample sizes are small~\cite{arik2021tabnet,huang2020tabtransformer,somepalli2021saint,borisov2022deep,gorishniy2021revisiting}. Radiomics pipelines engineer thousands of texture descriptors from CT volumes, and 3D convolutional neural networks (CNNs) achieve strong internal performance on datasets such as NLST and LIDC, yet their scanner sensitivity and propensity for shortcut learning often negate cross-site gains: external validations reveal double-digit AUC drops when voxel spacing, reconstruction kernels, or case mix shift, and models can latch onto hospital-specific artifacts rather than biological signals~\cite{ardila_end--end_2019,zech_variable_2018,garau_external_2020,hellin2024unraveling}.

More recently, tabular foundation models and large tabular language models have emerged as promising directions. TabPFN meta-learns a transformer that approximates Bayesian posterior predictions across millions of synthetic tabular tasks, delivering hyperparameter-free, small-sample inference via in-context learning~\cite{hollmann2025accurate,schneider2024foundation}. Successors such as TabPFN-2.5 and drift-resilient TabPFN extend context length, relax attention bottlenecks, and incorporate simulated drifts into the prior~\cite{noauthor_prior_nodate,noauthor_realistic_nodate,noauthor_automldrift-resilient_tabpfn_2025}. Other work explores more realistic priors and cross-domain training curricula~\cite{noauthor_closer_nodate,noauthor_realistic_nodate}, and ``TabLLM''-style approaches serialize rows into prompts to reuse general-purpose reasoning from large language models~\cite{eremeev_turning_2025}. Parallel efforts investigate federated optimization and continual or on-device learning so that models can absorb new hospital evidence without breaching privacy constraints~\cite{guan2021domain,musa2025addressing}. Collectively, these developments define a new generation of AI systems for tabular healthcare data.

However, cross-hospital transfer remains fragile because three dominant pathologies of medical tabular data co-occur. First, sample scarcity: most pulmonary nodule cohorts contain only a few hundred labeled patients, which limits the stability of purely supervised training and amplifies overfitting~\cite{borisov2022deep}. Second, distribution shift: label prevalence, scanner kernels, demographics, and clinical workflows alter the marginal $P(X)$ and even the conditional $P(Y \mid X)$ between hospitals~\cite{koch2024distribution,guo_evaluation_2022}. Third, feature heterogeneity: sites log disjoint biomarker panels, adopt different measurement units, and follow distinct coding policies, which invalidates naive feature alignment and introduces missingness shifts~\cite{orouji_domain_nodate}. Domain adaptation research in imaging and wearables shows that adversarial training, cycle-consistent style transfer, optimal transport, and statistical moment matching can recover some performance under shift~\cite{guan2021domain,ahn_unsupervised_2023}, but these methods are rarely specialized for structured clinical data. Benchmarks such as TableShift and Wild-Time demonstrate that off-the-shelf robustness mechanisms still incur large out-of-distribution (OOD) gaps even when in-distribution accuracy is high~\cite{gardner_benchmarking_2024,yao2022wild}. Large-scale regulators and hospital governance boards increasingly regard shift detection, recalibration, and drift monitoring as core AI-safety requirements rather than optional post hoc checks~\cite{koch2024distribution}.

Tabular foundation models partially alleviate data scarcity, yet they inherit a closed-world assumption: the context set used during in-context learning is assumed to reflect the same joint distribution and feature schema as the query samples~\cite{schneider2024foundation}. When shifts in biomarkers, acquisition settings, or schemata emerge, attention weights may anchor on non-comparable neighbors, yielding overconfident yet incorrect predictions~\cite{noauthor_realistic_nodate,noauthor_automldrift-resilient_tabpfn_2025}. Emerging variants such as TabPFN-2.5 and drift-resilient TabPFN extend context length and inject synthetic drifts into the prior~\cite{noauthor_prior_nodate,noauthor_automldrift-resilient_tabpfn_2025}, but they remain sensitive to mismatched feature spaces and unlabeled target domains in the absence of explicit alignment. Tabular LLM approaches add reasoning capacity but incur substantial latency, quantization error for numerical values, and lack built-in clinical calibration, especially when lab panels or race-specific prevalences deviate from training distributions~\cite{eremeev_turning_2025}. Consequently, bridging the gap between high internal accuracy and safe cross-site deployment requires combining foundation models with principled unsupervised domain adaptation and feature selection that respect clinical realities.

Pulmonary nodule malignancy prediction is an archetypal stress test for these issues. Traditional clinical scores and their LASSO or GBDT successors were derived from narrowly defined cohorts with fixed demographic profiles and scanner protocols, so their coefficients silently encode source-specific prevalence, upper-lobe priors, and calcification heuristics~\cite{swensen1997chest,mcwilliams2013probability,li2011development,he2021novel,zhang_comprehensive_2022}. Meta-analyses across Asian screening programs and European cancer centers show that the same score threshold yields widely varying sensitivities (50--90\%) once smoking histories, granulomatous disease burdens, or acquisition kernels change~\cite{garau_external_2020,liu_establishment_2024}. Radiomics signatures tuned on sharp-kernel CTs lose discriminatory power on smooth-kernel images unless aggressively harmonized, and even then residual scanner bias can dominate texture features~\cite{hellin2024unraveling}. 3D CNNs for end-to-end malignancy prediction exhibit similar behavior, with performance degrading under scanner upgrades or demographic shifts~\cite{ardila_end--end_2019,zech_variable_2018}. These failures underscore that, without explicit feature pruning and alignment, both classical and modern models can become confidently wrong.

Similar tensions arise in population-health settings such as the BRFSS race-shift diabetes task. Demographic composition, socioeconomic exposures, and survey-year wording alter the marginal distribution of risk factors, while diabetes prevalence rises from roughly 12.5\% in White respondents to 17.4\% in non-White cohorts, causing fixed operating points to misfire~\cite{gardner_benchmarking_2024}. Benchmarks such as TableShift and Wild-Time make explicit that covariate shift ($P_s(X) \neq P_t(X)$), label shift ($P_s(Y) \neq P_t(Y)$), and concept shift ($P_s(Y \mid X) \neq P_t(Y \mid X)$) often co-occur, and that classical empirical risk minimization (ERM) on the source domain does not control the divergence term that drives target error~\cite{gardner_benchmarking_2024,koch2024distribution,yao2022wild}. In practice, these shifts invalidate the implicit closed-world assumptions behind most off-the-shelf models.

Feature engineering and feature selection choices are therefore as important as model class. Clinical tables mix continuous laboratory values, ordinal scores, sparse categorical codes, and structured missingness; naive one-hot encoding can expand dimensionality and encode site-specific artifacts. Stability-driven feature pruning, hierarchical encoding of categorical variables, and unit-aware normalization reduce spurious site signatures and focus attention on shared, clinically interpretable signals~\cite{sun2019informative}. Recursive feature elimination (RFE) across domains further enforces schema overlap, trading a slight drop in ceiling accuracy for substantial gains in portability when hospitals differ, and it is particularly helpful in small-sample, high-dimensional, and imbalanced regimes such as pulmonary nodules and radiomics panels~\cite{sun2019informative,borisov2022deep}.

Taken together, the research gap is stark. Tree ensembles and deep tabular networks struggle with small, heterogeneous cohorts and typically require retraining when schemas change~\cite{chen2016xgboost,gorishniy2021revisiting,borisov2022deep}. Foundation models improve small-sample performance but assume matched domains and aligned schemas~\cite{hollmann2025accurate,schneider2024foundation,noauthor_realistic_nodate}. Generic domain adaptation methods rarely account for missing features, label drift, or unlabeled targets in clinical tables~\cite{guan2021domain,ahn_unsupervised_2023,gardner_benchmarking_2024,koch2024distribution}. Federated and continual learning strategies help with privacy and incremental updates but do not by themselves guarantee cross-hospital calibration~\cite{guan2021domain,musa2025addressing}. A credible solution must (i) retain sample efficiency via strong pre-trained priors, (ii) discard site-specific signals that cannot transfer, and (iii) align source and target representations without target labels, while exposing calibration behavior under prevalence drift.

This study therefore adopts a pragmatic stance and introduces \emph{PANDA} (Pretrained Adaptation Network with Domain Alignment), a framework designed to transform diagnostic prediction through advanced ML and data analytics in realistic cross-hospital settings. PANDA chains three complementary components. First, a pre-trained tabular foundation model (TabPFN) supplies strong inductive priors for small cohorts by meta-learning across millions of synthetic tasks and enabling hyperparameter-free inference~\cite{hollmann2025accurate,schneider2024foundation}. Second, cross-domain RFE prunes to biomarkers that are consistently available and stable across sites, mitigating schema mismatch and hospital-specific artifacts~\cite{sun2019informative}. Third, a statistical alignment module based on Transfer Component Analysis (TCA) projects source and target cohorts into a shared reproducing-kernel subspace using unlabeled target data, minimizing distributional divergence while preserving clinical variance~\cite{pan2010domain}. PANDA targets the explicit goal of cross-hospital pulmonary nodule prediction with screening-level sensitivity and is further validated on the TableShift BRFSS Diabetes race-shift benchmark~\cite{gardner_benchmarking_2024}, ensuring that the proposed approach addresses both clinical and population-level distribution shifts without adding bespoke modeling components for each dataset.

In summary, cross-hospital pulmonary nodule prediction and BRFSS race-shift diabetes prediction expose the same deployment realities: privacy constraints, schema mismatch, prevalence drift, and the need for sensitivity at clinically actionable thresholds~\cite{koch2024distribution,gardner_benchmarking_2024}. Existing AI toolkits—tree ensembles, deep tabular networks, tabular foundation models, tabular LLMs, and generic domain adaptation—each leave gaps relative to these constraints~\cite{chen2016xgboost,arik2021tabnet,huang2020tabtransformer,somepalli2021saint,borisov2022deep,schneider2024foundation,guan2021domain,ahn_unsupervised_2023}. By integrating pre-trained tabular priors, schema-aware feature selection, and unsupervised domain alignment into a single pipeline, PANDA aims to restore calibration and discrimination under realistic deployment shifts. The remainder of this manuscript formalizes the cross-domain problem, surveys related work in tabular learning and medical domain adaptation, and presents PANDA as a practical instantiation of this design philosophy.

\label{sec:intro-end}


\section{Results}

\subsection{Study Cohort}

This study utilized structured clinical data from two authoritative cancer centers in China to construct a cross-institutional training–testing framework for evaluating model generalization in real-world medical scenarios. The training cohort (Cohort A, $n=295$) was derived from Sun Yat-sen University Cancer Center (Guangzhou, China), consisting of patients diagnosed with solitary pulmonary nodules (SPNs) between January 2011 and December 2016. The external testing cohort (Cohort B, $n=190$) was obtained from Henan Tumor Hospital (Zhengzhou, China) between January 2013 and [month/year redacted].

All enrolled patients met the following inclusion criteria: (1) presence of SPNs with lesion diameter $\leq$ 3 cm detected by chest CT; (2) no history of extrapulmonary malignancies; (3) complete clinical, imaging, and laboratory records within 7 days before diagnosis; and (4) histologically confirmed diagnosis via CT-guided biopsy, bronchoscopy, thoracoscopy, or surgical resection.

Each sample was annotated with a binary label indicating malignancy status (0 = benign, 1 = malignant), and the core objective of this study was to develop a robust classifier for malignancy prediction under domain shift.

Cohort A contained 63 structured features covering demographics, vital signs, biochemistry, tumor markers, and imaging descriptors. Due to partial data omissions, Cohort B included 58 features—a subset of Cohort A. To ensure domain consistency, we applied recursive feature elimination (RFE) on the source cohort and selected the top 9 predictive features. One feature (Feature40, CYSC) was unavailable in Cohort B and excluded from final modeling. The remaining 8 features were used consistently across both domains and formed the basis for domain-adaptive learning.

Model performance was evaluated under two settings. First, we conducted 10-fold cross-validation on the training cohort (Cohort A) to assess within-domain predictive stability and robustness. Then, the final model was retrained using all samples in Cohort A and deployed on the target cohort (Cohort B) for external evaluation. Specifically, Cohort B was equally divided into 10 subsets, and the trained model was used to generate predictions on each subset without retraining. The final performance on the test set was reported as the mean and standard deviation across the 10 subsets. This strategy enables robust estimation of generalization ability under distribution shift while preserving the integrity of a single trained model.

Summary statistics of key variables in both cohorts are provided in Table~\ref{tab:cohort_summary}. While the distributions of several variables—such as age, upper lobe location, and tumor marker levels (CEA, CRE, NSE)—showed moderate shifts between domains, their structural alignment and clinical relevance make them suitable candidates for subsequent feature-aligned domain adaptation.


\begin{table}[htbp]
\centering
\caption{Summary statistics of the training (Cohort A) and testing (Cohort B) cohorts.}
\label{tab:cohort_summary}
\begin{tabular}{lcc}
\hline
\textbf{Characteristic} & \textbf{Cohort A (n = 295)} & \textbf{Cohort B (n = 190)} \\
\hline
Upper lobe & & \\
\quad Yes/Positive & 121 (41.0\%) & 98 (51.6\%) \\
\quad No/Negative & 174 (59.0\%) & 92 (48.4\%) \\
Age (years) & 56.95 $\pm$ 11.03 & 58.26 $\pm$ 9.57 \\
Lobe location (upper) & & \\
\quad Category 1 & 161 (54.6\%) & 98 (51.6\%) \\
\quad Category 2 & 29 (9.8\%) & 18 (9.5\%) \\
\quad Category 3 & 105 (35.6\%) & 74 (38.9\%) \\
DLCO1 & 5.90 $\pm$ 2.89 & 6.31 $\pm$ 1.55 \\
VC & 3.33 $\pm$ 0.80 & 2.92 $\pm$ 0.73 \\
CEA & 4.23 $\pm$ 6.90 & 4.15 $\pm$ 10.61 \\
CRE & 73.41 $\pm$ 17.16 & 62.94 $\pm$ 13.64 \\
NSE & 13.07 $\pm$ 3.90 & 13.82 $\pm$ 4.36 \\
Outcome (Malignant) & & \\
\quad Yes/Positive & 189 (64.1\%) & 125 (65.8\%) \\
\quad No/Negative & 106 (35.9\%) & 65 (34.2\%) \\
\hline
\end{tabular}
\end{table}




\subsection{Quantitative Performance}

To comprehensively evaluate the performance of our proposed framework, we constructed a multidimensional evaluation protocol encompassing classification metrics, statistical confidence estimation, visualization-based analysis, and domain alignment assessment. This protocol was applied across both internal (source-domain) and external (target-domain) validation settings, reflecting realistic deployment scenarios in multi-institutional medical environments.

Classification performance was assessed using five standard metrics widely adopted in clinical machine learning: area under the receiver operating characteristic curve (AUC), accuracy, F1-score, sensitivity (recall), and specificity (precision). These metrics collectively measure overall discriminative ability, correct classification rate, class balance, and positive/negative case detection capability. All performance indicators were averaged over 10-fold cross-validation to ensure robustness and reproducibility.

To quantify uncertainty in performance estimates, 95\% confidence intervals (CIs) for AUC were calculated via 1000-round bootstrap resampling. This provides a statistical measure of model stability and offers meaningful interpretability for clinical application.

We also employed visualization-based assessments, including ROC curves, calibration plots, and decision curve analysis (DCA), to examine model reliability and potential clinical benefit under varying decision thresholds. These tools offer intuitive comparisons between methods and supplement numerical evaluation.

To assess the effectiveness of domain adaptation, we further evaluated the distributional shift between source and target domains using both qualitative and quantitative analyses. Principal component analysis (PCA) and t-distributed stochastic neighbor embedding (t-SNE) were used to visualize domain overlap, while standardized Wasserstein distance, symmetric KL divergence, and maximum mean discrepancy (MMD) were calculated to quantify distribution alignment before and after adaptation.

Benchmark comparisons included four method groups: (1) the pre-trained tabular foundation model without domain adaptation; (2) traditional clinical scoring systems (e.g., PKUPH, Mayo); (3) classical machine learning algorithms (e.g., SVM, RF, GBDT, XGBoost); and (4) our TCA-enhanced foundation model. This rigorous comparative design enables clear attribution of performance gains to the domain adaptation strategy and highlights its value in real-world medical deployment scenarios.




\subsubsection{Internal Validation Performance (Train Cohort A)}

To evaluate the generalization capacity of different models on the source domain, we conducted 10-fold cross-validation using the training cohort (Cohort A). Model performance was assessed across five metrics: area under the ROC curve (AUC), accuracy, F1-score, precision, and recall. The averaged results for each method are summarized in descending order of AUC to highlight the most globally effective model.

Figure~\ref{fig:performance-heatmaps} presents a comprehensive comparison of model performance across both internal and external validation settings. In the source domain (Figure~\ref{fig:performance-heatmaps}a), our proposed model, based on a pre-trained tabular foundation model, achieved superior performance across all five metrics: AUC of 0.826, accuracy of 0.743, F1-score of 0.807, precision of 0.786, and recall of 0.841. These results demonstrate the model's strong classification capability and high sensitivity for identifying positive cases, which is critical in clinical screening scenarios.

In comparison, classical machine learning methods showed moderate performance. Random forest (RF) and XGBoost ranked third and fourth in AUC (0.752 and 0.742, respectively), maintaining relatively stable results but still falling short of our method. Other methods, such as GBDT, SVM, and decision tree (DT), exhibited lower F1-scores and accuracy, indicating limited modeling capacity on high-dimensional medical data.

Clinical scoring systems performed substantially worse than data-driven models. In particular, the Mayo score failed entirely in this setting, with an F1-score, precision, and recall of 0.000. Analysis of the confusion matrices revealed that the Mayo model consistently predicted all samples as benign (negative class), resulting in zero true positive predictions and consequently zero precision and recall for malignancy detection. The PKUPH score also underperformed across all metrics.

The rule-based reference model Paper method exhibited decent AUC (0.763) and F1-score (0.810) but was inferior to our method in terms of accuracy (0.722) and precision (0.723). This may indicate potential overfitting or sensitivity to class imbalance in the training data.

\subsubsection{External Validation Performance (Cross-Institutional Generalization A→B)}

To simulate a real-world clinical deployment across institutions, we evaluated model generalization under a cross-domain setting: training on Cohort A and testing on Cohort B. As illustrated in Figure~\ref{fig:performance-heatmaps}b, the TCA-augmented foundation model achieved the best performance across all external validation metrics: AUC of 0.709, F1-score of 0.811, and recall of 0.944. These results significantly outperformed the version without domain adaptation (AUC = 0.698), underscoring the critical role of TCA in mitigating domain shift and enhancing sensitivity on unseen target data.

In contrast, the non-adaptive version—though strong on the source domain—exhibited clear degradation on Cohort B, exemplifying a typical domain shift effect and validating the necessity of domain adaptation in cross-institutional scenarios.

Among other methods, Paper LR, RF, and PKUPH showed moderate generalization with AUCs ranging from 0.63 to 0.67. While Paper LR achieved high recall (0.943), it suffered from low precision (0.682) and accuracy (0.674), suggesting a tendency toward over-identifying positive cases. Classical ML models such as SVM, GBDT, XGBoost, and DT performed poorly, with AUCs below 0.63; notably, XGBoost and DT dropped to 0.567 and 0.509, respectively, indicating sensitivity to distribution shifts.

The Mayo clinical score failed entirely on the target domain, with F1-score, precision, and recall all equal to 0.000. Similar to the source domain performance, the Mayo model consistently predicted all samples as benign, resulting in zero true positive detections.

\begin{figure}[htbp]
    \centering
    \includegraphics[width=\linewidth]{combined_heatmaps_nature.pdf}
    \caption{\textbf{Performance comparison across source and target domains.} 
    \textbf{a} Source domain 10-fold cross-validation performance heatmap across five classification metrics (AUC, accuracy, F1-score, precision, recall). Models are sorted by descending AUC. The pre-trained tabular foundation model (PANDA) achieves the best overall performance across all metrics. \textbf{b} Cross-domain performance heatmap on the external validation set (Cohort B). All models were trained on Cohort A and evaluated on Cohort B without retraining. The TCA-enhanced model (PANDA) shows the highest AUC and recall, indicating improved generalization under domain shift.}
    \label{fig:performance-heatmaps}
\end{figure}



Taken together, the TCA-enhanced tabular foundation model demonstrated robust external performance and outperformed both machine learning and clinical baselines, validating its potential for reliable deployment in real-world multi-center settings.




\subsection{Domain Adaptation Evaluation}

To comprehensively assess the effectiveness of Transfer Component Analysis (TCA) in improving cross-domain classification, we conducted a three-part evaluation including feature space visualization, quantitative domain distance analysis, and metric-level performance comparison between domain-adaptive and non-adaptive models.

\textbf{(1) Feature Space Visualization via PCA and t-SNE}

Figure~\ref{fig:tca-visualization} illustrates the latent feature distributions of source and target domains before and after TCA transformation. Prior to adaptation (left panels), the target and source domain samples already show partial overlap in both PCA and t-SNE projections, but their central tendencies and density structures remain misaligned. After applying TCA (right panels), the target domain samples become more tightly aligned with the source domain, with cluster centers in closer proximity and distribution densities showing greater similarity. This indicates that TCA not only preserves the inherent structure of each domain but also learns a shared latent space where the two distributions are better matched, thereby facilitating more reliable cross-domain generalization.

\begin{figure}[htbp]
    \centering
    \includegraphics[width=1\linewidth]{TCA_dimensionality_reduction.pdf}
    \caption{TCA-based domain adaptation visualization and dimensionality reduction analysis. \textbf{a} PCA visualization of source and target domain samples before TCA transformation, showing partial overlap but differences in central tendency and density distribution. \textbf{b} PCA visualization after TCA transformation, where target samples align more closely with source samples and exhibit more consistent density structures. \textbf{c} t-SNE visualization before TCA transformation, indicating moderate clustering overlap but noticeable differences in distribution compactness. \textbf{d} t-SNE visualization after TCA transformation, showing improved alignment of cluster centers and more similar distribution densities across domains. These results confirm that TCA reduces subtle distribution mismatches by enhancing center alignment and density consistency, thereby supporting robust inter-institutional generalization.}
    \label{fig:tca-visualization}
\end{figure}

\textbf{(2) Domain Distance Quantification}

To quantify domain shift reduction achieved by TCA, we computed four standard distributional discrepancy metrics before and after adaptation. The results showed consistent improvements across all measures:

The normalized linear discrepancy between source and target domains decreased by 0.070 after TCA transformation, indicating improved alignment in the projected latent space. Similarly, the normalized Frechét distance was reduced by 0.018, the Wasserstein distance by 0.006, and the symmetric KL divergence by 0.022. These reductions, although varying in magnitude, consistently reflect the same trend—TCA effectively mitigates inter-domain statistical divergence under multiple distance metrics.

Taken together, these results confirm that TCA narrows domain discrepancies from complementary mathematical perspectives, thereby enabling more stable and transferable feature representations across hospital datasets.


\textbf{(3) Classification Performance Improvement}

Compared to the non-adaptive baseline, the TCA-enhanced model exhibited consistent improvements across multiple classification metrics on the target domain (Cohort B):

\begin{itemize}
  \item Accuracy improved from 0.658 to 0.711 (+5.3\%).
  \item F1-score increased from 0.767 to 0.811.
  \item Precision rose from 0.695 to 0.711.
  \item Recall increased from 0.856 to 0.944, reflecting substantial improvement in positive case detection.
  \item AUC improved from 0.681 to 0.709 (+2.8\%).
\end{itemize}

These consistent gains further validate the efficacy of TCA in enhancing cross-domain model performance, especially in sensitive clinical applications where recall is a key requirement.

Taken together, from latent space structure to statistical alignment and classification performance, the proposed TCA-enhanced framework demonstrated clear advantages in mitigating domain shift and improving generalization across hospital settings.




\subsection{Model Explainability via Feature Selection}

Interpretability and compactness are essential for clinical deployment of machine learning models. To reduce model complexity while maintaining predictive accuracy, we performed recursive feature elimination (RFE) using our pre-trained tabular foundation model as the estimator. This approach ranks all 63 input features based on their contribution to classification performance, as measured by AUC, accuracy, and F1-score under 10-fold cross-validation.

To identify the optimal feature subset, we evaluated model performance across different feature set sizes. As shown in Figure~\ref{fig:rfe-performance}a, model performance (measured by AUC) steadily increased with the number of selected features, reaching an optimal plateau in the range of 9 to 13 features. This pattern exemplifies Occam's Razor—the principle that simpler models should be preferred when they achieve comparable performance to more complex alternatives.

Based on these results, the top 9 features were initially selected as the optimal subset. However, one of these—Feature 40—was missing in the target domain (Cohort B). To ensure cross-domain consistency, we excluded this feature and retained the remaining top 8 features for all downstream training and evaluation. This feature selection strategy not only preserved predictive accuracy but also enhanced domain robustness by reducing the risk of overfitting to cohort-specific noise.

The 8-features configuration achieved strong performance in both internal and external validation, demonstrating that a compact and interpretable feature set is sufficient for effective pulmonary nodule malignancy prediction. This parsimonious approach contributes to more stable and efficient real-world deployment while maintaining clinical interpretability—a critical requirement for medical AI systems.

\begin{figure}[htbp]
    \centering
    \includegraphics[width=1\linewidth]{feature_performance_comparison_comprehensive.pdf}
    \caption{Comprehensive feature selection and performance analysis using recursive feature elimination (RFE). \textbf{a} Model performance curves showing AUC, accuracy, and F1-score as functions of the number of selected features. Performance reaches an optimal plateau at 9-13 features, consistent with Occam's Razor principle favoring parsimonious models. Shaded areas represent cross-validation variance across 10 folds. \textbf{b} Class-specific accuracy analysis demonstrating model performance for both malignant and benign cases across different feature subset sizes, ensuring balanced predictive capability. \textbf{c} Training time complexity evaluation showing computational efficiency as a function of feature dimensionality, with training time measured in seconds per cross-validation fold. \textbf{d} Performance stability assessment using coefficient of variation across 10-fold cross-validation, indicating the robustness and reliability of feature selection at different subset sizes. Lower values indicate more stable performance. \textbf{e} Cost-effectiveness index combining multiple criteria: Performance×0.45 + Simplicity×0.25 + Efficiency×0.15 + Stability×0.15, providing a comprehensive metric for optimal feature subset selection that balances predictive accuracy with practical deployment considerations.}
    \label{fig:rfe-performance}
\end{figure}


\subsection{Model Reliability and Clinical Utility}

Beyond predictive accuracy, clinical deployment of machine learning models requires reliable probability estimates and demonstrable decision-making value. We therefore conducted a comprehensive multi-dimensional evaluation of our pre-trained tabular foundation model, encompassing receiver operating characteristic (ROC) analysis, calibration assessment, and decision curve analysis (DCA) across both source and target domains.

To assess model discrimination performance, we visualized ROC curves for all baseline and proposed methods on both domains. As shown in Figure~\ref{fig:combined_analysis}a, the source domain ROC curves were computed from 10-fold cross-validation on Cohort A. The pre-trained tabular foundation model achieved the highest AUC of 0.820, clearly outperforming traditional machine learning models such as SVM (AUC = 0.712), XGBoost (AUC = 0.734), and random forest (AUC = 0.742), indicating superior fitting and discriminative capacity. Note that AUC values in ROC curves are computed directly from aggregated predictions across all folds, while performance heatmaps (e.g., Figure~\ref{fig:performance-heatmaps}a) report fold-averaged AUC values, explaining minor numerical differences between visualizations. Figure~\ref{fig:combined_analysis}b shows ROC curves on the target domain (Cohort B) under cross-institutional generalization settings. The TCA-enhanced model maintained the highest AUC (0.709) with curves consistently above all baselines, demonstrating strong robustness against domain shift. In contrast, the non-adaptive version experienced noticeable degradation (AUC = 0.698), reflecting typical domain shift effects. Notably, the Mayo clinical scoring system (AUC = 0.584) showed poor discriminative performance, though slightly better than random chance.

Calibration curves assess the agreement between predicted probabilities and observed event rates, where the dashed diagonal represents perfect calibration. As illustrated in Figure~\ref{fig:combined_analysis}c, in the source domain ou't demonstrates closer alignment to the ideal calibration line than conventional machine learning baselines, indicating more trustworthy risk estimates. Figure~\ref{fig:combined_analysis}d shows that in the target domain, applying TCA-based unsupervised domain adaptation further reduces deviations from perfect calibration, mitigating both systematic underestimation and overestimation across probability bins. This improvement is particularly relevant for risk-stratified decision-making, where accurate probability outputs directly influence clinical thresholds.

Decision curve analysis quantifies the net clinical benefit of different models over a range of decision thresholds. The dashed and dotted black lines correspond to the strategies of treating all patients or treating none, respectively. As shown in Figure~\ref{fig:combined_analysis}e, in the source domain our consistently delivers higher net benefit than baseline models across clinically relevant thresholds. Figure~\ref{fig:combined_analysis}f demonstrates that in the target domain, integrating TCA-based adaptation yields additional gains in net benefit, underscoring the value of domain adaptation for improving decision-making utility in external cohorts.

Overall, the ROC curve comparison provides an intuitive and comprehensive view of model discrimination performance across domains. The calibration analysis validates probability reliability for clinical risk assessment. The decision curve analysis confirms superior clinical utility across relevant threshold ranges. Collectively, this multi-dimensional evaluation demonstrates the robustness and reliability of our TCA-enhanced foundation model approach, validating its effectiveness for cross-institutional deployment and supporting its readiness for real-world clinical applications.

\begin{figure}[htbp]
    \centering
    \includegraphics[width=1\linewidth]{combined_analysis_figure.pdf}
    \caption{\textbf{Comprehensive model performance analysis across source and target domains.} 
    \textbf{a,b} ROC curve comparison for source-domain (10-fold cross-validation on Cohort A) and target-domain (external testing on Cohort B) performance. The TCA-enhanced model consistently achieves the highest AUC across both domains. \textbf{c,d} Model calibration analysis showing agreement between predicted probabilities and observed event rates. The dashed diagonal represents perfect calibration. Our method exhibits improved calibration over conventional baselines in the source domain, with further gains when applying TCA-based UDA in the target domain. \textbf{e,f} Decision curve analysis for clinical utility assessment across a range of decision thresholds. The dashed and dotted black lines correspond to treating all patients or treating none, respectively. Our method consistently outperforms baseline methods in net benefit, with TCA-based UDA further improving decision-making value in the target domain.}
    \label{fig:combined_analysis}
\end{figure}




\section{Methods}

\subsection{Ethics statement and Data Collection}
This study was approved by the Institutional Review Boards of Sun Yat-sen University Cancer Center (Guangzhou, China) and Henan Tumor Hospital (Zhengzhou, China), and conducted in accordance with the ethical principles of the Declaration of Helsinki. All patient data used in this study were retrospectively collected from electronic medical records and fully de-identified prior to analysis. Written informed consent for research use of clinical data was obtained from all patients diagnosed with solitary pulmonary nodules (SPNs) at the time of hospital admission. No identifiable personal information was retained.

The training cohort (Cohort A, $n=295$) was derived from Sun Yat-sen University Cancer Center (Guangzhou, China) between January 2011 and December 2016. The external test cohort (Cohort B, $n=190$) was collected at Henan Tumor Hospital (Zhengzhou, China) from January 2013 to xxxxxx. All patients provided written informed consent for the use of clinical data in scientific research at the time of hospital admission.

Inclusion criteria were as follows: (1) presence of a solitary pulmonary nodule with diameter $\leq 3$ cm identified by chest computed tomography (CT); (2) no evidence of extrapulmonary malignancy; (3) histopathological diagnosis confirmed by surgical resection, CT-guided transthoracic needle biopsy, or bronchoscopy; (4) complete electronic medical records, including clinical, laboratory, and imaging data collected within 7 days prior to anti-tumor treatment. Patients were excluded if they had prior thoracic malignancies, incomplete records, or other comorbidities interfering with diagnostic interpretation.

The collected variables included patient demographics (age, sex, height, weight, body mass index), smoking history, family history of cancer, and symptoms (e.g., fever, cough, hemoptysis, chest pain). Radiologic features of SPNs were recorded, including anatomical location (lung side and lobe), nodule diameter and area, presence of calcification, cavity, spiculation, pleural thickening, and adhesion. Laboratory tests encompassed hematologic and biochemical indices such as white blood cell count (WBC), neutrophil-to-lymphocyte ratio (NLR), platelet-to-lymphocyte ratio (PLR), albumin/globulin ratio (AGR), liver and renal function markers, and tumor biomarkers including CEA, Cyfra21-1, and NSE.

To facilitate reproducibility and verification, all key raw data used in this study have been deposited on the Research Data Deposit public platform (www.researchdata.org.cn) under approval number xxxxxx.


\subsection{Overview of the PANDA Framework}

Cross-institutional deployment of medical AI systems faces three critical challenges that significantly impede clinical adoption: (1) \textit{limited sample sizes} due to the inherent rarity of medical conditions and privacy constraints that restrict data sharing across institutions, (2) \textit{distributional heterogeneity} arising from systematic differences in patient populations, clinical protocols, and measurement equipment across hospitals, and (3) \textit{domain shift} where models trained at one institution often exhibit degraded performance when deployed at different clinical sites due to varying data collection practices and patient demographics. Traditional machine learning approaches struggle with these constraints, particularly in medical screening applications where high sensitivity is paramount and false negatives carry severe clinical consequences.

The PANDA framework addresses these fundamental challenges through a principled integration of pre-trained foundation models and unsupervised domain adaptation, specifically designed for cross-institutional medical AI deployment. As illustrated in Figure~\ref{fig:model_details}a, PANDA operates through a three-stage pipeline: pre-training on synthetic datasets to establish generalizable tabular reasoning capabilities, training with feature selection and model adaptation to optimize performance on limited medical data, and prediction with unsupervised domain adaptation to maintain robust performance across institutional boundaries.

This end-to-end design enables robust cross-institutional generalization by combining the representational power of foundation models with principled domain adaptation. The framework is particularly suited for medical applications where training data is scarce, class distributions are imbalanced, and deployment across different clinical sites is required.

\paragraph{Implementation Overview.} The PANDA framework implementation encompasses three interconnected stages that build upon each other to achieve robust cross-institutional medical AI deployment.

\textbf{Stage 1 (Pre-train)} establishes the foundational capabilities through extensive pre-training on synthetic tabular datasets, as depicted in the lower section of Figure~\ref{fig:model_details}a. The synthetic task generator creates diverse classification problems with varying statistical patterns, feature types, and data distributions, feeding synthetic datasets to train the Pre-trained Tabular Foundation Model that acquires generalizable tabular reasoning capabilities without requiring massive medical datasets.

\textbf{Stage 2 (Train)} involves the medical-specific adaptation illustrated in Figure~\ref{fig:model_details}a. The feature selection process (shown on the left) identifies the most discriminative clinical variables, producing the Development Cohort (Cohort A) with Features + Label. This cohort then undergoes the sophisticated data preprocessing pipeline detailed in Figure~\ref{fig:model_details}b, which includes four parallel branches with feature order rotation, distribution transformations, and categorical encoding strategies, ultimately feeding into the Pre-trained Tabular Foundation Model.

\textbf{Stage 3 (Predict)} addresses the cross-institutional deployment challenge through unsupervised domain adaptation, as shown in Figure~\ref{fig:model_details}a. The UDA process transforms target domain data into the Adapted Cohort (Cohort B' without label), which then follows the identical processing pathway as the training data: through the data preprocessing pipeline (Figure~\ref{fig:model_details}b) and the Pre-trained Tabular Foundation Model, ultimately generating Output Predicted Class Probabilities. This three-stage architecture systematically addresses the challenges of data scarcity, feature discriminativeness, and domain shift inherent in cross-institutional medical AI deployment.

\begin{figure}[htbp]
    \centering
    \includegraphics[width=\linewidth]{Pre-trained Tabular Foundation Mode Pipeline_new.pdf}
    \caption{\textbf{The PANDA framework architecture.}
    (a) Detailed architecture of the pre-trained tabular foundation model showing the complete pipeline from original tabular data through ensemble training, prediction aggregation, class imbalance adjustment, to final classification output.
    (b) Detailed data preprocessing pipeline showing four parallel branches for ensemble diversity. Each branch applies feature order rotation followed by different combinations of distribution transformation (no transformation or quantile transformation) and categorical feature handling (treating as numeric or ordinal encoding). Each branch generates 16 parallel inferences, and all branches are combined through ensemble aggregation to produce the final prediction.}
    \label{fig:model_details}
\end{figure}

\subsection{Data Preprocessing}

Medical AI systems frequently fail when deployed across different hospitals due to subtle but systematic data preprocessing inconsistencies that are often overlooked in single-institution studies. Unlike standardized imaging or genomic data, tabular clinical data exhibits profound institutional variations in feature ordering (demographics-first versus lab-values-first organization), measurement protocols (different equipment calibrations and units), and categorical encoding schemes (institution-specific staging systems and risk classifications). These seemingly minor preprocessing differences can cause substantial performance degradation, with models trained at one institution showing accuracy drops of 10-20\% when deployed elsewhere, severely limiting the real-world impact of medical AI innovations.

Cross-institutional medical datasets present unique preprocessing challenges that demand principled solutions rather than ad-hoc standardization approaches. Traditional preprocessing methods that impose uniform transformations across institutions risk suppressing important clinical variations or introducing systematic biases that favor specific data distributions. The PANDA framework addresses these challenges through a sophisticated preprocessing pipeline designed to maximize ensemble diversity while preserving clinical interpretability and cross-institutional robustness.

The preprocessing strategy tackles three fundamental issues in medical tabular data that are particularly critical for cross-institutional deployment: positional bias from arbitrary feature ordering, distributional heterogeneity across institutions, and inconsistent categorical variable representations. These three issues were prioritized based on empirical analysis of multi-institutional medical datasets, which revealed that (1) Transformer-based models exhibit sensitivity to feature ordering despite theoretical position invariance, (2) clinical measurement protocols and equipment calibrations vary significantly across hospitals, leading to systematic distributional shifts, and (3) categorical medical variables (e.g., staging systems, risk classifications) often employ institution-specific encoding schemes that compromise model transferability.

Rather than applying uniform transformations that may inadvertently suppress important institutional variations or introduce systematic biases favoring specific data distributions, PANDA employs a diversified preprocessing approach that generates multiple complementary data representations. This strategy allows the model to learn robust patterns across different data perspectives while preserving the natural variability that reflects real-world clinical heterogeneity.

This multi-faceted preprocessing pipeline integrates three synergistic components that work collectively to enhance cross-domain robustness. Each component addresses specific aspects of data heterogeneity while contributing to overall ensemble diversity, enabling the model to learn from different perspectives of the same clinical information without losing essential medical semantics.

\subsubsection{Feature Rotation}

Clinical datasets often exhibit arbitrary feature ordering that can introduce systematic biases in Transformer-based models, despite their theoretical position invariance. Different medical institutions may organize clinical variables in varying sequences (e.g., demographics first vs. laboratory values first), creating subtle but persistent ordering patterns that models may inadvertently exploit. This rotation mechanism serves two critical purposes: (1) it eliminates systematic biases that could arise from arbitrary feature ordering in clinical datasets, and (2) it forces the Transformer encoder to learn position-invariant representations, enhancing robustness to feature arrangement variations across different clinical institutions.

To address this challenge, feature rotation introduces positional diversity across ensemble members to counteract ordering biases inherent in the Transformer architecture. Each ensemble member applies a cyclical permutation to input features before processing:

\[
\mathbf{x}^{(k)}_{\text{rotated}} = \text{rotate}(\mathbf{x}, k) = [x_{(k) \bmod d}, x_{(k+1) \bmod d}, \ldots, x_{(k+d-1) \bmod d}]
\]

\noindent
where $k \in [0, K_{\max})$ represents the rotation offset specific to ensemble member $k$, and $d$ is the feature dimensionality. The rotation offset is generated using a deterministic sequence starting from a random seed:

\[
k_i = (\text{start} + i) \bmod K_{\max}, \quad i = 0, 1, \ldots, N-1
\]

\noindent
where $\text{start} \sim \text{Uniform}(0, K_{\max})$ and $N=64$ ensemble members. To ensure diversity, rotation offsets are sampled without replacement to guarantee unique permutations across ensemble members. The rotation ensures consistent and reproducible feature permutations during both training and inference phases.

\subsubsection{Adaptive Feature Transformation}

Medical institutions employ diverse measurement protocols, equipment calibrations, and laboratory standards that result in systematic distributional differences across sites. For instance, blood biomarker values may exhibit institution-specific ranges due to different assay methods, while imaging measurements can vary based on scanner manufacturers and acquisition protocols. These distributional heterogeneities pose significant challenges for cross-institutional model deployment, as models trained on one institution's data distribution may perform poorly when applied to data from institutions with different measurement characteristics.

To address this challenge, the preprocessing pipeline employs a dual-strategy approach to balance distribution normalization with feature preservation, addressing the heterogeneous nature of clinical data. Two complementary transformation strategies are implemented:

\noindent
\textbf{Enhanced Feature Transformation:} Applies quantile transformation followed by dimensionality expansion:

\[
\mathbf{x}_{\text{quantile}} = \text{QuantileTransformer}(\mathbf{x}, n_{\text{quantiles}} = \max(\lfloor n_{\text{samples}}/10 \rfloor, 2))
\]

\noindent
where the quantile transformer maps each feature to a uniform distribution $U(0,1)$ using empirical quantiles. Following quantile transformation, SVD-based dimensionality expansion is applied:

\[
\mathbf{X}_{\text{expanded}} = \text{SVD}(\mathbf{X}_{\text{quantile}}, n_{\text{components}} = \min(4, d))
\]

\noindent
The final feature representation concatenates original and transformed features:

\[
\mathbf{x}_{\text{final}} = [\mathbf{x}_{\text{original}}; \mathbf{x}_{\text{quantile}}; \mathbf{x}_{\text{SVD}}]
\]

\noindent
This increases dimensionality from 7 to 18 features (7 original + 7 quantile + 4 SVD), enhancing the model's representational capacity.

\noindent
\textbf{Preserved Feature Transformation:} Preserves raw feature distributions through identity transformation:

\[
\mathbf{x}_{\text{preserved}} = \mathbf{x}_{\text{original}}
\]

\noindent
This maintains natural scale and distribution characteristics of clinical variables, resulting in unchanged 7-dimensional feature vectors. This configuration ensures that ensemble diversity encompasses both normalized and raw feature representations.

\subsubsection{Intelligent Categorical Encoding}

Categorical variable encoding represents a critical challenge in medical AI systems, where inappropriate encoding strategies can fundamentally distort model learning and lead to spurious clinical predictions. The core issue lies in the fact that naive numerical encoding (0, 1, 2, …) artificially imposes ordinal relationships on categorical variables that may be purely nominal, causing models to learn false mathematical relationships. For instance, if a blood type feature is encoded as A=0, B=1, AB=2, O=3, the model might incorrectly learn that AB is “twice as much” as B, or that there is a meaningful progression from A to O. Such artificial numerical relationships can lead to clinically meaningless predictions and compromise model reliability.

Traditional one-hot encoding, while avoiding artificial ordinality, becomes computationally inefficient and may not capture the full diversity needed for robust ensemble training. Even when features are provided as anonymized identifiers (Feature01, Feature02, etc.), intelligent encoding strategies are essential to prevent the model from learning spurious numerical patterns while maintaining the diversity necessary for robust cross-institutional deployment.

To maximize robustness and ensemble diversity, categorical feature encoding employs two distinct strategies applied across different branches of the preprocessing pipeline:

\noindent
\textbf{Ordinal Encoding with Frequency Filtering:} This strategy applies selective ordinal encoding with frequency-based filtering:

\[
\text{encode}(x_{ij}) = \begin{cases}
\phi_j(x_{ij}) & \text{if feature $j$ has frequently occurring categories} \\
x_{ij} & \text{otherwise}
\end{cases}
\]

\noindent
where $x_{ij}$ represents the categorical value of feature $j$ for sample $i$, and frequently occurring categories are defined as features with individual category counts $\geq 10$ and total unique categories $|U_j| < n_{\text{samples}}/10$. The ordinal mapping employs randomized category-to-integer assignment for ensemble diversity:

\[
\phi_j = \pi(\{0, 1, \ldots, |U_j|-1\})
\]

\noindent
where $\phi_j$ represents the ordinal mapping function for feature $j$, $\pi(\cdot)$ denotes a random permutation operator, and $U_j$ is the set of unique categorical values in feature $j$.

This approach ensures that only sufficiently represented categories undergo ordinal encoding, preventing overfitting to rare categorical values common in medical datasets.

\noindent
\textbf{Numeric Treatment Strategy:} This strategy treats categorical features as continuous numeric values:

\[
\text{encode}(x_{ij}) = \text{float}(x_{ij})
\]

This approach enables direct processing of categorical variables through quantile transformations and other numerical operations, particularly effective for ordinal categorical variables with natural numeric interpretations.

\paragraph{Overall Ensemble Configuration Summary}

The complete 64-member ensemble integrates all preprocessing components through a systematic 4-branch design as illustrated in Figure~\ref{fig:model_details}b:

\begin{itemize}
    \item \textbf{Branch 1} (No Distribution Transformation + Numeric Categorical): 16 ensemble members
    \item \textbf{Branch 2} (No Distribution Transformation + Ordinal Encoding): 16 ensemble members
    \item \textbf{Branch 3} (Quantile Transformation + Numeric Categorical): 16 ensemble members
    \item \textbf{Branch 4} (Quantile Transformation + Ordinal Encoding): 16 ensemble members
\end{itemize}

Within each branch, the 16 ensemble members are differentiated through unique feature rotation patterns, ensuring comprehensive coverage of both transformation strategies and categorical encoding approaches. This systematic design guarantees balanced representation across all preprocessing variations while maximizing ensemble diversity for robust medical predictions across institutional boundaries.


\subsection{Feature Selection}

Robust feature selection is indispensable for cross-institutional medical AI, where the data landscape is typically high-dimensional, heterogeneous, and severely constrained in sample size. Without principled selection, models risk overfitting to site-specific noise or redundant variables, undermining both predictive performance and generalizability. To directly address these challenges, we employed recursive feature elimination (RFE) powered by a Pre-trained Tabular Foundation Model as the base estimator (Figure~\ref{fig:feature_selection_uda}a). This approach enables the identification of clinically meaningful and highly discriminative variables while simultaneously minimizing overfitting risk.

RFE relies on permutation-based feature importance for iterative elimination. Permutation importance measures each feature’s contribution by randomly shuffling its values across samples and observing the resulting performance degradation—larger performance drops indicate more important features. This model-agnostic evaluation is particularly suited to pre-trained foundation models, where gradient-based or weight-based importance scores are not directly interpretable due to multiple attention layers and ensemble components. By directly quantifying each feature’s predictive contribution, the permutation approach offers a transparent and robust mechanism to rank features regardless of architectural complexity.

This wrapper-based strategy iteratively removes the least informative features to produce a compact, domain-consistent subset optimized for downstream classification. As shown in Figure~\ref{fig:feature_selection_uda}a, our pipeline systematically applies RFE to Original Cohort A and Original Cohort B—each comprising 58 clinical variables with labels—yielding refined Development (Cohort A) and Validation (Cohort B) cohorts. This cross-institutional feature selection ensures that the retained features maintain high predictive power and clinical interpretability across different hospital settings.

\begin{figure}[htbp]
    \centering
    \includegraphics[width=\linewidth]{Feature Selection and UDA.pdf}
    \caption{\textbf{Feature Selection and Unsupervised Domain Adaptation in the PANDA framework.} (a) \textbf{Feature Selection}: Recursive Feature Elimination (RFE) with Pre-trained Tabular Foundation Model systematically identifies the most discriminative clinical variables from the original 58-feature sets, reducing dimensionality while preserving predictive power to generate optimized feature subsets for both Development Cohort (Cohort A) and Validation Cohort (Cohort B). (b) \textbf{Unsupervised Domain Adaptation}: Transfer Component Analysis (TCA) performs distributional alignment between source and target domains without using labels, transforming the Validation Cohort (Cohort B) into an Adapted Cohort (Cohort B') that maintains clinical interpretability while being optimally aligned with the source domain distribution for robust cross-institutional prediction.}
    \label{fig:feature_selection_uda}
\end{figure}

\paragraph{Algorithm Overview.}
Given a dataset $\mathcal{D} = \{(\mathbf{x}_i, y_i)\}_{i=1}^n$ with $\mathbf{x}_i \in \mathbb{R}^d$ and $y_i \in \{0, 1\}$, the RFE algorithm selects $k < d$ features by repeating the following steps:
\begin{enumerate}
    \item Train the Pre-trained Tabular Foundation Model $f_\Theta^{(t)}$ on the current feature subset $\mathcal{F}^{(t)}$.
    \item Estimate feature importance scores $\mathbf{I}^{(t)} = [I^{(t)}_1, I^{(t)}_2, \dots, I^{(t)}_{|\mathcal{F}^{(t)}|}]$ using permutation-based evaluation.
    \item Eliminate the feature with the lowest importance score:\\
    $\mathcal{F}^{(t+1)} \leftarrow \mathcal{F}^{(t)} \setminus \{\arg\min_j I^{(t)}_j\}$.
    \item Repeat until the target feature count $|\mathcal{F}^{(t+1)}| = k$ is reached.
\end{enumerate}

\paragraph{Permutation Importance.}
To assess the contribution of each feature, we adopt permutation importance, a robust and model-agnostic metric. For a given feature $x_j$, its importance score is defined as the expected decrease in performance upon random shuffling:

\[
I_j = \frac{1}{R} \sum_{r=1}^{R} \left[ \mathrm{AUC}(f_\Theta, \mathcal{D}) - \mathrm{AUC}(f_\Theta, \mathcal{D}_{\text{perm}(j)}^{(r)}) \right],
\]

\noindent
where $\mathcal{D}_{\text{perm}(j)}^{(r)}$ denotes the dataset with the $j$-th feature randomly permuted in the $r$-th repetition, and $R$ is the number of repeats (we use $R=5$). Larger $I_j$ values indicate stronger influence on model predictions.

\paragraph{Implementation.}
As illustrated in Figure~\ref{fig:feature_selection_uda}a, the feature selection process begins with Original Cohort A and Original Cohort B, both containing the full set of $d=58$ clinical features with their corresponding labels. We applied RFE with the Pre-trained Tabular Foundation Model to both cohorts simultaneously to identify features with consistent discriminative power across institutions. This cross-cohort approach ensures that selected features maintain their predictive value in both source and target domains.

The RFE process systematically eliminated weakly contributing variables using permutation importance evaluation across both cohorts, ultimately yielding a ranked list of the top 9 features. However, one of these features (Feature40) was unavailable in the Original Cohort B, creating an inconsistency for cross-domain deployment. To ensure robust cross-institutional applicability, we excluded Feature40 and retained the remaining 7 features, producing the final Development Cohort (Cohort A) and Validation Cohort (Cohort B) with consistent feature subsets. This cross-validated feature selection strategy ensures that the selected clinical variables maintain high discriminative power across different institutional settings.


\paragraph{Multi-dimensional Performance Analysis.}
Beyond simple accuracy metrics, our RFE analysis incorporates multiple evaluation dimensions to ensure robust feature selection for clinical deployment. The following three analyses provide complementary perspectives on feature subset quality: 
Figure~\ref{fig:rfe-performance}b presents class-specific accuracy analysis across different feature subset sizes. The balanced performance between malignant (positive) and benign (negative) cases demonstrates that our feature selection process maintains diagnostic sensitivity across both clinical scenarios. This balanced predictive capability is crucial for medical screening applications, where both false positives and false negatives carry significant clinical consequences.

\paragraph{Computational Efficiency Assessment.}
Training time complexity is a critical consideration for clinical deployment scalability. Figure~\ref{fig:rfe-performance}c illustrates the computational efficiency as a function of feature dimensionality, measured in seconds per cross-validation fold. The analysis reveals a near-linear relationship between feature count and training time, with the 9-features configuration achieving an optimal balance between computational efficiency and predictive performance. This efficiency profile supports real-time clinical decision-making requirements while maintaining model interpretability.

\paragraph{Performance Stability Evaluation.}
Model reliability in clinical settings requires consistent performance across different data splits and patient populations. Figure~\ref{fig:rfe-performance}d presents performance stability assessment using coefficient of variation (CV) across 10-fold cross-validation. Lower CV values indicate more stable and reliable performance, with our selected 9-features subset demonstrating superior stability compared to both smaller and larger feature configurations. This stability analysis ensures that the selected features provide robust predictions across diverse clinical scenarios and patient demographics.

\paragraph{Multi-criteria Optimization Framework.}
Feature selection in medical applications requires balancing multiple competing objectives beyond simple predictive accuracy. Clinical deployment demands consideration of computational efficiency for real-time decision-making, performance stability across diverse patient populations, and model simplicity for clinical interpretability and regulatory compliance. Rather than optimizing for a single metric, which may lead to suboptimal solutions that excel in one dimension while failing in others, we developed a principled multi-criteria optimization approach.

To identify the globally optimal feature subset, we developed a comprehensive cost-effectiveness index that integrates multiple performance dimensions (Figure~\ref{fig:rfe-performance}e). Let $n$ denote the number of features in a given subset. The composite metric is defined as a weighted combination of four normalized scores:

\[
\text{CostEffectiveness}(n) = w_1 \cdot S_{\text{perf}}(n) + w_2 \cdot S_{\text{eff}}(n) + w_3 \cdot S_{\text{stab}}(n) + w_4 \cdot S_{\text{simp}}(n)
\]

\noindent
where $(w_1, w_2, w_3, w_4) = (0.45, 0.15, 0.15, 0.25)$ represent theory-driven weights prioritizing performance for medical applications, with balanced consideration of stability, efficiency, and Occam's Razor simplicity. Each component score is normalized to $[0,1]$ as follows:

\begin{itemize}
    \item \textbf{Performance Score} ($S_{\text{perf}}$): A weighted combination of classification metrics:
    \[
    S_{\text{perf}}(n) = 0.5 \cdot \text{AUC}(n) + 0.3 \cdot \text{Accuracy}(n) + 0.2 \cdot \text{F1}(n)
    \]

    \item \textbf{Efficiency Score} ($S_{\text{eff}}$): Training time efficiency, normalized using min-max scaling:
    \[
    S_{\text{eff}}(n) = 1 - \frac{T(n) - T_{\min}}{T_{\max} - T_{\min}}
    \]
    where $T(n)$ is the mean training time for $n$ features, and shorter training times yield higher scores.

    \item \textbf{Stability Score} ($S_{\text{stab}}$): Cross-validation consistency based on performance variance:
    \[
    S_{\text{stab}}(n) = 1 - \frac{\bar{\sigma}(n) - \bar{\sigma}_{\min}}{\bar{\sigma}_{\max} - \bar{\sigma}_{\min}}
    \]
    where $\bar{\sigma}(n) = \frac{1}{3}[\sigma_{\text{AUC}}(n) + \sigma_{\text{Accuracy}}(n) + \sigma_{\text{F1}}(n)]$ is the average standard deviation across metrics.

    \item \textbf{Simplicity Score} ($S_{\text{simp}}$): Model complexity penalty following Occam's Razor principle, implemented as an exponential decay function that naturally favors simpler models:
    \[
    S_{\text{simp}}(n) = \exp(-\alpha \cdot n)
    \]
    where $\alpha = 0.015$ is the complexity penalty coefficient. This pure mathematical function ensures that fewer features always receive higher scores, strictly adhering to Occam's Razor without arbitrary cutoffs or baseline assumptions.
\end{itemize}

The optimization problem becomes:
\[
n^* = \arg\max_{n} \text{CostEffectiveness}(n)
\]

This multi-criteria optimization framework objectively determines that the 9-features configuration ($n^* = 9$) achieves the globally optimal trade-off across all evaluation dimensions. The theory-driven weight allocation prioritizes performance (0.45) and simplicity (0.25) as the primary concerns in medical applications, while maintaining balanced consideration of stability (0.15) and efficiency (0.15). The exponential decay simplicity function $\exp(-0.015 \cdot n)$ ensures pure adherence to Occam's Razor without arbitrary assumptions, naturally favoring simpler models while allowing empirical performance metrics to determine the optimal balance point. This methodology is both theoretically sound and practically deployable in real-world clinical environments, as evidenced by the peak in Figure~\ref{fig:rfe-performance}e at precisely 9 features.

\paragraph{Cross-Domain Feature Consistency.}
The cross-domain feature alignment process is detailed in Figure~\ref{fig:feature_selection_uda}a. Starting with Original Cohort A and Original Cohort B, each containing the full set of 58 clinical features, recursive feature elimination with the Pre-trained Tabular Foundation Model identifies the most discriminative variables across both institutions. To ensure robust cross-institutional deployment, we account for varying feature availability across different clinical sites by applying RFE simultaneously to both cohorts. This cross-validated approach produces the Development Cohort (Cohort A) and Validation Cohort (Cohort B) with consistent feature subsets that are available and discriminative in both institutions, guaranteeing reliable model deployment while preserving the most clinically relevant variables for malignancy prediction.





\subsection{Pre-trained Tabular Foundation Model}

Medical tabular classification faces unique challenges that distinguish it from traditional machine learning applications. Medical datasets typically contain heterogeneous feature types (continuous measurements, categorical variables, ordinal scales) with complex non-linear interactions that are difficult to capture using conventional algorithms. Furthermore, small sample sizes relative to feature dimensionality create high variance in model predictions, while institutional differences in data collection protocols lead to distribution shifts that compromise generalization. Traditional machine learning approaches often fail to adequately model these complex feature interactions while maintaining robustness across different clinical environments.

Foundation models have revolutionized artificial intelligence by demonstrating remarkable capabilities in learning generalizable representations from large-scale data that can be adapted to diverse downstream tasks. These models, exemplified by large language models in NLP and vision transformers in computer vision, leverage massive pre-training on heterogeneous datasets to acquire broad knowledge that facilitates few-shot learning and cross-domain transfer. The success of foundation models stems from their ability to learn universal patterns and relationships during pre-training that generalize beyond specific tasks or domains. This paradigm shift from task-specific model training to pre-training followed by adaptation offers particular promise for medical applications, where data scarcity and domain heterogeneity are prevalent challenges.

To address these challenges, we adopt and adapt TabPFN (Tabular Prior-Fitted Networks), a pre-trained Transformer-based foundation model for tabular data~\cite{hollmann2025accurate}. While TabPFN's core architecture treats each feature as a token and uses self-attention mechanisms for feature interaction modeling, our contribution lies in developing specialized preprocessing configurations and ensemble strategies specifically optimized for cross-institutional medical classification tasks. We enhance the original TabPFN framework with domain-adaptive preprocessing pipelines and systematic ensemble diversification strategies that address the unique challenges of medical tabular data across different clinical environments.

\subsubsection{Architecture and Feature Encoding}

\paragraph{Per-Feature Transformer Architecture.}
Each structured input sample $\mathbf{x} = [x_1, x_2, \ldots, x_d] \in \mathbb{R}^d$ is treated as a sequence of tokens, where each feature $x_i$ corresponds to an individual token. This per-feature tokenization enables the model to learn feature-specific representations while modeling inter-feature dependencies through attention mechanisms.

The feature encoding process follows a multi-step transformation:
\[
\mathbf{e}_i = \text{Embed}(x_i) + \mathbf{p}_i, \quad i = 1, \ldots, d
\]
where $\text{Embed}(\cdot): \mathbb{R} \rightarrow \mathbb{R}^{d_{\text{model}}}$ maps each feature value to a $d_{\text{model}}$-dimensional embedding space (where $d_{\text{model}} \in \{128, 192\}$ depending on model configuration), and $\mathbf{p}_i \in \mathbb{R}^{d_{\text{model}}}$ represents learned positional encodings that capture feature ordering information.

The embedded sequence $\mathbf{E} = [\mathbf{e}_1, \mathbf{e}_2, \ldots, \mathbf{e}_d]$ is processed through a 12-layer Transformer encoder. Each layer $\ell$ applies multi-head self-attention followed by a feedforward network:
\[
\mathbf{H}^{(\ell)} = \text{LayerNorm}(\text{MultiHead}(\mathbf{H}^{(\ell-1)}) + \mathbf{H}^{(\ell-1)})
\]
\[
\mathbf{H}^{(\ell+1)} = \text{LayerNorm}(\text{FFN}(\mathbf{H}^{(\ell)}) + \mathbf{H}^{(\ell)})
\]
where $\mathbf{H}^{(0)} = \mathbf{E}$, and the feedforward network $\text{FFN}$ has $4 \times d_{\text{model}}$ hidden units. The multi-head attention mechanism employs 4 or 6 attention heads (depending on configuration), enabling the model to capture diverse feature interaction patterns simultaneously.

\subsubsection{Pre-training}

\paragraph{Motivation for Pre-training.}
Medical tabular datasets present fundamental challenges that necessitate pre-training approaches. First, medical institutions typically possess small, specialized datasets that are insufficient for training robust deep learning models from scratch. The limited sample sizes (often hundreds to thousands of cases) combined with high-dimensional feature spaces create severe overfitting risks when using conventional supervised learning. Second, medical datasets exhibit significant heterogeneity across institutions due to differences in measurement protocols, equipment calibration, and patient populations, leading to substantial domain shift that compromises model generalizability. Third, unlike vision or language domains where massive public datasets exist, medical data sharing is severely constrained by privacy regulations and institutional policies, preventing the assembly of large-scale training corpora.

These challenges necessitate models that can: (1) extract generalizable patterns from diverse statistical distributions and feature-label relationships across multiple domains, (2) adapt to new medical classification tasks through in-context learning without requiring parameter updates or fine-tuning, and (3) maintain robustness against distributional shifts by understanding diverse data-generating processes inherent in clinical environments.

\paragraph{Synthetic Task Generation Strategy.}
Deep neural networks require substantial training data to learn effective representations, yet collecting and labeling extensive medical datasets is prohibitively expensive, time-intensive, and often infeasible due to privacy constraints. To address this data scarcity while ensuring broad coverage of tabular reasoning patterns, we employ a stochastic task generator that synthesizes classification problems from diverse function priors.

This synthetic data generation strategy enables the model to experience a vast range of tabular data characteristics, statistical patterns, and classification scenarios that would be impossible to encounter through real medical datasets alone. The task generator samples problems from a mixture of function priors, creating diverse synthetic classification tasks that collectively teach the model generalizable tabular reasoning capabilities applicable to real-world medical classification problems with varying sample sizes and domain characteristics.

Let $\mathcal{X}\subset\mathbb{R}^d$ and $\mathcal{Y}=\{1,\dots,C\}$ for classification (or $\mathbb{R}$ for regression). For each training batch, we first sample a prior family and its hyperparameters:
\[
r \sim \mathrm{Categorical}(\boldsymbol{\pi}),\qquad
\boldsymbol{\theta} \sim p(\boldsymbol{\theta}\mid r),
\]
where $r\in\{\text{gp},\text{mlp},\text{ridge},\text{mix\_gp}\}$ indexes different function priors including Gaussian Process (GP), Multi-Layer Perceptron (MLP), ridge regression, and mixed GP priors, $\boldsymbol{\pi}$ represents probability weights for prior family selection, and $\boldsymbol{\theta}$ is a hyperparameter vector containing kernel/architecture specifications, noise level, class count, and input scaling parameters.

We draw $T=n_{\text{ctx}}+n_{\text{eval}}$ inputs where $n_{\text{ctx}}$ is the number of context (training) examples per synthetic task, $n_{\text{eval}}$ is the number of evaluation (test) examples per synthetic task, and $T$ is the total sequence length. Inputs are sampled independently from a factorized base distribution and optionally transformed:
\[
\mathbf{x}_t \sim p_{\text{base}}(\mathbf{x}), \qquad \tilde{\mathbf{x}}_t = \psi_{\boldsymbol{\theta}}(\mathbf{x}_t)
\]
where $p_{\text{base}}(\mathbf{x})$ is a factorized base distribution (often uniform or Gaussian per feature) and $\psi_{\boldsymbol{\theta}}(\cdot)$ is a feature transform function (e.g., quantile-to-normal normalization, standardization).
\[
\mathbf{x}_t \sim p_X(\cdot\mid \boldsymbol{\theta}_X)=\prod_{j=1}^d p_{X_j}(x_{t,j}), 
\qquad 
\tilde{\mathbf{x}}_t=\psi_{\boldsymbol{\theta}}(\mathbf{x}_t), 
\quad t=1,\dots,T.
\]
Conditioned on $(r,\boldsymbol{\theta})$, a random function $f_\Theta$ generates latent outputs. For a GP prior ($r=\text{gp}$), each class logit is an independent draw from a GP with zero mean and kernel $k_{\boldsymbol{\theta}}$ (e.g., RBF with length-scale $\ell$ and output scale $\sigma_f^2$):
\[
f_c \sim \mathrm{GP}\!\big(0,k_{\boldsymbol{\theta}}\big),\qquad 
\mathbf{z}_t=\big(f_1(\tilde{\mathbf{x}}_t),\dots,f_C(\tilde{\mathbf{x}}_t)\big)\in\mathbb{R}^C .
\]
For a random–MLP prior ($r=\text{mlp}$), we sample depth/width and weights,
\[
L\sim p(L),\ \ h_\ell\sim p(h_\ell),\ \ 
\mathbf{W}_\ell \sim \mathcal{N}\!\Big(0,\frac{\sigma_w^2}{\mathrm{fan\_in}_\ell}\mathbf{I}\Big),\ \ 
\mathbf{b}_\ell \sim \mathcal{N}(0,\sigma_b^2\mathbf{I}),
\]
and define
\[
\mathbf{h}_0=\tilde{\mathbf{x}}_t,\quad 
\mathbf{h}_\ell=\phi\!\big(\mathbf{W}_\ell\mathbf{h}_{\ell-1}+\mathbf{b}_\ell\big)\ (\ell=1,\dots,L-1),\quad
\mathbf{z}_t=\mathbf{W}_L\mathbf{h}_{L-1}+\mathbf{b}_L,
\]
with nonlinearity $\phi$ (e.g., ReLU). For ridge regression priors ($r=\text{ridge}$), linear models with L2 regularization are employed to generate smooth, generalizable functions. For mixed GP priors ($r=\text{mix\_gp}$), multiple Gaussian Process kernels are combined to capture diverse statistical relationships and function characteristics across different length scales and patterns. Observation noise models variability:
\[
\boldsymbol{\varepsilon}_t \sim \mathcal{N}\big(\mathbf{0},\sigma^2\mathbf{I}\big).
\]
The model output dimensionality $n_{\text{out}}$ is determined by the task type:
\[
n_{\text{out}} = \begin{cases}
2 & \text{for regression with uncertainty (GaussianNLLLoss: mean and variance)} \\
C & \text{for } C\text{-class classification (CrossEntropyLoss)} \\
1 & \text{for binary classification (BCEWithLogitsLoss) or basic regression}
\end{cases}
\]
For regression, $y_t=z_t+\varepsilon_t\in\mathbb{R}$; optionally a bucketized ("bar") likelihood is used by choosing bin borders $b_0<\cdots<b_B$ (e.g., prior-predictive quantiles) and training a categorical density over bins. For classification, temperature-scaled logits yield class probabilities and labels,
\[
\mathbf{p}_t=\mathrm{softmax}\!\big(\mathbf{z}_t/\tau\big),\qquad 
y_t \sim \mathrm{Categorical}(\mathbf{p}_t),
\]
with an optional random class permutation to decorrelate semantic labels across tasks. The first $n_{\text{ctx}}$ pairs form the context $\mathcal{D}_{\text{ctx}}=\{(\tilde{\mathbf{x}}_t,y_t)\}_{t=1}^{n_{\text{ctx}}}$; the remaining inputs $\mathcal{Q}=\{\tilde{\mathbf{x}}_t\}_{t=n_{\text{ctx}}+1}^{T}$ are queries whose labels are withheld during the forward pass. We concatenate $(\mathcal{D}_{\text{ctx}},\mathcal{Q})$ into a single sequence for in-context conditioning and train the Transformer to predict $\{y_t\}_{t=n_{\text{ctx}}+1}^{T}$. The task distribution and objective are
\[
p(\mathcal{T})
=\sum_{r}\pi_r\!\int p(\boldsymbol{\theta}\mid r)
\prod_{t=1}^{T}\! \Big[p_X(\mathbf{x}_t\mid\boldsymbol{\theta})\, p\!\big(y_t\mid \tilde{\mathbf{x}}_t,\boldsymbol{\theta},r\big)\Big]\,
\mathrm{d}\boldsymbol{\theta},
\qquad
\min_{\Theta}\ \mathbb{E}_{\mathcal{T}\sim p(\cdot)}\!\left[\sum_{t=n_{\text{ctx}}+1}^{T} 
\ell\!\big(h_{\Theta}(\mathcal{D}_{\text{ctx}},\mathcal{Q})_t,\ y_t\big)\right],
\]
where $\ell$ is the task-specific loss function determined by task type and output configuration:
\begin{align}
\ell = \begin{cases}
\text{BCEWithLogitsLoss}(\mathbf{z}_t, y_t) & \text{for binary classification} \\
\text{CrossEntropyLoss}(\mathbf{z}_t, y_t) & \text{for multi-class classification} \\
\text{MSELoss}(z_t, y_t) & \text{for basic regression} \\
\text{GaussianNLLLoss}(\mu_t, y_t, |\sigma_t|) & \text{for regression with uncertainty, where } \\
& \quad \mu_t = \mathbf{z}_t[0], \sigma_t = \mathbf{z}_t[1] \\
\text{FullSupportBarDistribution}(\mathbf{z}_t, y_t) & \text{for discretized regression}
\end{cases}
\end{align}

\paragraph{Pre-training Details.}
The foundation model pre-training follows a systematic procedure designed to optimize learning across diverse synthetic tabular tasks. Pre-training is conducted using AdamW optimizer with learning rate determined by the OpenAI scaling law: $\text{lr} = 0.003 \cdot \sqrt{d_{\text{model}}/512}$, where $d_{\text{model}}$ is the model embedding dimension. The learning rate schedule employs cosine annealing with linear warmup over the first 50 epochs, followed by cosine decay over the remaining pre-training duration.

Each pre-training epoch processes multiple synthetic tasks in parallel with batch size of 1000 sequences. Gradient accumulation is employed over multiple batches before parameter updates, with gradient clipping at norm 1.0 to ensure pre-training stability. Mixed precision training using automatic mixed precision (AMP) is utilized to accelerate computation and reduce memory requirements while maintaining numerical stability.

The pre-training objective employs positional loss computation, where losses are calculated for each position in the sequence and averaged across valid positions. For in-context learning scenarios, only the query positions (beyond $n_{\text{ctx}}$) contribute to the loss calculation:
\[
\mathcal{L}_{\text{epoch}} = \frac{1}{N} \sum_{i=1}^{N} \frac{1}{n_{\text{eval}}} \sum_{t=n_{\text{ctx}}+1}^{T} \ell\!\big(h_{\Theta}(\mathcal{D}_{\text{ctx}}^{(i)},\mathcal{Q}^{(i)})_t,\ y_t^{(i)}\big)
\]
where $N$ is the batch size, and the inner sum averages over evaluation positions. Pre-training continues until convergence, typically requiring 200-500 epochs depending on model size and task complexity.

\subsubsection{Inference}

\paragraph{In-Context Learning Mechanism.}
Real-world cross-institutional deployment faces three compounding issues: \emph{small labeled cohorts} that make fine-tuning prone to overfitting, \emph{heterogeneous feature distributions} across hospitals (protocols, equipment, populations), and \emph{domain shift} that would otherwise require site-specific retraining with substantial computational and operational cost. To address these challenges without updating parameters at deployment, we adopt an \emph{in-context learning} (ICL)–based inference design. Our Pre-trained Tabular Foundation Model uses ICL as the primary mechanism: the model internalizes task-specific patterns by observing a small set of labeled examples (context) within the input sequence and then predicts on unlabeled query samples. Compared with conventional fine-tuning, this approach (1) \textbf{improves sample efficiency under data scarcity} by leveraging pre-trained knowledge to generalize from minimal examples; (2) \textbf{facilitates cross-institutional generalization} by adapting to local characteristics without separate training per site; and (3) \textbf{reduces computational burden} by avoiding gradient computation and parameter updates during inference, enabling rapid and resource-efficient clinical deployment.

At inference time, training and test samples are concatenated to form a composite input sequence:
\[
\mathbf{X}_{\text{context}} = [\mathbf{X}_{\text{train}}; \mathbf{X}_{\text{test}}] \in \mathbb{R}^{(n_{\text{train}} + n_{\text{test}}) \times 1 \times f}
\]
where $n_{\text{train}}$ is the number of training samples (e.g., 295 for medical dataset A), $n_{\text{test}}$ is the number of test samples (e.g., 190 for medical dataset B), and $f$ is the number of features (e.g., 7 features for the \texttt{best7} configuration). The middle dimension of 1 indicates that each sample is processed as an independent batch for sequential processing, rather than processing multiple samples simultaneously in a traditional training batch. For multi-batch scenarios, this extends to $(B, n_{\text{train}} + n_{\text{test}}, f)$ where $B$ represents the number of parallel inference tasks.

Only training labels are provided during forward pass through label masking:
\[
\mathbf{y}_{\text{context}} = [\mathbf{y}_{\text{train}}; \varnothing] \in \{0,1\}^{n_{\text{train}}} \cup \{\varnothing\}^{n_{\text{test}}}
\]
where $\varnothing$ represents masked positions that the Transformer attention mechanism ignores during forward propagation. Specifically, the model applies causal masking to these positions, preventing gradient flow from test sample predictions back to model parameters. This masking enables the model to infer predictions for test samples without gradient updates, improving generalizability while maintaining computational efficiency.

\paragraph{Train.}
\textbf{Train} involves the medical-specific adaptation illustrated in Figure~\ref{fig:model_details}a. The feature selection process identifies the most discriminative clinical variables, producing the \textbf{Development Cohort (Cohort A)} with Features + Label. This cohort then undergoes the sophisticated data preprocessing pipeline detailed in Figure~\ref{fig:model_details}b, which includes four parallel branches with \emph{feature order rotation}, \emph{distribution transformations} (no transform vs.\ quantile transform), and \emph{categorical encoding strategies} (treat-as-numeric vs.\ ordinal encoding), ultimately feeding the frozen \textbf{Pre-trained Tabular Foundation Model} in an in-context manner (no parameter updates).

\paragraph{Predict.}
\textbf{Predict} addresses cross-institutional deployment through unsupervised domain adaptation, as shown in Figure~\ref{fig:model_details}a. The \textbf{UDA} process transforms target-domain data into the \textbf{Adapted Cohort (Cohort B' without label)}, which then follows the \emph{identical} processing pathway as the training data—namely, the data preprocessing pipeline (Figure~\ref{fig:model_details}b) and the \textbf{Pre-trained Tabular Foundation Model}—ultimately generating \textbf{Output Predicted Class Probabilities}. This Train→Predict design systematically addresses data scarcity, feature discriminativeness, and domain shift in cross-institutional medical AI.

\paragraph{Ensemble Inference and Diversity Strategies.}
To enhance robustness and mitigate overfitting to single parametric priors while emulating cross-site heterogeneity, the system employs a \textbf{64-member ensemble} realized by the four-branch pipeline in Figure~\ref{fig:model_details}b. Each branch applies feature order rotation followed by distinct combinations of distribution transformation and categorical encoding, creating four complementary representations that maximize ensemble diversity and improve generalization under domain shift. Each ensemble member outputs class logits, which are temperature-scaled and softmax-normalized; final predictions are obtained by averaging member probabilities:
\[
\hat{y} = \frac{1}{N} \sum_{i=1}^{N} \text{softmax}\!\left( \frac{z_i}{T} \right),
\]
where $z_i \in \mathbb{R}^{n_{\text{test}} \times C}$ denotes the logits from the $i$-th member, $T{=}0.9$ is the softmax temperature, and $N{=}64$ the number of members. Probability averaging (rather than logit averaging) preserves proper probabilistic interpretation. The four branches are:
\begin{itemize}
    \item \textbf{Branch 1}: Feature order rotation + no distribution transformation + treat categorical features as numeric (16 parallel inferences)
    \item \textbf{Branch 2}: Feature order rotation + no distribution transformation + ordinal encoding of categorical features (16 parallel inferences)
    \item \textbf{Branch 3}: Feature order rotation + quantile transformation + treat categorical features as numeric (16 parallel inferences)
    \item \textbf{Branch 4}: Feature order rotation + quantile transformation + ordinal encoding of categorical features (16 parallel inferences)
\end{itemize}
All 64 inferences (16 per branch) are aggregated to produce robust final predictions while maintaining computational efficiency.


\paragraph{Class Imbalance Correction.}
To address the intrinsic label imbalance in real-world medical datasets, inverse-frequency reweighting is applied during output aggregation when the \texttt{balance\_probabilities} flag is enabled. Given predicted class probabilities $\mathbf{p} = (p_1, \dots, p_C)$ and empirical class distribution from training data $\boldsymbol{\pi} = (\pi_1, \dots, \pi_C)$ where $\pi_i = \frac{\text{count}_i}{\sum_{j=1}^C \text{count}_j}$, the reweighted probabilities are:
\[
\hat{p}_i = \frac{p_i / \pi_i}{\sum_{j=1}^{C} p_j / \pi_j}, \quad \text{for } i = 1, \dots, C,
\]
where $\pi_i$ represents the observed frequency of class $i$ in the training dataset. This method ensures that minority class predictions are not diluted by the class imbalance, which is particularly critical in small-sample medical datasets where rare conditions must maintain adequate sensitivity for clinical screening applications.

\subsection{Transfer Component Analysis for Domain Adaptation}

Cross-institutional deployment of medical AI inevitably encounters distributional shifts arising from differences in patient demographics, measurement protocols, and institutional practices. Without explicit alignment, models trained on one hospital’s data often exhibit degraded performance when applied to another, even when the underlying clinical concepts remain stable. To mitigate this critical barrier to generalization, we incorporated Transfer Component Analysis (TCA), a kernel-based unsupervised domain adaptation method, into our preprocessing pipeline (Figure~\ref{fig:feature_selection_uda}b).

TCA projects both the Development Cohort (Cohort A without label) and the Validation Cohort (Cohort B without label) into a shared latent subspace in which their marginal distributions are statistically aligned. This process produces the Adapted Cohort (Cohort B' without label), which preserves essential clinical characteristics while reducing institution-specific biases and improving cross-domain compatibility. By learning a distribution-invariant representation without relying on labels, TCA provides a principled way to leverage all available data for model adaptation, a particularly valuable property for medical datasets where labeled data is scarce.

This unsupervised alignment directly addresses one of the most persistent challenges in medical AI—distribution drift across hospitals—ensuring that the predictive model can generalize robustly while maintaining clinical interpretability and consistency across heterogeneous healthcare settings.

\paragraph{Objective and Kernel Construction.}
Let $X_s \in \mathbb{R}^{n \times d}$ and $X_t \in \mathbb{R}^{m \times d}$ denote the source and target domain feature matrices, with $n + m$ total samples and $d$-dimensional features. TCA constructs a combined kernel matrix $K \in \mathbb{R}^{(n+m) \times (n+m)}$ using a linear kernel:
\[
K(x_i, x_j) = x_i^\top x_j.
\]
The composite kernel $K$ is partitioned as:
\[
K = 
\begin{bmatrix}
K_{ss} & K_{st} \\
K_{ts} & K_{tt}
\end{bmatrix},
\]
where $K_{ss} = X_s X_s^\top$, $K_{tt} = X_t X_t^\top$, and $K_{st} = X_s X_t^\top$. A projection matrix $W \in \mathbb{R}^{(n+m) \times k}$ is then learned by solving:
\[
\min_W \; \mathrm{tr}(W^\top K L K^\top W) + \mu \cdot \mathrm{tr}(W^\top K H K^\top W),
\]
where $L$ is a domain alignment matrix based on maximum mean discrepancy (MMD), $H$ is a centering matrix, and $\mu > 0$ is a regularization coefficient.

The alignment matrix $L$ is constructed as:
\[
L = 
\begin{bmatrix}
\frac{1}{n^2} \mathbf{1}_{n \times n} & -\frac{1}{nm} \mathbf{1}_{n \times m} \\
-\frac{1}{nm} \mathbf{1}_{m \times n} & \frac{1}{m^2} \mathbf{1}_{m \times m}
\end{bmatrix},
\]
which encourages samples from different domains to align while preserving within-domain relationships. The centering matrix is defined as:
\[
H = I - \frac{1}{n+m} \mathbf{1} \mathbf{1}^\top,
\]
ensuring that the projected features are zero-centered in the kernel space.

\paragraph{Projection and Prediction.}
The optimization problem is solved via eigen-decomposition. Specifically, the generalized eigensystem:
\[
(I + \mu K L K) S = K H K S
\]
is decomposed into eigenvectors $S = U \Lambda U^\top$, and the top-$k$ eigenvectors are used to construct the projection matrix $W = U_{[:,1:k]}$.

Source and target samples are then projected via:
\[
Z_s = K_s W, \quad Z_t = K_t W,
\]
where $K_s$ and $K_t$ are kernel submatrices corresponding to $X_s$ and $X_t$. A logistic regression classifier is trained on $(Z_s, y_s)$ and applied to $Z_t$ for prediction. This subspace alignment significantly improves performance under covariate shift, particularly in cross-institutional settings where the label space remains shared but marginal distributions differ.



\subsection{Evaluation Metrics}

To comprehensively evaluate the effectiveness of the proposed Pre-trained Tabular Foundation Model with TCA-based domain adaptation, we constructed a multidimensional assessment framework covering classification metrics, statistical confidence intervals, visualization-based analysis, and domain discrepancy measures.

\subsubsection*{1. Classification Performance Metrics}

Five widely used metrics were adopted to assess classification accuracy: AUC, accuracy, F1 score, sensitivity, and specificity. All results were averaged over 10-fold stratified cross-validation to ensure robustness against label imbalance. Let $TP$, $TN$, $FP$, and $FN$ denote the number of true positives, true negatives, false positives, and false negatives, respectively. The metrics are defined as:

\[
\begin{aligned}
&\text{True Positive Rate:} && TPR(\tau) = \frac{TP(\tau)}{TP(\tau) + FN(\tau)} \\
&\text{False Positive Rate:} && FPR(\tau) = \frac{FP(\tau)}{FP(\tau) + TN(\tau)} \\
&\text{AUC:} && AUC = \int_0^1 TPR(\tau)\, d(FPR(\tau)) \\
&\text{Accuracy:} && \frac{TP + TN}{TP + TN + FP + FN} \\
&\text{Precision:} && \frac{TP}{TP + FP} \\
&\text{Recall:} && \frac{TP}{TP + FN} \\
&\text{F1 Score:} && \frac{2 \cdot \text{Precision} \cdot \text{Recall}}{\text{Precision} + \text{Recall}} = \frac{2TP}{2TP + FP + FN} \\
&\text{Specificity:} && \frac{TN}{TN + FP}
\end{aligned}
\]

Let $\mathcal{D} = \{(\mathbf{x}_i, y_i)\}_{i=1}^n$ denote the full dataset, and $\mathcal{D}_k$ be the $k$-th fold. For metric $M$, the mean and standard deviation over $K=10$ folds are:

\[
\bar{M} = \frac{1}{K}\sum_{k=1}^K M_k, \quad \sigma_M = \sqrt{\frac{1}{K-1} \sum_{k=1}^K (M_k - \bar{M})^2}
\]

\subsubsection*{2. Confidence Interval Estimation}

To quantify uncertainty in AUC, we employed non-parametric bootstrap resampling ($B=1000$). For each bootstrap iteration $b$, a dataset $\mathcal{D}_b^*$ was sampled with replacement:

\[
\mathcal{D}_b^* = \{(\mathbf{x}_{i_j}^*, y_{i_j}^*)\}_{j=1}^n, \quad i_j \sim \text{Uniform}(1, \dots, n)
\]

The model was retrained on each $\mathcal{D}_b^*$ to compute $\text{AUC}_b^*$, forming an empirical distribution $\{\text{AUC}_1^*, \dots, \text{AUC}_B^*\}$. The 95\% confidence interval was then defined as:

\[
CI_{95\%} = \left[ Q_{2.5\%},\, Q_{97.5\%} \right]
\quad \text{where} \quad Q_\alpha := \text{quantile at } \alpha
\]

\subsubsection*{3. Visualization-Based Evaluation}

\begin{itemize}
    \item \textbf{ROC Curves:} Plot $TPR(\tau)$ versus $FPR(\tau)$ for $\tau \in [0,1]$ to visualize sensitivity-specificity trade-off. An ideal curve approaches the point $(0,1)$, while a random model lies along the diagonal $TPR = FPR$.

    \item \textbf{Calibration Curves:} Assess the agreement between predicted probability $\hat{p}_i$ and observed frequency $y_i$. For $K$ equal-width bins $B_k = [k/K, (k+1)/K)$:

    \[
    \bar{p}_k = \frac{1}{|B_k|} \sum_{i \in B_k} \hat{p}_i, \quad \bar{y}_k = \frac{1}{|B_k|} \sum_{i \in B_k} y_i
    \]

    \item \textbf{Decision Curve Analysis (DCA):} Evaluate net benefit $NB(p_t)$ under clinical cost-benefit assumptions:

    \[
    NB(p_t) = \frac{TP(p_t)}{n} - \frac{FP(p_t)}{n} \cdot \frac{p_t}{1 - p_t}
    \]

    With benchmark strategies:
    \[
    NB_{all}(p_t) = \text{Prevalence} - (1 - \text{Prevalence}) \cdot \frac{p_t}{1 - p_t}, \quad NB_{none} = 0
    \]
    where $\text{Prevalence} = \frac{1}{n} \sum_{i=1}^n y_i$
\end{itemize}

\subsubsection*{4. Domain Adaptation Evaluation}

To assess the effectiveness of TCA in aligning source and target distributions, we used both qualitative and quantitative tools:

\paragraph{(a) Dimensionality Reduction.}
PCA and t-SNE were applied to visualize domain overlap before and after adaptation.

\paragraph{(b) Normalized Domain Distance Metrics.}
Let $\mu(\cdot)$ and $\sigma(\cdot)$ denote feature-wise mean and std. The standardized features are:

\[
\hat{\mu} = \frac{\mu(\mathbf{X}_s) + \mu(\mathbf{X}_t)}{2}, \quad
\hat{\sigma} = \frac{\sigma(\mathbf{X}_s) + \sigma(\mathbf{X}_t)}{2}
\]
\[
\mathbf{X}_s^{\text{norm}} = \frac{\mathbf{X}_s - \hat{\mu}}{\hat{\sigma}}, \quad \mathbf{X}_t^{\text{norm}} = \frac{\mathbf{X}_t - \hat{\mu}}{\hat{\sigma}}
\]

Then, compute the following:

\begin{itemize}
    \item \textbf{Wasserstein Distance:}
    \[
    W_{\text{norm}}(\mathbf{X}_s, \mathbf{X}_t) = \frac{1}{d} \sum_{i=1}^d W_1(X_{s,i}^{\text{norm}}, X_{t,i}^{\text{norm}})
    \]

    \item \textbf{Symmetric KL Divergence:}
    \[
    KL_{\text{norm}}(\mathbf{X}_s, \mathbf{X}_t) = \frac{1}{d} \sum_{i=1}^d \frac{KL(P_{s,i}^{\text{norm}} || P_{t,i}^{\text{norm}}) + KL(P_{t,i}^{\text{norm}} || P_{s,i}^{\text{norm}})}{2}
    \]

    \item \textbf{MMD with RBF Kernel:}
    \[
    \text{MMD}^2(\mathbf{X}_s, \mathbf{X}_t) =
    \frac{1}{n_s(n_s-1)} \sum_{i \neq j} k(x_i^s, x_j^s)
    + \frac{1}{n_t(n_t-1)} \sum_{i \neq j} k(x_i^t, x_j^t)
    - \frac{2}{n_s n_t} \sum_{i,j} k(x_i^s, x_j^t)
    \]
    where $k(\mathbf{x}, \mathbf{y}) = \exp(-\gamma ||\mathbf{x} - \mathbf{y}||^2)$
\end{itemize}

Notably, while TCA uses a linear kernel for domain projection, RBF-kernel MMD offers a nonlinear complementary perspective.

\vspace{1em}
This comprehensive evaluation protocol enables rigorous, multidimensional validation of the proposed domain-adaptive foundation model across both predictive accuracy and cross-domain generalization.



\subsection{Baseline Methods}

To evaluate the effectiveness of the proposed Pre-trained Tabular Foundation Model with TCA-based domain adaptation, we designed a series of comparative experiments involving the following baseline groups:

\paragraph{(1) Foundation Model without Domain Adaptation.}  
This baseline directly applied the Pre-trained Tabular Foundation Model trained on the source domain to the target domain, without any domain alignment. It served as a foundation-only benchmark to isolate the contribution of domain adaptation.

\paragraph{(2) Conventional Clinical Risk Models.}  
We included several widely used rule-based clinical scoring systems for comparison, including the PKUPH model, the Mayo Clinic score, and a previously published logistic regression model (Paper\_LR). These methods reflect existing clinical heuristics based on handcrafted variables and were implemented following their original published formulas. Since these models do not involve data-driven training, their generalization relies solely on the stability of clinical rule transfer.

\paragraph{(3) Classical Machine Learning Algorithms.}  
To assess the generalization of standard supervised learners, we implemented several representative classifiers:
\begin{itemize}
    \item \textbf{Support Vector Machine (SVM)} using an RBF kernel, with hyperparameters tuned via grid search on the source domain.
    \item \textbf{Decision Tree (DT)} and \textbf{Random Forest (RF)} models using Gini impurity for splitting and evaluated under varying maximum depths and tree counts.
    \item \textbf{Gradient Boosted Decision Tree (GBDT)} and \textbf{XGBoost}, optimized with respect to learning rate, number of estimators, maximum tree depth, and subsample ratio. All models were trained on the source domain and tested directly on the target domain.
\end{itemize}

\paragraph{(4) Proposed Domain Adaptation Method: Pre-trained Tabular Foundation Model + TCA.}  
Our proposed method integrates the Pre-trained Tabular Foundation Model with Transfer Component Analysis (TCA) to mitigate domain shift. Specifically, TCA projects both source and target data into a shared latent space, after which predictions are made using the ensemble inference structure of the pre-trained model. This approach enables cross-domain generalization by aligning marginal distributions while retaining the rich semantic representations learned during pretraining.

\paragraph{Hyperparameter Tuning and Evaluation Protocol.}  
For all trainable baselines, hyperparameters were optimized using 10-fold stratified cross-validation within the source domain to avoid data leakage. Final model performance was assessed on the target domain using the same metrics and evaluation procedures as applied to our proposed method.

This comprehensive set of baselines enables a rigorous comparative analysis, revealing the relative contributions of domain alignment and model architecture to cross-institutional performance in medical tabular data.


\subsection{Computational resource}

\subsection{Reporting summary}


Predicting Malignancy of Pulmonary Nodules Across Hospitals Using Transferable Machine Learning on Small and Imbalanced Datasets
PANDA: A Robust Tool for Predicting Malignant Pulmonary Nodules Using Transferable Machine Learning Approach with Clinical, Radiological, Hematological Laboratory, and Pulmonary Function Multimodal Data
PANDA: A Reliable Machine Learning Tool for Predicting Malignant Pulmonary Nodules Using Clinical, Radiological, Hematological Laboratory, and Pulmonary Function Multimodal Data.
An Artificial Intelligence-Based Model for Assessing Malignancy of Solitary Pulmonary Nodules Incorporating Clinical, Radiological, Hematological Laboratory, and Pulmonary Function Multimodal Data
PANDA: A Robust Transferable Machine Learning Tool for Malignant Pulmonary Nodule Prediction Using Clinical, Radiological, Hematological, and Pulmonary Function Multimodal Data

Hao Chen 1\#, Qingyuan Liu2\#, Ning Xue3\#,      Shulin Chen1*, Wenqi Fan2,*
1.Department of Clinical Laboratory, State Key Laboratory of Oncology in South China, Collaborative Innovation Center for Cancer Medicine, Guangdong Provincial Clinical Research Center for Cancer, Sun Yat-sen University Cancer Center, Guangzhou, Guangdong, China
2. Department of Computing, The Hong Kong Polytechnic University, Hong Kong SAR, China
3.Department of Clinical Laboratory, Affiliated Cancer Hospital of Zhengzhou University, Zhengzhou Key Laboratory of Digestive Tumor Markers, Zhengzhou, Henan, People’s Republic of China

\# These authors made equal contributions to this work.
* Corresponding author.

Abstract
Medical AI systems face persistent challenges due to limited data, class imbalance, and inter-institutional heterogeneity. In particular, small sample sizes and skewed class distributions are common in clinical datasets, often introducing biases and undermining model reliability. Furthermore, models trained in one hospital or cohort frequently underperform when deployed at a different site, as differences in equipment, protocols, and patient demographics lead to domain shift. To address these issues, we propose a novel cross-domain adaptation framework that integrates a pre-trained Transformer-based foundation model with a classic domain alignment technique, Transfer Component Analysis (TCA). This approach leverages the rich representational power of Transformer models (pre-trained on large tabular data) and the distribution-matching capability of TCA to improve generalization under small-sample conditions. We validated our method on structured clinical datasets from two independent hospitals (295 patients in Hospital A for training, 190 patients in Hospital B for testing), focusing on malignant vs. benign pulmonary nodule classification. Our model achieved high area under the ROC curve (AUC) along with improved sensitivity and specificity on the external test set, outperforming several conventional domain adaptation baselines. These results demonstrate that the proposed framework achieves robust cross-domain prediction by aligning feature distributions and learning generalizable classification functions over structured clinical data. Interpretation: By effectively combating data scarcity and shift, the model exhibits strong generalizability and shows potential for reliable deployment in multi-center medical settings.


Introduction
Lung cancer is the most common and deadliest cancer in China. (PMID: 39654104) Early and accurate identification of benign versus malignant pulmonary nodules is vital but challenging in clinical radiology. To enhance diagnosis, predictive models like the Mayo Clinic (MC), Department of Veterans Affairs (VA), Peking University People’s Hospital (PKUPH), Shanghai, and Bayesian Inference Malignancy Calculator (BIMC) models have been developed, using clinical, radiological, or Hematological data. Despite this, their predictive performance, such as the MC model's AUC of only 0.67, (PMID: 39069970) and their generalization across different datasets, particularly internationally, require improvement. (PMID: 39705824, PMID: 34364866)
CT, pulmonary function, and Hematological Laboratory tests are typically part of routine physical exams but are seldom used together to assess pulmonary nodules. In particular, a broad spectrum of laboratory tests (such as albumin concentration and platelet-to-lymphocyte ratio), have been validated to be relevant to diagnostic and prognostic significance in cancer. These biomarkers are advantageous due to their low cost, easy accessibility, high consistency, and wide applicability in primary healthcare. Previously, we developed a predictive model for identifying MSPNs in patients with sPNs, incorporating clinical, CT, and laboratory data (including blood tests and pulmonary function tests). Our model (AUC=0.718) outperformed the PKUPH (AUC=0.674), Shanghai (AUC=0.632), and Mayo (AUC=0.562) models. However, our previous model only relied on traditional LASSO logistic regression for indicator selection and model development. In contrast, artificial intelligence (AI) offers significant advantages in integrating diverse test data, particularly laboratory results, to enhance clinical diagnosis and accurately characterize disease features.
 Meanwhile, implementing AI models in various clinical centers is highly challenging due to the diverse and limited nature of real-world medical data, which significantly affects the models' performance and fairness. In practical applications, different hospital cohorts may adopt unique clinical instruments, and the patient populations they serve also differ. Consequently, a model trained on one institution's data may show a significantly higher error rate when used at another institution. Unlike natural image tasks with millions of samples, clinical model datasets are usually small and often have significant class imbalances, such as fewer malignant cases compared to benign ones. In summary, data heterogeneity, small sample size, and imbalanced class distribution form a "triple barrier" that impedes the generalization of medical AI models and can lead to diagnostic errors and inconsistent performance across patient subgroups.\cite{guan2021domain}\cite{hellin2024unraveling}.
To mitigate data limitations, researchers are addressing data limitations by using pre-trained foundation models and transfer learning. Transformer-based architectures, initially successful in NLP and vision, are now being applied to tabular clinical data. These models, pre-trained on large structured datasets, excel at extracting features for various tasks. Notably, the TabPFN model has surpassed traditional methods on small datasets, achieving high accuracy quickly. This highlights Transformers' potential in small medical datasets by utilizing prior knowledge. Additionally, domain adaptation techniques like Transfer Component Analysis (TCA) help align data distributions between training and application domains. \cite{pan2010domain}. TCA operates in a reproducing kernel Hilbert space to find latent features that minimize domain divergence, enabling reliable model performance on target data after re-training. \cite{pan2010domain} This method has been effectively used in medical studies for domain adaptation, such as aligning mammography image features to improve breast cancer classification across databases.\cite{guan2021domain} Inspired by these successes, we propose integrating Transformer-based tabular modeling with TCA for enhanced representation learning and cross-domain generalization.
Pre-trained tabular Transformers offer a strong baseline, but fine-tuning them on a small clinical dataset often fails to ensure good performance in a different domain due to domain shift. Large models may fit source hospital data well but struggle with different target hospital data distributions\cite{guan2021domain}. Unsupervised Domain Adaptation (UDA) methods, which don't need target labels, can help but often become unstable with limited data. When target data is extremely scarce, aligning distributions is statistically challenging and prone to overfitting\cite{guan2021domain}. Recent studies indicate that a one-step adaptation may not work well in such cases; an intermediate fine-tuning on a related large dataset can enhance stability. Additionally, severe domain shift and class imbalance in medical data can cause models to memorize majority class features while ignoring minority patterns, leading to unstable training and misleading accuracy. This is problematic in clinical settings, as it can result in poor sensitivity for rare classes, like missing malignant nodules. Our goal is to create a framework that improves out-of-domain robustness, stabilizes adaptation on small samples, and reduces bias from imbalanced data. This involves integrating foundational model knowledge with a domain adaptation mechanism resilient to limited data and skewed distributions, ensuring reliable predictive performance across all classes and hospital settings.
This study presents PANDA (Pretrained Adaptation Network with Domain Alignment), a domain-adaptive framework aimed at improving malignant solitary pulmonary nodules (MSPN) prediction by utilizing clinical, radiological, laboratory, and pulmonary function data from two hospitals. PANDA integrates a pre-trained Transformer model with Transfer Component Analysis to tackle feature representation and distribution shifts in small-sample scenarios. It excels in cross-institutional pulmonary nodule cohorts, demonstrating improved generalization with limited data. PANDA is benchmarked against traditional machine learning models and clinical scoring systems, showcasing its stability and practical utility in real-world multi-center applications.

Methods
Study design and participants
This retrospective multicenter study included a training cohort of 295 patients with SPNs recruited from Sun Yat-sen University Cancer Center (Guangzhou, China) between January 2011 and December 2016, and an external validation cohort of 190 patients with SPNs recruited from Henan Tumor Hospital (Zhengzhou, China) between January 2013 and June 2018.The inclusion criteria were as follows: 1.SPN Detection: All patients were identified through chest CT scans showing solitary pulmonary nodules (SPNs), with final diagnoses confirmed via histopathologic examination of tissue obtained through CT-guided transthoracic needle biopsy, bronchoscopy, thoracoscopy, or surgical resection. 2. Nodule Size: Solitary pulmonary nodule lesions $\\leq$3 cm in diameter.3.Exclusion of Extrapulmonary Malignancy: No evidence of extrapulmonary malignancy. 4.Data Completeness: Availability of complete clinical, CT imaging, and laboratory data, collected from electronic medical records within 14 days prior to any anti-tumor treatment initiation.

Data preprocessing
A total of 62 universally applicable features from routine clinical practice were gathered, encompassing four categories: clinical features (age, sex, body mass index, smoking history, family history of cancer, Past History of Tumor, and symptoms (fever, cough, hemoptysis, sputum, chest pain)), radiologic characteristics (anatomical location, nodule diameter and area, presence of calcification, cavity, spiculation, pleural thickening, and adhesion.), lung function parameters (Vital Capacity (VC),	Forced Expiratory Volume in 1 Second(FEV1), Forced Expiratory Volume in 1 Second Percentage Predicted (FEV1\%), Ratio of Forced Expiratory Volume in 1 Second to Forced Vital Capacity (FEV1/FVC), Ratio of Residual Volume to Total Lung Capacity (RV/TLC), Diffusing Capacity of the Lung for Carbon Monoxide (DLCO1), Diffusing Capacity of the Lung for Carbon Monoxide Percentage Predicted (DLCO\%)), and blood-based biomarkers (white blood cell count (WBC), neutrophil-to-lymphocyte ratio (NLR), platelet-to-lymphocyte ratio (PLR), albumin/globulin ratio (AGR), liver and renal function markers, and tumor biomarkers including CEA, Cyfra21-1, and NSE).
Raw clinical data from both development (Cohort A) and validation (Cohort B) cohorts undergo systematic curation. Records containing missing values are excluded to ensure data quality, followed by recursive feature elimination to identify the most discriminative variables.

PANDA prediction model
As illustrated in Figure 1, PANDA operates through a three-stage pipeline: data preprocessing, training on source domain data, and domain-adaptive testing on target domain data.
Data Preprocessing Stage (Figure 1a): Raw clinical data from both development (Cohort A) and validation (Cohort B) cohorts undergo systematic curation. Records containing missing values are excluded to ensure data quality, followed by recursive feature elimination to identify the most discriminative variables. This preprocessing yields structured tabular datasets optimized for cross-domain learning while maintaining clinical interpretability.
Training Phase (Figure 1b): The curated source domain data (Cohort A) is fed into a pretrained tabular foundation model that leverages large-scale pretraining on diverse tabular datasets.
The model employs a 32 member ensemble architecture where each variant applies unique data trans-formation pipelines including feature permutation, quantile normalization, and categorical encoding strategies to enhance robustness and mitigate overfitting. The ensemble outputs classification probabilities through weighted aggregation, with inverse-frequency reweighting to address class imbalance inherent in medical datasets.
Testing Phase (Figure 1c): To handle distribution mismatch between institutions, PANDA incorporates Transfer Component Analysis (TCA) for unsupervised domain adaptation. TCA learns a shared latent representation where source and target domain distributions are statistically aligned using maximum mean discrepancy minimization. The domain-aligned features from both cohorts are then processed through the same pre-trained foundation model to generate predictions on the target domain without requiring labeled data from the validation cohort.

Feature Selection
To identify the most discriminative variables while mitigating the risk of overfitting, we employed recursive feature elimination (RFE) with the Pre-trained Tabular Foundation Model serving as the base estimator. As a wrapper-based feature selection strategy, RFE iteratively removes the least important features, ultimately generating a compact feature subset. This subset is not only optimized for downstream classification tasks but also ensures cross-domain consistency (Figure 6a). We initiated the process using the original Cohort A, which contained 63 features, and applied RFE to identify the most discriminative variables within this cohort. However, for ensuring robust cross-institutional model deployment, it is necessary to address the variability in feature availability across different clinical sites. As demonstrated in the "Select Overlapping Features" step, we intersected the top-ranked features derived from Cohort A with the available feature set of Cohort B. This intersection step ensured that only features consistently available across both institutions were retained for the final modeling process.

Statistical analysis
To assess the generalization of standard supervised learners, we implemented several representative classifiers: Support Vector Machine (SVM) using an RBF kernel, with hyperparameters tuned via grid search on the source domain. Decision Tree (DT) and Random Forest (RF) models using Gini impurity for splitting and evaluated under varying maximum depths and tree counts. Gradient Boosted Decision Tree (GBDT) and XGBoost, optimized with respect to learning rate, number of estimators, maximum tree depth, and subsample ratio. All models were trained on the source domain and tested directly on the target domain. For the conventional clinical risk models included in the comparative analysis, we incorporated several widely used clinical prediction models, namely the PKUPH model, the Mayo Clinic model, and Paper LR model (a previously published logistic regression model).
To comprehensively evaluate the performance of our proposed framework, we constructed a multidimensional evaluation protocol encompassing classification metrics, statistical confidence estimation, visualization-based analysis, and domain alignment assessment. 
Classification performance was assessed using five standard metrics widely adopted in clinical machine learning: area under the receiver operating characteristic curve (AUC), accuracy, F1-score, sensitivity (recall), and specificity (precision). All performance indicators were averaged over 10-fold cross-validation to ensure robustness and reproducibility. To quantify uncertainty in performance estimates, 95\% confidence intervals (CIs) for AUC were calculated via 1000-round bootstrap resampling. This provides a statistical measure of model stability and offers meaningful interpretability for clinical application. We also employed visualization-based assessments, including ROC curves, calibration plots, and decision curve analysis (DCA), to examine model reliability and potential clinical benefit under varying decision thresholds. 

Ethical statement
This study was approved by the ethics committees of two hospitals and followed the Declaration of Helsinki of 1975. All case data were anonymised, and the Institutional Review Board waived the requirement for written informed consent. The key raw data supporting the findings of this study have been deposited on the Research Data Deposit public platform (www.researchdata.org.cn) under approval number xxxxxxx.

Acknowledgments
This work is financially supported by Guang Dong Basic and Applied Basic Research Foundation (2024A1515011958),

Declaration of competing interest
The authors declare no conflict of interest.

Data availability
Data will be made available on request.

Role of the funding source
The funders of the study had no role in the study design, data collection, data analysis, data interpretation, or writing of the report.

Results
Between January 2011 and June 2018, a total of 485 individuals who met the inclusion criteria were enrolled in this study. Among them, cohort A (which served as the training set) comprised 295 participants, including 189 in the MSNP group and 106 in the pulmonary benign nodule control group. Cohort B, designated as the external validation set, consisted of 190 participants, with 125 in the MSNP group and 65 in the pulmonary benign nodule control group. In the training stage, Cohort A first undergoes feature selection to identify a stable and domain-relevant subset, followed by imputation, categorical processing, and class imbalance adjustment. Cohort A contained 63 structured features covering demographics, vital signs, biochemistry, tumor markers, and imaging descriptors. To ensure domain consistency, we applied recursive feature elimination (RFE) on the source cohort and selected the top 9 predictive features. One feature (CYSC) was unavailable in Cohort B and excluded from final modeling. The remaining 8 features were used consistently across both domains and formed the basis for domain-adaptive learning. The selected features include anatomical location, pulmonary function measurements, and key serum biomarkers such as CEA, CRE, and NSE, all of which are clinically relevant for distinguishing malignant lesions. The processed data is then passed into a pre-trained tabular foundation model configured as a 32-member ensemble. Each ensemble member applies distinct transformation pipelines to promote diversity and enhance structural robustness. 

The PANDA framework addresses key challenges in real-world medical tabular data, including small sample sizes, heterogeneous features, and domain shift. As shown in Fig. 1, it operates in two stages: model training on a labeled source domain and testing on an unlabeled target domain after domain alignment. To mitigate distribution mismatch, PANDA incorporates TCA during testing, projecting source and target samples into a shared latent space with better-aligned marginal distributions. Aligned target data undergoes the same ensemble processing, with final predictions obtained by aggregating weighted outputs across all members to address class imbalance.

We evaluated TCA's effectiveness in reducing domain shift using qualitative and quantitative analyses. Four standard distributional discrepancy metrics showed consistent post-adaptation improvements (all significant), confirming effective alignment. After TCA transformation, source-target normalized linear discrepancy decreased by 0.070 (indicating improved latent space alignment), normalized Frechét distance by 0.018, Wasserstein distance by 0.006, and symmetric KL divergence by 0.022. These results consistently demonstrate TCA's ability to mitigate inter-domain statistical divergence across multiple metrics.

Model generalization was assessed on Cohort A (source) and Cohort B (target; Fig. B) using AUC, accuracy, F1-score, precision, and recall. On Cohort A, our pre-trained tabular foundation model outperformed all comparators (AUC 0.826, accuracy 0.743, F1-score 0.807, precision 0.786, recall 0.841). Traditional models showed moderate performance (Random Forest AUC 0.752, XGBoost AUC 0.742), while GBDT/SVM/DT had lower metrics (limiting utility for high-dimensional data) and clinical scoring systems performed poorest (MC model: no predictive capacity with F1-score/precision/recall 0.000; PKUPH model: consistent underperformance). On Cohort B, our TCA-enhanced model achieved the highest external validation performance (AUC 0.709, F1-score 0.811, recall 0.944), significantly outperforming its non-adaptive version (AUC 0.698).

Model discrimination, calibration, and clinical utility were also evaluated. For discrimination, ROC analysis showed that our pre-trained model had the highest  AUC (0.820), outperforming SVM (0.717), XGBoost (0.734) and random forest (0.752) in Cohort A. The TCA-enhanced version maintained the top target-domain AUC (0.709), while its non-adaptive counterpart declined (AUC 0.698), confirming domain shift. For calibration (Fig. 5), our model aligned closer to the ideal probability-event rate line than traditional models on the source domain; TCA-based UDA further improved target-domain calibration by reducing over/underestimation. For clinical value (DCA, Fig. 5), our model delivered higher net benefits than baselines on the source domain, with TCA adaptation further boosting target-domain net benefits—underscoring its importance for cross-domain use.

Discussion

Currently, low-dose computed tomography (LDCT) is the only globally recognized screening method proven to reduce lung cancer mortality in high-risk groups. A pulmonary nodule is a small ($<$3 cm), well-defined radiographic opacity surrounded by normal lung tissue, seen in chest CT scans. The main objectives in managing these nodules are to minimize diagnostic-related harm while quickly assessing and treating potential lung cancer. The malignancy risk in small pulmonary nodules (sPNs) depends on size, smoking history, and other clinical factors, with imaging features being crucial. Although CT parameters help assess lung cancer invasiveness, measuring values in sPNs is difficult due to their irregular shape, risking inaccuracies. Recent models for predicting the nature of pulmonary nodules using imaging or biological characteristics, like ctDNA with imaging, DNA methylation, tsRNA, and Imaging omics, have limitations due to their reliance on single features, leading to high false-positive rates. (PMID: 39363284) A comprehensive approach considering factors like gender, smoking history, underlying diseases, tumor history, comorbidities, lab parameters, and CT imaging is necessary. In this multicenter retrospective study, we identified 8 features including anatomical location, pulmonary function, and serum biomarkers (CEA, CRE, NSE) from 63 clinical, radiological, laboratory, and pulmonary function metrics to develop the PANDA (or CRLP) model for accurately identifying patients with MSPN. As recommended by the American Thoracic Society (ATS), a biomarker for early lung cancer detection “should be used in clinical practice only if it reliably adds to a clinician’s judgment, resulting in a more favorable clinical outcome for the target population." (PMID: 28960111) Aligning with this guideline, all indicators incorporated into our model possess key attributes that support practical clinical application: they are not only readily accessible in routine clinical settings but also classified as mandatory items in regular physical examinations. Importantly, this design eliminates the need for additional testing or data collection, thus imposing no incremental economic burden on patients—addressing a critical barrier to the real-world translation of predictive models.
To our knowledge, this is the first study that leverages a comprehensive dataset—encompassing clinical, radiological, laboratory, and pulmonary function data—and integrates a pretrained Transformer model with Transfer Component Analysis (TCA), applying this combined framework as a stratification tool for clinical data in a real-world patient population.
Building multi-center medical models is challenging due to varied clinical features, equipment differences, and patient diversity, along with the imbalance in medical datasets where pathological cases are less common. This feature mismatch hinders the development of models that perform consistently across different clinical settings and are suitable for large-scale use. Current AI approaches in medicine usually focus on either improving model architecture or addressing domain shifts, but not both. Additionally, these methods often depend on large datasets, which is impractical in many medical situations where data is inherently limited.
To tackle challenges in cross-institutional medical AI, we propose PANDA, a framework with five innovations. Firstly, we create a recursive feature elimination strategy for multi-institutional contexts, reducing 63 clinical variables to 8 while maintaining predictive performance and clinical interpretability. Secondly, we apply pretrained transformer architectures to medical prediction, using a 32-member ensemble with domain-specific adaptations, enhancing generalization for small, imbalanced datasets. Thirdly, we integrate TCA with foundation model representations for domain adaptation, aligning statistical distributions across institutions while preserving discriminative information.
The PANDA framework overcomes key limitations in current medical AI by enabling effective cross-institutional deployment with a minimal, informative feature set. Achieving a high sensitivity of 94.4\%, it is well-suited for medical screenings where missing positive cases is critical. Using only 8 selected features, PANDA supports clinical integration and cost-efficiency, reducing reliance on extensive lab tests or specialized biomarkers. This approach promotes equitable AI-assisted diagnosis across various healthcare environments, from advanced academic centers to resource-limited community hospitals. The integration of foundation model capabilities with domain adaptation principles establishes a new paradigm for medical AI development, where models can leverage large-scale pretraining benefits while adapting to the specific challenges of medical deployment scenarios. This approach has broad implications beyond pulmonary nodule classification, potentially transforming how AI systems are developed and deployed across various medical specialties and clinical applications.
To handle distribution mismatch between institutions, PANDA incorporates Transfer Component Analysis (TCA) for unsupervised domain adaptation. TCA learns a shared latent representation where source and target domain distributions are statistically aligned using maximum mean discrepancy minimization. The domain-aligned features from both cohorts are then processed through the same pre-trained foundation model to generate predictions on the target domain without requiring labeled data from the validation cohort.
This end-to-end design enables robust cross-institutional generalization by combining the representational power of foundation models with principled domain adaptation. The framework is particularly suited for medical applications where training data is scarce, class distributions are imbalanced, and deployment across different clinical sites is required.
To evaluate our framework, we compared the TCA-integrated model against classical machine learning algorithms and clinical scoring systems. The results showed that our model consistently outperformed these baselines in key metrics, demonstrating strong external generalization. This confirms the effectiveness of TCA for domain alignment and highlights the model's potential for reliable use in diverse clinical settings.




\input{Section/Data availability}

\input{Section/Code availability}

\bibliographystyle{unsrt}       % ✅ Use numerical references only — conforms to: 
                                % “Please use numerical references only for citations”

\bibliography{refs}             % ⚠️ For writing convenience only.
                                % Before submission, you must:
                                % 1. Run BibTeX to generate a `.bbl` file.
                                % 2. Copy the contents of the `.bbl` file into this .tex file.
                                % 3. Delete both `\bibliography{}` and `\bibliographystyle{}`.
                                % → As per: “If you wish to use BibTeX, please copy the reference list from the .bbl file...”


\end{document}
