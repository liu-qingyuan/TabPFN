\section{Introduction}

Lung cancer is the most common and deadliest cancer in China. (PMID: 39654104) Early and accurate identification of benign versus malignant pulmonary nodules is vital but challenging in clinical radiology. 
To enhance diagnosis, predictive models like the Mayo Clinic (MC), Department of Veterans Affairs (VA), Peking University People's Hospital (PKUPH), Shanghai, and Bayesian Inference Malignancy Calculator (BIMC) models have been developed, using clinical, radiological, or serum data. 
Despite this, their predictive performance, such as the MC model's AUC of only 0.67, (PMID: 39069970) and their generalization across different datasets, particularly internationally, require improvement. (PMID: 39705824, PMID: 34364866)

CT, pulmonary function tests, and blood tests are typically part of routine physical exams but are seldom used together to assess pulmonary nodules. 
In particular, a broad spectrum of laboratory tests, such as albumin concentration and platelet-to-lymphocyte ratio, have been validated to be relevant to diagnostic and prognostic significance in cancer. 
These biomarkers are advantageous due to their low cost, easy accessibility, high consistency, and wide applicability in primary healthcare. 
Previously, we developed a predictive model for identifying MSPNs in patients with sPNs, incorporating clinical, CT, and laboratory data (including blood tests and pulmonary function tests). 
Our model (AUC=0.718) outperformed the PKUPH (AUC=0.674), Shanghai (AUC=0.632), and Mayo (AUC=0.562) models. 
However, our previous model only relied on traditional LASSO logistic regression for indicator selection and model development. 
In contrast, artificial intelligence (AI) offers significant advantages in integrating diverse test data, particularly laboratory results, to enhance clinical diagnosis and accurately characterize disease features.

 Meanwhile, implementing AI models in various clinical centers is highly challenging due to the diverse and limited nature of real-world medical data, which significantly affects the models' performance and fairness. In practical applications, different hospital cohorts may adopt unique clinical instruments, and the patient populations they serve also differ. Consequently, a model trained on one institution's data may show a significantly higher error rate when used at another institution. Unlike natural image tasks with millions of samples, clinical model datasets are usually small and often have significant class imbalances, such as fewer malignant cases compared to benign ones. In summary, data heterogeneity, small sample size, and imbalanced class distribution form a "triple barrier" that impedes the generalization of medical AI models and can lead to diagnostic errors and inconsistent performance across patient subgroups\cite{guan2021domain}\cite{hellin2024unraveling}.
 
To mitigate data limitations, researchers are addressing data limitations by using pre-trained foundation models and transfer learning. Transformer-based architectures, initially successful in NLP and vision, are now being applied to tabular clinical data. These models, pre-trained on large structured datasets, excel at extracting features for various tasks. Notably, the TabPFN model has surpassed traditional methods on small datasets, achieving high accuracy quickly. This highlights Transformers' potential in small medical datasets by utilizing prior knowledge. Additionally, domain adaptation techniques like Transfer Component Analysis (TCA) help align data distributions between training and application domains. \cite{pan2010domain}. TCA operates in a reproducing kernel Hilbert space to find latent features that minimize domain divergence, enabling reliable model performance on target data after re-training. \cite{pan2010domain} This method has been effectively used in medical studies for domain adaptation, such as aligning mammography image features to improve breast cancer classification across databases.\cite{guan2021domain} Inspired by these successes, we propose integrating Transformer-based tabular modeling with TCA for enhanced representation learning and cross-domain generalization.

Pre-trained tabular Transformers offer a strong baseline, but fine-tuning them on a small clinical dataset often fails to ensure good performance in a different domain due to domain shift. Large models may fit source hospital data well but struggle with different target hospital data distributions\cite{guan2021domain}. Unsupervised Domain Adaptation (UDA) methods, which don't need target labels, can help but often become unstable with limited data. When target data is extremely scarce, aligning distributions is statistically challenging and prone to overfitting\cite{guan2021domain}. Recent studies indicate that a one-step adaptation may not work well in such cases; an intermediate fine-tuning on a related large dataset can enhance stability. Additionally, severe domain shift and class imbalance in medical data can cause models to memorize majority class features while ignoring minority patterns, leading to unstable training and misleading accuracy. This is problematic in clinical settings, as it can result in poor sensitivity for rare classes, like missing malignant nodules. Our goal is to create a framework that improves out-of-domain robustness, stabilizes adaptation on small samples, and reduces bias from imbalanced data. This involves integrating foundational model knowledge with a domain adaptation mechanism resilient to limited data and skewed distributions, ensuring reliable predictive performance across all classes and hospital settings.

This study presents PANDA (Pretrained Adaptation Network with Domain Alignment), a domain-adaptive framework aimed at improving malignant solitary pulmonary nodules (MSPN) prediction by utilizing clinical, radiological, laboratory, and pulmonary function data from two hospitals. PANDA integrates a pre-trained Transformer model with Transfer Component Analysis to tackle feature representation and distribution shifts in small-sample scenarios. It excels in cross-institutional pulmonary nodule cohorts, demonstrating improved generalization with limited data. PANDA is benchmarked against traditional machine learning models and clinical scoring systems, showcasing its stability and practical utility in real-world multi-center applications.



\subsection{Background and Motivation}
Medical artificial intelligence systems face persistent challenges in real-world deployment, particularly due to limited training data, class imbalance, and inter-institutional heterogeneity. Small sample sizes and skewed class distributions are endemic in clinical datasets, often introducing biases that undermine model reliability and generalization. Moreover, models trained at one institution frequently exhibit degraded performance when deployed at different clinical sites, as variations in equipment, protocols, patient demographics, and clinical practices lead to domain shift—a phenomenon where the statistical properties of data differ between training and deployment environments.

The challenge of cross-institutional generalization is particularly acute in medical imaging and structured clinical data analysis, where domain-specific factors can significantly impact model performance. Traditional machine learning approaches often fail to account for these distributional differences, resulting in models that perform well in validation studies but struggle in real-world clinical deployment across multiple sites. This limitation is especially pronounced in oncological applications, where accurate diagnosis and treatment planning depend critically on consistent model performance across diverse healthcare settings.

Recent advances in foundation models for tabular data offer promising solutions to these challenges. Pre-trained transformers, originally developed for natural language processing, have demonstrated remarkable success when adapted for structured tabular data, showing superior performance on small datasets compared to traditional machine learning methods. However, the direct application of these models to cross-institutional medical scenarios requires careful consideration of domain adaptation techniques to address distributional shift while preserving the rich representational capabilities of foundation models.

\subsection{Problem Statement}
In the context of pulmonary nodule classification, this challenge manifests as the critical need to develop models that can accurately distinguish between malignant and benign nodules across different hospitals and healthcare systems. Each institution may have different CT scanning protocols, patient populations, clinical workflows, and available clinical variables, creating subtle but significant variations in the resulting clinical data. Furthermore, the inherent class imbalance in medical datasets—where pathological cases are typically less frequent than normal cases—compounds the difficulty of achieving robust cross-institutional performance.

The complexity is further amplified by the heterogeneity of clinical features available across institutions. While one hospital may have comprehensive laboratory panels and advanced imaging biomarkers, another may rely on a more limited set of clinical variables. This feature availability mismatch presents a fundamental challenge for developing generalizable models that can maintain performance across diverse clinical environments while remaining practical for widespread deployment.

Existing approaches to cross-institutional medical AI typically focus on either improving model architectures or addressing domain shift, but rarely tackle both challenges simultaneously in an integrated framework. Moreover, most current methods require large datasets for effective domain adaptation, making them unsuitable for many medical applications where data scarcity is a fundamental constraint rather than a temporary limitation.

\subsection{Our Contributions}
To address these challenges, we propose PANDA (Pretrained Adaptation Network with Domain Alignment), a comprehensive framework that introduces multiple technical innovations for robust cross-institutional medical AI. Our key contributions include:

\begin{enumerate}
\item \textbf{Cross-Domain Feature Selection with Clinical Consistency}: We introduce a novel recursive feature elimination strategy specifically designed for multi-institutional medical settings. Our approach not only identifies the most discriminative features but also ensures cross-domain availability and clinical interpretability. By systematically reducing 63 heterogeneous clinical variables to an optimal set of 8 features while maintaining predictive performance, we address a critical gap in translating laboratory results to real-world clinical deployment where feature availability varies significantly across institutions.

\item \textbf{Medical-Optimized Tabular Foundation Model Architecture}: We pioneer the application of pre-trained transformer architectures to cross-institutional medical prediction, introducing domain-specific adaptations including ensemble diversity strategies, class imbalance correction mechanisms, and medical-aware feature encoding. Our 32-member ensemble with specialized transformation pipelines represents the first systematic application of foundation model principles to small-sample, imbalanced medical datasets, achieving superior generalization compared to conventional approaches.

\item \textbf{Principled Domain Adaptation via Transfer Component Analysis}: We develop a mathematically rigorous approach to address inter-institutional distribution shift by integrating Transfer Component Analysis with foundation model representations. Our kernel-based domain alignment strategy learns a shared latent space where statistical distributions are aligned while preserving discriminative information, providing theoretical guarantees for cross-domain generalization in medical AI applications.

\item \textbf{Comprehensive Clinical Validation Framework}: We establish a rigorous multi-institutional evaluation protocol using real-world data from two major cancer centers, demonstrating not only superior predictive performance (AUC 0.709, sensitivity 94.4\%) but also clinical utility through calibration analysis and decision curve evaluation. Our validation approach sets new standards for evaluating cross-institutional medical AI systems.

\item \textbf{End-to-End Clinical Translation Pipeline}: We provide a complete framework that addresses the entire pipeline from heterogeneous multi-institutional data to deployable clinical predictions, bridging the gap between research innovation and practical clinical implementation in multi-center healthcare environments.
\end{enumerate}

\subsection{Clinical Impact and Implications}
The PANDA framework addresses fundamental limitations in current medical AI systems by providing a principled approach to cross-institutional deployment. Our results demonstrate that it is possible to achieve robust performance across different healthcare systems using a minimal yet highly informative feature set, making the approach practical for widespread clinical adoption. The high sensitivity achieved (94.4\%) is particularly important for medical screening applications where missing positive cases carries significant clinical consequences.

Furthermore, our framework's ability to maintain performance while using only 8 carefully selected features has important implications for clinical workflow integration and cost-effectiveness. By reducing the dependency on comprehensive laboratory panels or specialized biomarkers that may not be universally available, PANDA enables more equitable access to AI-assisted diagnosis across diverse healthcare settings, from specialized academic centers to community hospitals with more limited resources.

The integration of foundation model capabilities with domain adaptation principles establishes a new paradigm for medical AI development, where models can leverage large-scale pretraining benefits while adapting to the specific challenges of medical deployment scenarios. This approach has broad implications beyond pulmonary nodule classification, potentially transforming how AI systems are developed and deployed across various medical specialties and clinical applications.